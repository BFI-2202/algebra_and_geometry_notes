\documentclass{article}
\usepackage[utf8]{inputenc}

\usepackage[T2A]{fontenc}
\usepackage[utf8]{inputenc}
\usepackage[russian]{babel}

\usepackage{amsmath}
\usepackage{multienum}

\def\vec{\ensuremath\overrightarrow}
\def\proj{\ensuremath\text{Пр}}

\title{Алгебра и геометрия}
\author{Лисид Лаконский}
\date{November 2022}

\begin{document}

\maketitle

\tableofcontents
\pagebreak

\section{Алгебра и геометрия - 01.11.2022}

\subsection{Кривые второго порядка}

\subsubsection{Эллипс}

\begin{flushleft}

Эллипсом называется множество точек $M$ плоскости, сумма расстояний от которых до двух заданных точек $F_1$ и $F_2$, называемых фокусами, есть величина постоянная, большая $F_1 F_2$: $M F_1 + M F_2 = 2a > F_1 F_2$

\hfill

\textbf{Каноническое уравнение эллипса}: $\frac{x^2}{a^2} + \frac{y^2}{b^2} = 1$, где $a$ - большая полуось, $b$ - малая полуось, $a^2 - b^2 = c^2$

\hfill

Эксцентриситет: $\epsilon = \frac{c}{a}$ ($\epsilon < 1$)

Левый фокус: $F_1(-c; 0)$, правый фокус: $F_2(0; c)$, центр эллипса: $O(0; 0)$

Уравнения директрис: $x = \pm \frac{a}{\epsilon}$

\hfill

Если $a < b$, то $b^2 - a^2 = c^2$, $\epsilon = \frac{c}{b}$, $F_1(0; -c)$, $F_2(0; c)$, тогда уравнения директрис: $y = \pm \frac{b}{\epsilon}$

Если $a = b = r$, то $c = 0 \implies F_1(-c; 0) = F_2(c; 0) = O(0; 0)$, $x^2 + y^2 = r^2$, $\epsilon = 0$

\end{flushleft}

\subsubsection{Гипербола}

\begin{flushleft}

Гиперболой называется множество точек $M$ плоскости, модуль разности расстояний от которых до двух заданных точек $F_1$ и $F_2$, есть величина постоянная, меньшая $F_1 F_2$: $| M F_1 - M F_2 | = 2a < F_1 F_2$

\hfill

\textbf{Каноническое уравнение:} $\frac{x^2}{a^2} - \frac{y^2}{b^2} = 1$, где $a$ - действительная полуось, $b$ - мнимая полуось, $a^2 + b^2 = c^2$

\hfill

Эксцентриситет: $\epsilon = \frac{c}{a} > 1$

Левый фокус: $F_1(-c; 0)$, правый фокус: $F_2(c; 0)$, центр: $0(0; 0)$

$x = \pm \frac{a}{\epsilon}$ - уравнения директрис, $y = \pm \frac{b}{a} x$ - уравнения асимптот

\hfill

$-\frac{x^2}{a^2} + \frac{y^2}{b^2} = 1$ - гипербола, сопряженная канонически

$y = \pm \frac{b}{a} x$, $F_1(0; -c)$, $F_2(0; c)$, $\epsilon = \frac{c}{b}$

\subsubsection{Парабола}

Параболой называется множество точек $M$ плоскости, равноудаленных от данной точки $F$ (фокус) и от данной прямой $l$ (директриса)

$M F = r (M, l)$

\hfill

\textbf{Каноническое уравнение (OX):} $y^2 = 2 p x$, $F (\frac{p}{2}; 0)$, $x = -\frac{p}{2}$ - уравнение директрис

\textbf{Каноническое уравнение (OY):} $x^2 = 2p y$, $F(0; \frac{p}{2}$, $y = -\frac{p}{2}$

\end{flushleft}

\subsection{Параллельный перенос координат}

Пусть точка $M$ имеет координаты $(x; y)$ в системе $O x y$ и $(x'; y')$ в системе $O' x' y'$, причем новое начало $O'$ в старой системе имеет координаты $(a; b)$, тогда 

\begin{equation}
    \begin{cases}
        x = x' + a \\
        y = y' + b
    \end{cases}
\end{equation}

\subsection{Примеры решения задач}

\subsubsection{Пример 1}

\begin{flushleft}

Найти экцентриситет, что-то там еще и много чего еще, если $4x^2 - 25y^2 + 50y - 24x - 89 = 0$

\hfill

$4x^2 - 24x = 4(x^2 - 6x) = 4 (x^2 - 6x + 9 - 9) = 4 (x^2 - 6x + 9) - 36 = 4 (x - 3)^2 - 36$

$-25y^2 + 50y = -25 (y^2 - 2y) = -25 (y^2 - 2y + 1 - 1) = -25 (y^2 - 2y + 1) + 25 = -25 (y - 1)^2 + 25$

\hfill

$4 (x - 3)^2 - 36 - 25 (y - 1)^2 + 25 - 89 = 0 \Longleftrightarrow 4 (x - 3)^2 - 25 (y - 1)^2 - 100 = 0 \Longleftrightarrow \frac{(x - 3)^2}{25} - \frac{(y - 1)^2}{4} = 1$ - уравнение гиперболы, $a = 5$, $b = 2$

\hfill

Введем новые координаты: $x - 3 = x'$, $y - 1 = y'$, $O'(3; 1)$

$\frac{(x')^2}{5^2} - \frac{(y')^2}{2^2} = 1$, $a = 5$ - действительная полуось, $b = 2$ - мнимая полуось

\hfill

\textbf{Уравнения директрис:} $x' = \pm \frac{a}{c} = \pm \frac{5 * 5}{\sqrt{29}} = \pm \frac{25}{\sqrt{29}}$, \textbf{уравнения асимтот:} $y' = \pm \frac{b}{a} x' = \pm \frac{2}{5} x'$, \textbf{экцентриситет:} $\frac{c}{a} = \frac{\sqrt{29}}{5}$

$F_1(-\sqrt{29}; 0)$, $F(\sqrt{29}; 0)$

\end{flushleft}

\end{document}