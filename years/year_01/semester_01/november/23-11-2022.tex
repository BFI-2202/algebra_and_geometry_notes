\documentclass{article}
\usepackage[utf8]{inputenc}

\usepackage[T2A]{fontenc}
\usepackage[utf8]{inputenc}
\usepackage[russian]{babel}

\usepackage{amsmath}
\usepackage{multienum}
\usepackage{pgfplots}
\usepackage{tikz}

\def\vec{\ensuremath\overrightarrow}
\def\proj{\ensuremath\text{Пр}}

\title{Алгебра и геометрия}
\author{Лисид Лаконский}
\date{November 2022}

\begin{document}

\maketitle

\tableofcontents
\pagebreak

\section{Алгебра и геометрия - 23.11.2022}

\subsection{Прямая в пространстве}

\begin{flushleft}

Вектор $\vec{S}$, являющийся коллинеарным данной прямой, называется направляющим.

\subsubsection{Уравнения прямой в пространстве}

$\frac{x - x_0}{m} = \frac{y - y_0}{n} = \frac{z - z_0}{p}$, где $M_0(x_0; y_0; z_0)$ - фиксированная точка, лежащая на прямой; $\vec{S} = \{ m; n; p \}$, $m^2 + n^2 + p^2 \ne 0$ - \textbf{каноническое} уравнение прямой.

$\frac{x - x_1}{x_2 - x_1} = \frac{y - y_1}{y_2 - y_1} = \frac{z - z_1}{z_2 - z_1}$ - уравнение прямой \textbf{через две точки} $M_1(x_1; y_1; z_1)$ и $M_2(x_2; y_2; z_2)$

Уравнение прямой, заданной \textbf{параметрически} ($t \in (- \infty; +\infty) $):

\begin{equation}
    \begin{cases}
        x = x_0 + mt \\
        y = y_0 + n t \\
        z = z_0 + pt
    \end{cases}
\end{equation}

Уравнение прямой, полученной в результате \textbf{пересечения двух плоскостей} ($\vec{N_1} = \{ A_1; B_1; C_1 \} \ne \lambda \vec{N_2} = \{ A_2; B_2; C_2 \}$):

\begin{equation}
    \begin{cases}
        A_1 x + B_1 y + C_1 z + D_1 = 0 \\
        A_2 x + B_2 y + C_2 z + D_2 = 0
    \end{cases}
\end{equation}

\subsubsection{Связь между направляющим вектором и векторами нормали}

Направляющий вектор $\vec{S}$ прямой $l$: $\vec{N_1} = \{ A_1; B_1; C_1 \}$ и $\vec{N_2} = \{ A_2; B_2; C_2 \}$: $S = \vec{N_1} \times \vec{N_2}$

\subsubsection{Косинус угла между двумя прямыми}

Даны две прямые: $l_1$: $\frac{x - x_1}{m_1} = \frac{y - y_1}{n_1} = \frac{z - z_1}{p_1}$, $l_2$: $\frac{x - x_2}{m_2} = \frac{y - y_2}{n_2} = \frac{z - z_2}{p_2}$

$\cos (l_1; l_2) = \pm \frac{\vec{S_1} * \vec{S_2}}{|\vec{S_1}| * |\vec{S_2}|} = \pm \frac{m_1 m_2 + n_1 n_2 + p_1 p_2}{\sqrt{m_1^2 + n_1^2 + p_1^2} * \sqrt{m_2^2 + n_2^2 + p_2^2}} \ge 0$

\subsubsection{Угол между прямой и плоскостью}

Даны прямая $l$: $\frac{x - x_0}{m} = \frac{y - y_0}{n} = \frac{z - z_0}{p}$ и плоскость $\alpha$: $Ax + By + C z + D = 0$

$\angle (l; \alpha) = \pm \frac{\vec{S} * \vec{N}}{|\vec{S}| * |\vec{N}|} = \pm \frac{A m + B n + C p}{\sqrt{A^2 + B^2 + C^2} * \sqrt{m^2 + n ^2 + p^2}}$

\subsubsection{Примеры решения задач}

\paragraph{Пример 1.} Составьте канонические и параметрические уравнения высоты $l$, опущенной на плоскость $(B C D)$ в пирамиде $A B C D$, если т. $A(0; 5; -2)$, т. $B(1; 2; -1)$, т. $C(4; 5; 0)$, т. $D(1; 1; 1)$

$\vec{BC} = \{ 3; 3; 1 \}, \vec{B D} = \{ 0; -1; 2 \}, \vec{S} = \vec{BC} \times \vec{B D} = \begin{vmatrix}
    \vec{i} & \vec{j} & \vec{k} \\
    3 & 3 & 1 \\
    0 & -1 & 2
\end{vmatrix} = \{ 7; 6; -3 \}$

Найдем каноническое уравнение: $\frac{x - x_0}{m} = \frac{y - y_0}{n} = \frac{z - z_0}{p} \Longleftrightarrow \frac{x}{7} = \frac{y - y_5}{-6} = \frac{z + 2}{-3}$, $\vec{S} = \{ m; n; p \} = \{ 7; -6; -3 \}$, $M_0(x_0; y_0; z_0) = A(0; 5; -2)$

Найдем параметрическое уравнение данной прямой:

\begin{equation}
    \begin{cases}
        x = x_0 + mt \\
        y = y_0 + nt \\
        z = z_0 + pt
    \end{cases}
    \Longleftrightarrow
    \begin{cases}
        x = 0 + 7t \\
        y = 5 - 6t \\
        z = -2 - 3t
    \end{cases}
\end{equation}

\paragraph{Пример 2.} Составьте канонические и параметрические уравнения прямой, проходящей через точку $A(3; 2; -1)$ перпендикулярно плоскости $x O z$

$\vec{j} = \{ 0; 1; 0 \}$ - вспомогательный вектор.

Найдем каноническое уравнение: $\frac{x - 3}{0} = \frac{y - 2}{1} = \frac{z + 1}{0}$

Найдем параметрическое уравнение данной прямой:

\begin{equation}
    \begin{cases}
        x = 3 \\
        y = 2 + t \\
        z = -1
    \end{cases}
\end{equation}

\paragraph{Пример 3.} Найдите угол между прямыми $l_1$: $\frac{x - 3}{2} = \frac{y + 4}{2} = \frac{z - 5}{-1}$ и $l_2$:

\begin{equation}
    \begin{cases}
        x + y - 2z - 1 = 0 \\
        2x - z + 8 = 0
    \end{cases}
\end{equation}

$\vec{N_1} = \{ 1; 1; -2 \}$, $\vec{N_2} = \{ 2; 0; -1 \}$, $\vec{S_1} = \{ 2; 2; -1 \}$, $\vec{S_2} = \vec{N_1} \times \vec{N_2} = \begin{vmatrix}
    \vec{i} & \vec{j} & \vec{k} \\
    1 & 1 & -2 \\
    2 & 0 & -1
\end{vmatrix} = \{ -1; -3; -2 \} $

$\cos(l_1; l_2) = \pm \frac{\vec{S_1} * \vec{S_2}}{|\vec{S_1}| * |\vec{S_2}|} = \pm \frac{-2 + (-6) + 2}{3 * \sqrt{14}} = \frac{2}{\sqrt{14}} \approx 0.535$, $\angle (l_1; l_2) \approx 57.7^\circ$

\end{flushleft}

\end{document}