\documentclass{article}
\usepackage[utf8]{inputenc}

\usepackage[T2A]{fontenc}
\usepackage[utf8]{inputenc}
\usepackage[russian]{babel}

\usepackage{amsmath}
\usepackage{multienum}
\usepackage{pgfplots}
\usepackage{tikz}

\def\vec{\ensuremath\overrightarrow}
\def\proj{\ensuremath\text{Пр}}

\title{Алгебра и геометрия}
\author{Лисид Лаконский}
\date{November 2022}

\begin{document}

\maketitle

\tableofcontents
\pagebreak

\section{Алгебра и геометрия — 18.11.2022}

\subsection{Разбор контрольной работы №1}

Первые три задания — матрицы и все, что с ними связано; четвертое и пятое — прямые, уравнения прямых, углы между ними; шестое — кривые второго порядка.

\subsubsection{Первое задание}

\paragraph{Разобранный пример}

\begin{flushleft}

Пусть $B = A^{-1}$, где $A = \begin{pmatrix}
    -1 & 0 & 2 & 3 \\
    0 & 1 & -1 & 2 \\
    -2 & 1 & 1 & 3 \\
    0 & -3 & 1 & -1
\end{pmatrix}$

Найдите элемент $b_{31}$ матрицы $B$.

\bigskip

$A^{-1} = \frac{1}{\det A} \begin{pmatrix}
    A_{11} & \dots & A_{n 1} \\
    \dots & \dots & \dots \\
    A_{1 n} & \dots & A_{n n}
\end{pmatrix}$

$A_{i j} = (-1)^{i + j} M_{i j}$

$\Delta A = -1 * \begin{vmatrix}
    1 & -1 & 2 \\
    1 & 1 & 3 \\
    -3 & 1 & -1
\end{vmatrix} + 2 * \begin{vmatrix}
    0 & 1 & 2 \\
    -2 & 1 & 3 \\
    0 & -3 & -1
\end{vmatrix} - 3 * \begin{vmatrix}
    0 & 1 & -1 \\
    -2 & 1 & 1 \\
    0 & -3 & 1
\end{vmatrix} = -1 (-1 + 2 + 9 + 6 - 3 - 1) + 2 (12 - 2) - 3 (-6 + 2) = -12 + 20 + 12 = 20$

\bigskip

$A_{13} = (-1)^{1 + 3} \begin{vmatrix}
    0 & 1 & 2 \\
    -2 & 1 & 3 \\
    0 & -3 & -1
\end{vmatrix} = 1 * 10 = 10$

$b_{31} = \frac{1}{\det A} * A_{13} = \frac{1}{20} * 10 = \frac{1}{2}$

\paragraph{Другие возможные вариации}

\textbf{Возведение матрицы в определенную степень.} Решается перемножением матрицы самой на себя. 

\pagebreak
\subsubsection{Второе задание}

\paragraph{Разобранный пример}

Найдите общее решение системы

\begin{equation}
    \begin{cases}
        x_1 - 2x_2 - 4x_3 = -7 \\
        2x_1 + x_2 - 3x_3 = 1 \\
        x_1 + 8x_2 + 6x_3 = 23
    \end{cases}
\end{equation}

Распишем расширенную матрицу системы и упростим ее, чтобы решить систему методом Гаусса:

$A = \begin{pmatrix}
    1 & -2 & -4 \\
    2 & 1 & -3 \\
    1 & 8 & 6
\end{pmatrix}$, $B = \begin{pmatrix}
    -7 \\
    1 \\
    23
\end{pmatrix}$, $(A|B) = \begin{pmatrix}
    1 & -2 & -4 & -7 \\
    2 & 1 & -3 & 1 \\
    1 & 8 & 6 & 23
\end{pmatrix} \approx \begin{pmatrix}
    1 & -2 & -4 & -7 \\
    0 & 5 & 5 & 15 \\
    0 & 10 & 10 & 30
\end{pmatrix} \approx \begin{pmatrix}
    1 & -2 & -4 & -7 \\
    0 & 1 & 1 & 3
\end{pmatrix} \approx \begin{pmatrix}
    1 & 0 & -2 & -1 \\
    0 & 1 & 1 & 3
\end{pmatrix}$

Данная матрица эквивалентна следующей системе уравнений:

\begin{equation}
    \begin{cases}
        x_1 - 2x_3 = -1 \\
        x_2 + x_3 = 3
    \end{cases}
\end{equation}

Пусть $x_3 = 1$, тогда $x_1 = 1$, $x_2 = 2$, решением системы будет $X = c \begin{pmatrix}
    1 \\
    2 \\
    1
\end{pmatrix}$, $c \in R$

\pagebreak
\subsubsection{Третье задание}

\paragraph{Разобранный пример}

Решите матричное уравнение

$\begin{pmatrix}
    2 & -1 \\
    1 & 0
\end{pmatrix} * X = \begin{pmatrix}
    3 & -2 & -1 \\
    1 & -1 & 0
\end{pmatrix}$

\bigskip

Вспоминаем, что

$A * X = B$, $X = A^{-1} * B$

Нам нужно найти обратную матрицу и умножить ее на матрицу $B$

\bigskip

$\Delta A = 1$

$A_{11} = 1 * 0 = 0 , A_{12} = -1 * 1 = -1, A_{21} = -1 * (-1) = 1, A_{22} = 1 * 2 = 2$

$A^{-1} = \begin{pmatrix}
    0 & 1 \\
    -1 & 2
\end{pmatrix}$

\bigskip

$X = A^{-1} * B = \begin{pmatrix}
    0 & 1 \\
    -1 & 2
\end{pmatrix} \begin{pmatrix}
    3 & -2 & -1 \\
    1 & -1 & 0
\end{pmatrix} = \begin{pmatrix}
    1 & -1 & 0 \\
    -1 & 0 & 1
\end{pmatrix}$

\paragraph{Другие возможные вариации}

\textbf{Нахождение элементов исходной матрицы из обратной.} Вспоминаем, что $A * A^{-1} = E$ — таким образом, для решения нужно умножить обратную матрицу на единичную — получить исходную матрицу — указать тот элемент, который нас просят найти.

\pagebreak
\subsubsection{Четвертое задание}

\paragraph{Разобранный пример}

Найдите точку, симметричную точке $A(2; -1)$ относительно прямой $y = x - 1$

\def\size{5} % natural number

\begin{tikzpicture}
  \begin{axis}[grid=both,ymin=-5,ymax=5,xmax=5,xmin=-5,xticklabel=\empty,yticklabel=\empty,
               minor tick num=1,axis lines = middle,xlabel=$x$,ylabel=$y$,label style =
               {at={(ticklabel cs:1.1)}}]
    \addplot[color=red]{x - 1};
    \addlegendentry{\(x - 1\)}
    \addplot[color=blue]{-x + 1};
    \addlegendentry{\(-x + 1\)}
    \addplot[mark=*] coordinates {(2,-1)} node[pin={[pin distance=1mm]1:{$A$}}]{} ;
    \addplot[mark=*] coordinates {(1,0)} node[pin={[pin distance=1mm]90:{$C$}}]{} ;
    \addplot[mark=*] coordinates {(0,1)} node[pin={[pin distance=1mm]180:{$B$}}]{} ;
  \end{axis}
\end{tikzpicture}

Если мы ищем симметричную точку, то она должна быть расположена на линии, перпендикулярной нашей прямой по обратную сторону на том же самом расстоянии.

Исходя из уравнений наших прямых мы имеем две прямые: проходящую через нашу точку $A$, и перпендикулярную ей.

\bigskip

$l_1 \perp l_2$, $k_2 = -\frac{1}{k_1}$

$y = x - 1 \implies k_1 = 1$, $k_2 = -1$

$y - y_0 = k(x - x_0)$

$y + 1 = -1 (x - 2)$, $y = -x + 1$ — уравнение прямой, перпендикулярной исходной

\bigskip

Найдем точку пересечения двух линий:

\begin{equation}
    \begin{cases}
        y = x - 1 \\
        y = -x + 1
    \end{cases}
\end{equation}

Точка пересечения — $C(1; 0)$

От точки $C$ до точки $A$ и искомой точки $B$ расстояние одинаково. Таким образом, $C$ является центром масс. Понимая это, найдем координаты искомой точки $B$:

\hspace{5mm} $x_c = \frac{x_a + x_b}{2}$, $y_c = \frac{y_a + y_b}{2}$

Координаты точки $B$ — $B(0; 1)$

\pagebreak
\subsubsection{Пятое задание}

\paragraph{Разобранный пример}

Найдите косинус угла между прямыми $2x - 3y + 1 = 0$ и $3x - 2y + 5 = 0$

Найдем угловые коэффициенты. Но для начала приведем эти уравнения к нормальному виду:

$y = \frac{-2x - 1}{-3} = \frac{2}{3}x + \frac{1}{3}$, $y = \frac{-3x - 5}{-2} = \frac{3}{2}x + \frac{5}{2}$

\bigskip

$\tg \alpha = \frac{k}{1 + k_1 k_2} = \pm \frac{k_2 - k_1}{1 + k_1 k_2} \ge 0$

$\tg \alpha = \pm \frac{\frac{3}{2} - \frac{2}{3}}{1 + 1} \ge 0$

$\tg \alpha = \frac{5}{12}$

\bigskip

Для того, чтобы найти синус и косинус, посчитаем $\sqrt{5^2 + 12^2} = \sqrt{25 + 144} = \sqrt{169} = 13$

Синус будет равен $\frac{5}{13}$, а косинус $\frac{12}{13}$

\pagebreak
\subsubsection{Шестое задание}

\paragraph{Разобранный пример}

Найдите расстояние между фокусами кривой $5x^2 - 4y^2 - 10x - 16y - 31 = 0$

Выведем полные квадраты, найдем $A$ и $B$, найдем $c$, и далее найдем расстояние между ними:

\bigskip

$5x - 10x = 5(x^2 - 2x + 1 - 1) = 5(x-1)^2 - 5$

$-4y^2 - 16y = -4(y^2 + 4y) = -4(y^2 + 4y + 4 - 4) = -4(y + 2)^2 + 16$

$5 (x - 1)^2 - 5 - 4(y + 2)^2 + 16 - 31 = 0$

$5 (x - 1)^2 - 4(y + 2)^2 = 20$

В каноническом виде мы должны получать после равно единицу, следовательно нам нужно поделить левую и правую часть на двадцать:

$\frac{(x - 1)^2}{4} - \frac{(y+2)^2}{5} = 1$ — каноническое уравнение гиперболы

$c^2 = a^2 + b^2$, $c = \sqrt{a^2 + b^2} = \sqrt{4 + 5} = \sqrt{9} = 3$

\bigskip

Следовательно, мы имеем два фокуса: $F_1(-3; 0)$, $F_2(3; 0)$. Расстояние между ними — 6

\pagebreak
\subsubsection{Другие возможные вариации}

\paragraph{Вариация 1}. Матрица $B = A^2$, дана матрица $A$, нужно найти какой-то элемент матрицы $B$. Следовательно, мы умножаем матрицу $A$ саму на себя и ищем этот элемент.

\paragraph{Вариация 2}. Решить систему и найти значение линейной комбинации — нужно найти методом Крамера свободные переменные, базисные переменные, выразить базисные через свободные и подставить в линейную комбинацию — найти численное значение этой комбинации.

\paragraph{Вариация 3}. Имеем матрицу $X$, умноженную на другую матрицу, в результате чего мы получаем еще другую матрицу.

\paragraph{Другие вариации}

Составьте уравнение прямой, проходящей через точку перпендикулярно прямой; найти расстояние от точки до прямой; найти эксцентриситет кривой; найти алгебраическое дополнение для квадратной матрицы; найти общее решение системы; даны три вершины квадрата, найти координаты четвертой вершины; составьте уравнения биссектрис углов, образованных прямыми; найдите расстояние между директрисами кривой; при каких значениях $x$, $y$ и $z$ обратная матрица будет обратной для исходной матрицы — перемножить эту матрицу с обратной, получить по свойствам обратных матриц единичную — понять, при каких значениях эта единичная матрица будет образовываться; найти все значения $a$, при которой система является совместной — то есть, имеет решения — либо через теорему Кронекера-Капелли, либо через решение; найти расстояние между прямыми; найти котангенс угла в треугольнике с вершинами; найти координаты фокуса параболы; заданы матрицы, найти определитель матрицы $3A - 2B$; найти какой-либо элемент обратной матрицы — умножить эту обратную матрицу на единичную, получить исходную — найти в ней искомый элемент;

\end{flushleft}

\end{document}