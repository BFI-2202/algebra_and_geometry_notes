\documentclass{article}
\usepackage[utf8]{inputenc}

\usepackage[T2A]{fontenc}
\usepackage[utf8]{inputenc}
\usepackage[russian]{babel}

\usepackage{amsmath}
\usepackage{multienum}

\def\vec{\ensuremath\overrightarrow}
\def\proj{\ensuremath\text{Пр}}

\title{Алгебра и геометрия}
\author{Лисид Лаконский}
\date{November 2022}

\begin{document}

\maketitle

\tableofcontents
\pagebreak

\section{Алгебра и геометрия - 12.11.2022}

\subsection{Векторное произведение векторов}

\begin{flushleft}

\textbf{Векторным произведением} векторов $\vec{a}$ и $\vec{b}$ называется вектор $\vec{c}$, удовлетворяющий следующим условиям:

\begin{enumerate}
    \item $|\vec{c}| = |\vec{a}| |\vec{b}| \sin \alpha$, где $\alpha$ - угол между векторами $\vec{a}$ и $\vec{b}$
    \item $\vec{c} \perp \vec{a}$, $\vec{c} \perp \vec{b}$
    \item $\vec{a}, \vec{b}, \vec{c}$ образуют правую тройку
\end{enumerate}

Если \textbf{один из векторов нулевой} или \textbf{эти векторы параллельны}, то их векторное произведение - тоже нулевой вектор.

\paragraph{Геометрическая интерпретация}

Длина вектора векторного произведения численно равна \textbf{площади параллелограмма}, построенного на этих векторах.

\paragraph{Свойства векторного произведения}

\begin{multienumerate}
    \mitemxx{$[ a \times b] = -[b \times a]$}{$[(\alpha a) \times b] = \alpha [a \times b]$}
    \mitemxx{$[(a + b) \times c] = [a \times c] + [b \times c]$}{$a \times a = 0$}
\end{multienumerate}

\paragraph{Запись в виде определителя}

\hfill

$\vec{a} = \{ x_1, y_1, z_1 \}, \vec{b} = \{ x_2, y_2, z_2 \}$, $\vec{a} \times \vec{b} = \begin{vmatrix}
    i & j & k \\
    x_1 & y_1 & z_1 \\
    x_2 & y_2 & z_2
\end{vmatrix} = \vec{i} \begin{vmatrix}
    y_1 & z_1 \\
    y_2 & z_2
\end{vmatrix} - \vec{j} \begin{vmatrix}
    x_1 & z_1 \\
    x_2 & z_2
\end{vmatrix} + \vec{k} \begin{vmatrix}
    x_1 & y_1 \\
    x_2 & y_2
\end{vmatrix}$

\paragraph{Примеры}

\hfill

\textbf{Пример 1.} Дано: $|\vec{a}| = 3, |\vec{b}| = 4, \phi = 90^{\circ}$. Найти $|(3\vec{a} - \vec{b}) \times (\vec{a} - 2\vec{b})|$

$|(3\vec{a} - \vec{b}) \times (\vec{a} - 2\vec{b})| = |3\vec{a} \times \vec{a} - \vec{b} \times \vec{a} - 3\vec{a} \times 2 \vec{b} + \vec{b} \times 2 \vec{b}| = |-\vec{b} \times \vec{a} + 6\vec{b} \times \vec{a}| = |5\vec{b} \times \vec{a}| = 5 |\vec{b}| |\vec{a}| = 5 * 3 * 4 = 60$

\hfill

\textbf{Пример 2.} Дано: $A(1; 2; 0), B(3; 0; -3), C(5; 2; 6)$. Найти площадь треугольника, образуемого этими точками.

$\vec{AB} = \{ 2; 2; -3 \}, \vec{AC} = \{4; 0; 6 \}, \vec{AB} \times \vec{AC} = \begin{vmatrix}
    i & j & k \\
    2 & -2 & -3 \\
    4 & 0 & 6
\end{vmatrix} = \vec{i} \begin{vmatrix}
    -2 & -3 \\
    0 & 6
\end{vmatrix} - \vec{j} \begin{vmatrix}
    2 & -3 \\
    4 & 6
\end{vmatrix} + \vec{k} \begin{vmatrix}
    2 & -2 \\
    4 & 0
\end{vmatrix} = -12\vec{i} - 24\vec{j} + 8\vec{k}, |\{ -12; -24; 8 \}| = \sqrt{144 + 576 + 64} = 28$

Искомая площадь треугольника: $S = \frac{1}{2} |\{ -12; -24; 8 \}| = 14$

\hfill

\textbf{Пример 3}

Дано: $\vec{x} \perp \vec{a}, \vec{x} \perp \vec{b}, \vec{a} = \{ 4; -2; 3\}, \vec{b} = \{ 0; 1; 3 \}, |\vec{x}| = 26$, найти координаты данного $\vec{x}$, образующего с осью OY тупой угол.

$\vec{a} \times \vec{b} = \begin{vmatrix}
    i & j & k \\
    4 & -2 & 3 \\
    0 & 1 & 3
\end{vmatrix} = -3\vec{i} - 12\vec{j} + 4\vec{k}, \vec{x} = \{ 3 \lambda; 12 \lambda; -4 \lambda \}, |x| = \sqrt{9\lambda^2 + 144\lambda^2 + 16\lambda^2} = 13|\lambda|, \lambda = \pm 2$

$\vec{x} = \{ -6; -24; 8 \}$ - \textbf{ответ, ибо именно он образует тупой угол}, или $\vec{x} = \{ 6; 24; -8 \}$

\pagebreak
\subsection{Смешанное произведение трех векторов}

\textbf{Смешанным произведением трех векторов} называется \textbf{число}, которое получается, если вектор $\vec{a}$ умножить векторно на $\vec{b}$, а потом результат этого произведения скалярно умножить на $c$:

$$\vec{a} \vec{b} \vec{c} = ([\vec{a} \times \vec{b}]) \vec{c}$$

$\vec{a} \vec{b} \vec{c} > 0$ - если тройка векторов правая, $\vec{a} \vec{b} \vec{c} < 0$ - если тройка векторов левая.

\paragraph{Циклическая перестановка}

$\vec{a} \vec{b} \vec{c} = \vec{c} \vec{a} \vec{b} = \vec{b} \vec{c} \vec{a}$

Если же перестановка соседних векторов, то происходит смена знака: $\vec{a} \vec{b} \vec{c} = - \vec{b} \vec{a} \vec{c}$

\paragraph{Запись в виде определителя}

$$\vec{a} \vec{b} \vec{c} = \begin{vmatrix}
    x_1 & y_1 & z_1 \\
    x_2 & y_2 & z_2 \\
    x_3 & y_3 & z_3
\end{vmatrix}$$

\paragraph{Условие компланарности} Условие компланарности трех векторов - их смешанное произведение равно нулю.

\paragraph{Геометрическая интерпретация} Смешанное произведение векторов по модулю равно \textbf{объему параллелепипеда}, построенного на этих векторах.

$V_{\text{пир}} = \frac{1}{6} V_{\text{пар}}$

Если нам необходимо найти высоту чего-то там, то $h = \frac{|a b c|}{|[a \times b]|}$

\paragraph{Примеры}

\hfill

\textbf{Пример 1.} Дано: $\vec{a} = \{ 2; 3; 1 \}, \vec{b} = \{ 1; -1; 3 \}, \vec{c} = \{ 1; 9; -11 \}$. Определить, компланарны ли данные векторы.

$\vec{a} \vec{b} \vec{c} = \begin{vmatrix}
    2 & 3 & -1 \\
    1 & -1 & 3 \\
    1 & 9 & -11
\end{vmatrix} = -\begin{vmatrix}
    1 & -1 & 3 \\
    2 & 3 & -1 \\
    1 & 9 & -11
\end{vmatrix} = -\begin{vmatrix}
    1 & -1 & 3 \\
    0 & 5 & -7 \\
    0 & 10 & -14
\end{vmatrix} = -\begin{vmatrix}
    1 & -1 & 3 \\
    0 & 5 & -7
\end{vmatrix}$

Таким образом, данные три вектора \textbf{компланарны}

\hfill

\textbf{Пример 2.} Дано: $A(2; -1; 1), B(5; 5; 4), C(3; 2; -1), D(4; 1; 3)$. Найти объем тетраэдра, образуемого данными точками.

$\vec{AB} = \{ 3; 6; 3 \}, \vec{AC} = \{ 1; 3; -2 \}, \vec{AD} = \{ 2; 2; 2 \}$

$\vec{AB} \vec{AC} \vec{AD} = \begin{vmatrix}
    3 & 6 & 3 \\
    1 & 3 & -2 \\
    2 & 2 & 2
\end{vmatrix} = 3 * 2 \begin{vmatrix}
    1 & 2 & 1 \\
    1 & 3 & -2 \\
    1 & 1 & 1
\end{vmatrix} = 3 * 2 \begin{vmatrix}
    1 & 2 & 1 \\
    0 & 1 & -3 \\
    0 & -1 & 0
\end{vmatrix} = 3 * 2 \begin{vmatrix}
    1 & 2 & 1 \\
    0 & 1 & -2 \\
    0 & 0 & -3
\end{vmatrix} = -18$

\textbf{Объем тетраэдра}: $S = \frac{1}{8} * |-18| = 3$

\pagebreak
\subsection{Двойное векторное произведение}

Пусть $\vec{a}$ умножается векторно на $\vec{b}$, а полученный вектор векторно умножается на $\vec{c}$ - результат называется \textbf{двойным векторным произведением}: $[ [ \vec{a} \times \vec{b} ] \times \vec{c}] = [ [ \vec{a} \vec{b} ] \vec{c}] \ne [ \vec{a} \times [ \vec{b} \times \vec{c} ] ]$

$[ [ \vec{a} \times \vec{b} ] \times \vec{c} ] = \vec{b} (\vec{a} \vec{c}) - \vec{a} (\vec{b} \vec{c})$

\paragraph{Примеры}

\hfill

Допустим, имеем $\vec{a} = \{ x_1; 0; 0 \}, \vec{b} = \{ x_2; y_2; 0 \}, \vec{c} = \{ x_3; y_3; z_3 \}$, тогда

$[ \vec{a} \times \vec{b} ] = \begin{vmatrix}
    i & j & k \\
    x_1 & 0 & 0 \\
    x_2 & y_2 & 0
\end{vmatrix} = \vec{i} * 0 - \vec{j} * 0 + \vec{k} \begin{vmatrix}
    x_1 & 0 \\
    x_2 & y_2
\end{vmatrix} = \{ 0; 0; x_1 y_2 \}$

$[ [ \vec{a} \times \vec{b} ] \times \vec{c} ] = \begin{vmatrix}
    i & j & k \\
    0 & 0 & x_1 y_2 \\
    x_3 & y_3 & z_3
\end{vmatrix} = \vec{i} (0 - x_1 y_2 y_3) - \vec{j} (0 - x_3 x_1 y_2 ) + \vec{k} * 0 = \{ -x_1 y_2 y_3; x_3 x_1 y_2; 0 \}$

$a * c = x_1 x_3, b * c = x_2 x_3 + y_2 y_3, \vec{b} (a * c) = \{x_1 x_2 x_3; x_1 y_2 x_3; 0 \}, \vec{a} (\vec{b} \vec{c}) = \{ x_1 (x_2 x_3 + y_2 y_3); 0; 0 \}$

$\vec{b} (\vec{a} \vec{c}) - \vec{a} (\vec{b} \vec{c}) = \{ x_1 x_2 x_3 - x_1 (x_2 x_3 + y_2 y_3); x_1 y_2 x_3; 0 \}$

\end{flushleft}

\end{document}