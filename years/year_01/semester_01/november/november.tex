\documentclass{article}
\usepackage[utf8]{inputenc}

\usepackage[T2A]{fontenc}
\usepackage[utf8]{inputenc}
\usepackage[russian]{babel}

\usepackage{amsmath}
\usepackage{multienum}
\usepackage{pgfplots}
\usepackage{tikz}

\def\vec{\ensuremath\overrightarrow}
\def\proj{\ensuremath\text{Пр}}

\title{Алгебра и геометрия}
\author{Лисид Лаконский}
\date{November 2022}

\begin{document}

\maketitle

\tableofcontents
\pagebreak

\section{Алгебра и геометрия - 01.11.2022}

\subsection{Кривые второго порядка}

\subsubsection{Эллипс}

\begin{flushleft}

Эллипсом называется множество точек $M$ плоскости, сумма расстояний от которых до двух заданных точек $F_1$ и $F_2$, называемых фокусами, есть величина постоянная, большая $F_1 F_2$: $M F_1 + M F_2 = 2a > F_1 F_2$

\hfill

\textbf{Каноническое уравнение эллипса}: $\frac{x^2}{a^2} + \frac{y^2}{b^2} = 1$, где $a$ - большая полуось, $b$ - малая полуось, $a^2 - b^2 = c^2$

\hfill

Эксцентриситет: $\epsilon = \frac{c}{a}$ ($\epsilon < 1$)

Левый фокус: $F_1(-c; 0)$, правый фокус: $F_2(0; c)$, центр эллипса: $O(0; 0)$

Уравнения директрис: $x = \pm \frac{a}{\epsilon}$

\hfill

Если $a < b$, то $b^2 - a^2 = c^2$, $\epsilon = \frac{c}{b}$, $F_1(0; -c)$, $F_2(0; c)$, тогда уравнения директрис: $y = \pm \frac{b}{\epsilon}$

Если $a = b = r$, то $c = 0 \implies F_1(-c; 0) = F_2(c; 0) = O(0; 0)$, $x^2 + y^2 = r^2$, $\epsilon = 0$

\end{flushleft}

\subsubsection{Гипербола}

\begin{flushleft}

Гиперболой называется множество точек $M$ плоскости, модуль разности расстояний от которых до двух заданных точек $F_1$ и $F_2$, есть величина постоянная, меньшая $F_1 F_2$: $| M F_1 - M F_2 | = 2a < F_1 F_2$

\hfill

\textbf{Каноническое уравнение:} $\frac{x^2}{a^2} - \frac{y^2}{b^2} = 1$, где $a$ - действительная полуось, $b$ - мнимая полуось, $a^2 + b^2 = c^2$

\hfill

Эксцентриситет: $\epsilon = \frac{c}{a} > 1$

Левый фокус: $F_1(-c; 0)$, правый фокус: $F_2(c; 0)$, центр: $0(0; 0)$

$x = \pm \frac{a}{\epsilon}$ - уравнения директрис, $y = \pm \frac{b}{a} x$ - уравнения асимптот

\hfill

$-\frac{x^2}{a^2} + \frac{y^2}{b^2} = 1$ - гипербола, сопряженная канонически

$y = \pm \frac{b}{a} x$, $F_1(0; -c)$, $F_2(0; c)$, $\epsilon = \frac{c}{b}$

\subsubsection{Парабола}

Параболой называется множество точек $M$ плоскости, равноудаленных от данной точки $F$ (фокус) и от данной прямой $l$ (директриса)

$M F = r (M, l)$

\hfill

\textbf{Каноническое уравнение (OX):} $y^2 = 2 p x$, $F (\frac{p}{2}; 0)$, $x = -\frac{p}{2}$ - уравнение директрис

\textbf{Каноническое уравнение (OY):} $x^2 = 2p y$, $F(0; \frac{p}{2}$, $y = -\frac{p}{2}$

\end{flushleft}

\subsection{Параллельный перенос координат}

Пусть точка $M$ имеет координаты $(x; y)$ в системе $O x y$ и $(x'; y')$ в системе $O' x' y'$, причем новое начало $O'$ в старой системе имеет координаты $(a; b)$, тогда 

\begin{equation}
    \begin{cases}
        x = x' + a \\
        y = y' + b
    \end{cases}
\end{equation}

\subsection{Примеры решения задач}

\subsubsection{Пример 1}

\begin{flushleft}

Найти экцентриситет, что-то там еще и много чего еще, если $4x^2 - 25y^2 + 50y - 24x - 89 = 0$

\hfill

$4x^2 - 24x = 4(x^2 - 6x) = 4 (x^2 - 6x + 9 - 9) = 4 (x^2 - 6x + 9) - 36 = 4 (x - 3)^2 - 36$

$-25y^2 + 50y = -25 (y^2 - 2y) = -25 (y^2 - 2y + 1 - 1) = -25 (y^2 - 2y + 1) + 25 = -25 (y - 1)^2 + 25$

\hfill

$4 (x - 3)^2 - 36 - 25 (y - 1)^2 + 25 - 89 = 0 \Longleftrightarrow 4 (x - 3)^2 - 25 (y - 1)^2 - 100 = 0 \Longleftrightarrow \frac{(x - 3)^2}{25} - \frac{(y - 1)^2}{4} = 1$ - уравнение гиперболы, $a = 5$, $b = 2$

\hfill

Введем новые координаты: $x - 3 = x'$, $y - 1 = y'$, $O'(3; 1)$

$\frac{(x')^2}{5^2} - \frac{(y')^2}{2^2} = 1$, $a = 5$ - действительная полуось, $b = 2$ - мнимая полуось

\hfill

\textbf{Уравнения директрис:} $x' = \pm \frac{a}{c} = \pm \frac{5 * 5}{\sqrt{29}} = \pm \frac{25}{\sqrt{29}}$, \textbf{уравнения асимтот:} $y' = \pm \frac{b}{a} x' = \pm \frac{2}{5} x'$, \textbf{экцентриситет:} $\frac{c}{a} = \frac{\sqrt{29}}{5}$

$F_1(-\sqrt{29}; 0)$, $F(\sqrt{29}; 0)$

\end{flushleft}

\pagebreak
\section{Алгебра и геометрия - 12.11.2022}

\subsection{Векторное произведение векторов}

\begin{flushleft}

\textbf{Векторным произведением} векторов $\vec{a}$ и $\vec{b}$ называется вектор $\vec{c}$, удовлетворяющий следующим условиям:

\begin{enumerate}
    \item $|\vec{c}| = |\vec{a}| |\vec{b}| \sin \alpha$, где $\alpha$ - угол между векторами $\vec{a}$ и $\vec{b}$
    \item $\vec{c} \perp \vec{a}$, $\vec{c} \perp \vec{b}$
    \item $\vec{a}, \vec{b}, \vec{c}$ образуют правую тройку
\end{enumerate}

Если \textbf{один из векторов нулевой} или \textbf{эти векторы параллельны}, то их векторное произведение - тоже нулевой вектор.

\paragraph{Геометрическая интерпретация}

Длина вектора векторного произведения численно равна \textbf{площади параллелограмма}, построенного на этих векторах.

\paragraph{Свойства векторного произведения}

\begin{multienumerate}
    \mitemxx{$[ a \times b] = -[b \times a]$}{$[(\alpha a) \times b] = \alpha [a \times b]$}
    \mitemxx{$[(a + b) \times c] = [a \times c] + [b \times c]$}{$a \times a = 0$}
\end{multienumerate}

\paragraph{Запись в виде определителя}

\hfill

$\vec{a} = \{ x_1, y_1, z_1 \}, \vec{b} = \{ x_2, y_2, z_2 \}$, $\vec{a} \times \vec{b} = \begin{vmatrix}
    i & j & k \\
    x_1 & y_1 & z_1 \\
    x_2 & y_2 & z_2
\end{vmatrix} = \vec{i} \begin{vmatrix}
    y_1 & z_1 \\
    y_2 & z_2
\end{vmatrix} - \vec{j} \begin{vmatrix}
    x_1 & z_1 \\
    x_2 & z_2
\end{vmatrix} + \vec{k} \begin{vmatrix}
    x_1 & y_1 \\
    x_2 & y_2
\end{vmatrix}$

\paragraph{Примеры}

\hfill

\textbf{Пример 1.} Дано: $|\vec{a}| = 3, |\vec{b}| = 4, \phi = 90^{\circ}$. Найти $|(3\vec{a} - \vec{b}) \times (\vec{a} - 2\vec{b})|$

$|(3\vec{a} - \vec{b}) \times (\vec{a} - 2\vec{b})| = |3\vec{a} \times \vec{a} - \vec{b} \times \vec{a} - 3\vec{a} \times 2 \vec{b} + \vec{b} \times 2 \vec{b}| = |-\vec{b} \times \vec{a} + 6\vec{b} \times \vec{a}| = |5\vec{b} \times \vec{a}| = 5 |\vec{b}| |\vec{a}| = 5 * 3 * 4 = 60$

\hfill

\textbf{Пример 2.} Дано: $A(1; 2; 0), B(3; 0; -3), C(5; 2; 6)$. Найти площадь треугольника, образуемого этими точками.

$\vec{AB} = \{ 2; 2; -3 \}, \vec{AC} = \{4; 0; 6 \}, \vec{AB} \times \vec{AC} = \begin{vmatrix}
    i & j & k \\
    2 & -2 & -3 \\
    4 & 0 & 6
\end{vmatrix} = \vec{i} \begin{vmatrix}
    -2 & -3 \\
    0 & 6
\end{vmatrix} - \vec{j} \begin{vmatrix}
    2 & -3 \\
    4 & 6
\end{vmatrix} + \vec{k} \begin{vmatrix}
    2 & -2 \\
    4 & 0
\end{vmatrix} = -12\vec{i} - 24\vec{j} + 8\vec{k}, |\{ -12; -24; 8 \}| = \sqrt{144 + 576 + 64} = 28$

Искомая площадь треугольника: $S = \frac{1}{2} |\{ -12; -24; 8 \}| = 14$

\hfill

\textbf{Пример 3}

Дано: $\vec{x} \perp \vec{a}, \vec{x} \perp \vec{b}, \vec{a} = \{ 4; -2; 3\}, \vec{b} = \{ 0; 1; 3 \}, |\vec{x}| = 26$, найти координаты данного $\vec{x}$, образующего с осью OY тупой угол.

$\vec{a} \times \vec{b} = \begin{vmatrix}
    i & j & k \\
    4 & -2 & 3 \\
    0 & 1 & 3
\end{vmatrix} = -3\vec{i} - 12\vec{j} + 4\vec{k}, \vec{x} = \{ 3 \lambda; 12 \lambda; -4 \lambda \}, |x| = \sqrt{9\lambda^2 + 144\lambda^2 + 16\lambda^2} = 13|\lambda|, \lambda = \pm 2$

$\vec{x} = \{ -6; -24; 8 \}$ - \textbf{ответ, ибо именно он образует тупой угол}, или $\vec{x} = \{ 6; 24; -8 \}$

\subsection{Смешанное произведение трех векторов}

\textbf{Смешанным произведением трех векторов} называется \textbf{число}, которое получается, если вектор $\vec{a}$ умножить векторно на $\vec{b}$, а потом результат этого произведения скалярно умножить на $c$:

$$\vec{a} \vec{b} \vec{c} = ([\vec{a} \times \vec{b}]) \vec{c}$$

$\vec{a} \vec{b} \vec{c} > 0$ - если тройка векторов правая, $\vec{a} \vec{b} \vec{c} < 0$ - если тройка векторов левая.

\paragraph{Циклическая перестановка}

$\vec{a} \vec{b} \vec{c} = \vec{c} \vec{a} \vec{b} = \vec{b} \vec{c} \vec{a}$

Если же перестановка соседних векторов, то происходит смена знака: $\vec{a} \vec{b} \vec{c} = - \vec{b} \vec{a} \vec{c}$

\paragraph{Запись в виде определителя}

$$\vec{a} \vec{b} \vec{c} = \begin{vmatrix}
    x_1 & y_1 & z_1 \\
    x_2 & y_2 & z_2 \\
    x_3 & y_3 & z_3
\end{vmatrix}$$

\paragraph{Условие компланарности} Условие компланарности трех векторов - их смешанное произведение равно нулю.

\paragraph{Геометрическая интерпретация} Смешанное произведение векторов по модулю равно \textbf{объему параллелепипеда}, построенного на этих векторах.

$V_{\text{пир}} = \frac{1}{6} V_{\text{пар}}$

Если нам необходимо найти высоту чего-то там, то $h = \frac{|a b c|}{|[a \times b]|}$

\paragraph{Примеры}

\hfill

\textbf{Пример 1.} Дано: $\vec{a} = \{ 2; 3; 1 \}, \vec{b} = \{ 1; -1; 3 \}, \vec{c} = \{ 1; 9; -11 \}$. Определить, компланарны ли данные векторы.

$\vec{a} \vec{b} \vec{c} = \begin{vmatrix}
    2 & 3 & -1 \\
    1 & -1 & 3 \\
    1 & 9 & -11
\end{vmatrix} = -\begin{vmatrix}
    1 & -1 & 3 \\
    2 & 3 & -1 \\
    1 & 9 & -11
\end{vmatrix} = -\begin{vmatrix}
    1 & -1 & 3 \\
    0 & 5 & -7 \\
    0 & 10 & -14
\end{vmatrix} = -\begin{vmatrix}
    1 & -1 & 3 \\
    0 & 5 & -7
\end{vmatrix}$

Таким образом, данные три вектора \textbf{компланарны}

\hfill

\textbf{Пример 2.} Дано: $A(2; -1; 1), B(5; 5; 4), C(3; 2; -1), D(4; 1; 3)$. Найти объем тетраэдра, образуемого данными точками.

$\vec{AB} = \{ 3; 6; 3 \}, \vec{AC} = \{ 1; 3; -2 \}, \vec{AD} = \{ 2; 2; 2 \}$

$\vec{AB} \vec{AC} \vec{AD} = \begin{vmatrix}
    3 & 6 & 3 \\
    1 & 3 & -2 \\
    2 & 2 & 2
\end{vmatrix} = 3 * 2 \begin{vmatrix}
    1 & 2 & 1 \\
    1 & 3 & -2 \\
    1 & 1 & 1
\end{vmatrix} = 3 * 2 \begin{vmatrix}
    1 & 2 & 1 \\
    0 & 1 & -3 \\
    0 & -1 & 0
\end{vmatrix} = 3 * 2 \begin{vmatrix}
    1 & 2 & 1 \\
    0 & 1 & -2 \\
    0 & 0 & -3
\end{vmatrix} = -18$

\textbf{Объем тетраэдра}: $S = \frac{1}{8} * |-18| = 3$

\subsection{Двойное векторное произведение}

Пусть $\vec{a}$ умножается векторно на $\vec{b}$, а полученный вектор векторно умножается на $\vec{c}$ - результат называется \textbf{двойным векторным произведением}: $[ [ \vec{a} \times \vec{b} ] \times \vec{c}] = [ [ \vec{a} \vec{b} ] \vec{c}] \ne [ \vec{a} \times [ \vec{b} \times \vec{c} ] ]$

$[ [ \vec{a} \times \vec{b} ] \times \vec{c} ] = \vec{b} (\vec{a} \vec{c}) - \vec{a} (\vec{b} \vec{c})$

\paragraph{Примеры}

\hfill

Допустим, имеем $\vec{a} = \{ x_1; 0; 0 \}, \vec{b} = \{ x_2; y_2; 0 \}, \vec{c} = \{ x_3; y_3; z_3 \}$, тогда

$[ \vec{a} \times \vec{b} ] = \begin{vmatrix}
    i & j & k \\
    x_1 & 0 & 0 \\
    x_2 & y_2 & 0
\end{vmatrix} = \vec{i} * 0 - \vec{j} * 0 + \vec{k} \begin{vmatrix}
    x_1 & 0 \\
    x_2 & y_2
\end{vmatrix} = \{ 0; 0; x_1 y_2 \}$

$[ [ \vec{a} \times \vec{b} ] \times \vec{c} ] = \begin{vmatrix}
    i & j & k \\
    0 & 0 & x_1 y_2 \\
    x_3 & y_3 & z_3
\end{vmatrix} = \vec{i} (0 - x_1 y_2 y_3) - \vec{j} (0 - x_3 x_1 y_2 ) + \vec{k} * 0 = \{ -x_1 y_2 y_3; x_3 x_1 y_2; 0 \}$

$a * c = x_1 x_3, b * c = x_2 x_3 + y_2 y_3, \vec{b} (a * c) = \{x_1 x_2 x_3; x_1 y_2 x_3; 0 \}, \vec{a} (\vec{b} \vec{c}) = \{ x_1 (x_2 x_3 + y_2 y_3); 0; 0 \}$

$\vec{b} (\vec{a} \vec{c}) - \vec{a} (\vec{b} \vec{c}) = \{ x_1 x_2 x_3 - x_1 (x_2 x_3 + y_2 y_3); x_1 y_2 x_3; 0 \}$

\end{flushleft}

\pagebreak
\section{Алгебра и геометрия — 18.11.2022}

\subsection{Разбор контрольной работы №1}

Первые три задания — матрицы и все, что с ними связано; четвертое и пятое — прямые, уравнения прямых, углы между ними; шестое — кривые второго порядка.

\subsubsection{Первое задание}

\paragraph{Разобранный пример}

\begin{flushleft}

Пусть $B = A^{-1}$, где $A = \begin{pmatrix}
    -1 & 0 & 2 & 3 \\
    0 & 1 & -1 & 2 \\
    -2 & 1 & 1 & 3 \\
    0 & -3 & 1 & -1
\end{pmatrix}$

Найдите элемент $b_{31}$ матрицы $B$.

\bigskip

$A^{-1} = \frac{1}{\det A} \begin{pmatrix}
    A_{11} & \dots & A_{n 1} \\
    \dots & \dots & \dots \\
    A_{1 n} & \dots & A_{n n}
\end{pmatrix}$

$A_{i j} = (-1)^{i + j} M_{i j}$

$\Delta A = -1 * \begin{vmatrix}
    1 & -1 & 2 \\
    1 & 1 & 3 \\
    -3 & 1 & -1
\end{vmatrix} + 2 * \begin{vmatrix}
    0 & 1 & 2 \\
    -2 & 1 & 3 \\
    0 & -3 & -1
\end{vmatrix} - 3 * \begin{vmatrix}
    0 & 1 & -1 \\
    -2 & 1 & 1 \\
    0 & -3 & 1
\end{vmatrix} = -1 (-1 + 2 + 9 + 6 - 3 - 1) + 2 (12 - 2) - 3 (-6 + 2) = -12 + 20 + 12 = 20$

\bigskip

$A_{13} = (-1)^{1 + 3} \begin{vmatrix}
    0 & 1 & 2 \\
    -2 & 1 & 3 \\
    0 & -3 & -1
\end{vmatrix} = 1 * 10 = 10$

$b_{31} = \frac{1}{\det A} * A_{13} = \frac{1}{20} * 10 = \frac{1}{2}$

\paragraph{Другие возможные вариации}

\textbf{Возведение матрицы в определенную степень.} Решается перемножением матрицы самой на себя. 


\subsubsection{Второе задание}

\paragraph{Разобранный пример}

Найдите общее решение системы

\begin{equation}
    \begin{cases}
        x_1 - 2x_2 - 4x_3 = -7 \\
        2x_1 + x_2 - 3x_3 = 1 \\
        x_1 + 8x_2 + 6x_3 = 23
    \end{cases}
\end{equation}

Распишем расширенную матрицу системы и упростим ее, чтобы решить систему методом Гаусса:

$A = \begin{pmatrix}
    1 & -2 & -4 \\
    2 & 1 & -3 \\
    1 & 8 & 6
\end{pmatrix}$, $B = \begin{pmatrix}
    -7 \\
    1 \\
    23
\end{pmatrix}$, $(A|B) = \begin{pmatrix}
    1 & -2 & -4 & -7 \\
    2 & 1 & -3 & 1 \\
    1 & 8 & 6 & 23
\end{pmatrix} \approx \begin{pmatrix}
    1 & -2 & -4 & -7 \\
    0 & 5 & 5 & 15 \\
    0 & 10 & 10 & 30
\end{pmatrix} \approx \begin{pmatrix}
    1 & -2 & -4 & -7 \\
    0 & 1 & 1 & 3
\end{pmatrix} \approx \begin{pmatrix}
    1 & 0 & -2 & -1 \\
    0 & 1 & 1 & 3
\end{pmatrix}$

Данная матрица эквивалентна следующей системе уравнений:

\begin{equation}
    \begin{cases}
        x_1 - 2x_3 = -1 \\
        x_2 + x_3 = 3
    \end{cases}
\end{equation}

Пусть $x_3 = 1$, тогда $x_1 = 1$, $x_2 = 2$, решением системы будет $X = c \begin{pmatrix}
    1 \\
    2 \\
    1
\end{pmatrix}$, $c \in R$


\subsubsection{Третье задание}

\paragraph{Разобранный пример}

Решите матричное уравнение

$\begin{pmatrix}
    2 & -1 \\
    1 & 0
\end{pmatrix} * X = \begin{pmatrix}
    3 & -2 & -1 \\
    1 & -1 & 0
\end{pmatrix}$

\bigskip

Вспоминаем, что

$A * X = B$, $X = A^{-1} * B$

Нам нужно найти обратную матрицу и умножить ее на матрицу $B$

\bigskip

$\Delta A = 1$

$A_{11} = 1 * 0 = 0 , A_{12} = -1 * 1 = -1, A_{21} = -1 * (-1) = 1, A_{22} = 1 * 2 = 2$

$A^{-1} = \begin{pmatrix}
    0 & 1 \\
    -1 & 2
\end{pmatrix}$

\bigskip

$X = A^{-1} * B = \begin{pmatrix}
    0 & 1 \\
    -1 & 2
\end{pmatrix} \begin{pmatrix}
    3 & -2 & -1 \\
    1 & -1 & 0
\end{pmatrix} = \begin{pmatrix}
    1 & -1 & 0 \\
    -1 & 0 & 1
\end{pmatrix}$

\paragraph{Другие возможные вариации}

\textbf{Нахождение элементов исходной матрицы из обратной.} Вспоминаем, что $A * A^{-1} = E$ — таким образом, для решения нужно умножить обратную матрицу на единичную — получить исходную матрицу — указать тот элемент, который нас просят найти.


\subsubsection{Четвертое задание}

\paragraph{Разобранный пример}

Найдите точку, симметричную точке $A(2; -1)$ относительно прямой $y = x - 1$

\def\size{5} % natural number

\begin{tikzpicture}
  \begin{axis}[grid=both,ymin=-5,ymax=5,xmax=5,xmin=-5,xticklabel=\empty,yticklabel=\empty,
               minor tick num=1,axis lines = middle,xlabel=$x$,ylabel=$y$,label style =
               {at={(ticklabel cs:1.1)}}]
    \addplot[color=red]{x - 1};
    \addlegendentry{\(x - 1\)}
    \addplot[color=blue]{-x + 1};
    \addlegendentry{\(-x + 1\)}
    \addplot[mark=*] coordinates {(2,-1)} node[pin={[pin distance=1mm]1:{$A$}}]{} ;
    \addplot[mark=*] coordinates {(1,0)} node[pin={[pin distance=1mm]90:{$C$}}]{} ;
    \addplot[mark=*] coordinates {(0,1)} node[pin={[pin distance=1mm]180:{$B$}}]{} ;
  \end{axis}
\end{tikzpicture}

Если мы ищем симметричную точку, то она должна быть расположена на линии, перпендикулярной нашей прямой по обратную сторону на том же самом расстоянии.

Исходя из уравнений наших прямых мы имеем две прямые: проходящую через нашу точку $A$, и перпендикулярную ей.

\bigskip

$l_1 \perp l_2$, $k_2 = -\frac{1}{k_1}$

$y = x - 1 \implies k_1 = 1$, $k_2 = -1$

$y - y_0 = k(x - x_0)$

$y + 1 = -1 (x - 2)$, $y = -x + 1$ — уравнение прямой, перпендикулярной исходной

\bigskip

Найдем точку пересечения двух линий:

\begin{equation}
    \begin{cases}
        y = x - 1 \\
        y = -x + 1
    \end{cases}
\end{equation}

Точка пересечения — $C(1; 0)$

От точки $C$ до точки $A$ и искомой точки $B$ расстояние одинаково. Таким образом, $C$ является центром масс. Понимая это, найдем координаты искомой точки $B$:

\hspace{5mm} $x_c = \frac{x_a + x_b}{2}$, $y_c = \frac{y_a + y_b}{2}$

Координаты точки $B$ — $B(0; 1)$


\subsubsection{Пятое задание}

\paragraph{Разобранный пример}

Найдите косинус угла между прямыми $2x - 3y + 1 = 0$ и $3x - 2y + 5 = 0$

Найдем угловые коэффициенты. Но для начала приведем эти уравнения к нормальному виду:

$y = \frac{-2x - 1}{-3} = \frac{2}{3}x + \frac{1}{3}$, $y = \frac{-3x - 5}{-2} = \frac{3}{2}x + \frac{5}{2}$

\bigskip

$\tg \alpha = \frac{k}{1 + k_1 k_2} = \pm \frac{k_2 - k_1}{1 + k_1 k_2} \ge 0$

$\tg \alpha = \pm \frac{\frac{3}{2} - \frac{2}{3}}{1 + 1} \ge 0$

$\tg \alpha = \frac{5}{12}$

\bigskip

Для того, чтобы найти синус и косинус, посчитаем $\sqrt{5^2 + 12^2} = \sqrt{25 + 144} = \sqrt{169} = 13$

Синус будет равен $\frac{5}{13}$, а косинус $\frac{12}{13}$


\subsubsection{Шестое задание}

\paragraph{Разобранный пример}

Найдите расстояние между фокусами кривой $5x^2 - 4y^2 - 10x - 16y - 31 = 0$

Выведем полные квадраты, найдем $A$ и $B$, найдем $c$, и далее найдем расстояние между ними:

\bigskip

$5x - 10x = 5(x^2 - 2x + 1 - 1) = 5(x-1)^2 - 5$

$-4y^2 - 16y = -4(y^2 + 4y) = -4(y^2 + 4y + 4 - 4) = -4(y + 2)^2 + 16$

$5 (x - 1)^2 - 5 - 4(y + 2)^2 + 16 - 31 = 0$

$5 (x - 1)^2 - 4(y + 2)^2 = 20$

В каноническом виде мы должны получать после равно единицу, следовательно нам нужно поделить левую и правую часть на двадцать:

$\frac{(x - 1)^2}{4} - \frac{(y+2)^2}{5} = 1$ — каноническое уравнение гиперболы

$c^2 = a^2 + b^2$, $c = \sqrt{a^2 + b^2} = \sqrt{4 + 5} = \sqrt{9} = 3$

\bigskip

Следовательно, мы имеем два фокуса: $F_1(-3; 0)$, $F_2(3; 0)$. Расстояние между ними — 6


\subsubsection{Другие возможные вариации}

\paragraph{Вариация 1}. Матрица $B = A^2$, дана матрица $A$, нужно найти какой-то элемент матрицы $B$. Следовательно, мы умножаем матрицу $A$ саму на себя и ищем этот элемент.

\paragraph{Вариация 2}. Решить систему и найти значение линейной комбинации — нужно найти методом Крамера свободные переменные, базисные переменные, выразить базисные через свободные и подставить в линейную комбинацию — найти численное значение этой комбинации.

\paragraph{Вариация 3}. Имеем матрицу $X$, умноженную на другую матрицу, в результате чего мы получаем еще другую матрицу.

\paragraph{Другие вариации}

Составьте уравнение прямой, проходящей через точку перпендикулярно прямой; найти расстояние от точки до прямой; найти эксцентриситет кривой; найти алгебраическое дополнение для квадратной матрицы; найти общее решение системы; даны три вершины квадрата, найти координаты четвертой вершины; составьте уравнения биссектрис углов, образованных прямыми; найдите расстояние между директрисами кривой; при каких значениях $x$, $y$ и $z$ обратная матрица будет обратной для исходной матрицы — перемножить эту матрицу с обратной, получить по свойствам обратных матриц единичную — понять, при каких значениях эта единичная матрица будет образовываться; найти все значения $a$, при которой система является совместной — то есть, имеет решения — либо через теорему Кронекера-Капелли, либо через решение; найти расстояние между прямыми; найти котангенс угла в треугольнике с вершинами; найти координаты фокуса параболы; заданы матрицы, найти определитель матрицы $3A - 2B$; найти какой-либо элемент обратной матрицы — умножить эту обратную матрицу на единичную, получить исходную — найти в ней искомый элемент;

\end{flushleft}

\pagebreak
\section{Алгебра и геометрия - 21.11.2022}

\subsection{Плоскость в пространстве}

\begin{flushleft}

$\vec{N}$ - ненулевой вектор, перпендикулярный плоскости, называется \textbf{вектором нормали} данной плоскости.

\subsubsection{Уравнения плоскости}

$Ax + By + C z + D = 0$ - общее уравнение плоскости, $A^2 + B^2 + C^2 \ne 0$, $\vec{N} = \{ A; B; C \}$

$A(x - x_0) + B(y - y_0) + C(z - z_0) = 0$ - уравнение плоскости, проходящей через точку $M_0(x_0, y_0, z_0)$ с данным $\vec{N} = \{ A; B; C \} \ne 0$ 


$\begin{vmatrix}
    x - x_0 & y - y_0 & z - z_0 \\
    a_1 & a_2 & a_3 \\
    b_1 & b_2 & b_3
\end{vmatrix} = 0$ - 
уравнение плоскости, проходящей через точку $M_0(x_0; y_0; z_0)$ параллельно двум неколлинеарным векторам $\vec{a} = \{ a_1; a_2; a_3 \}$ и $\vec{b} = \{ b_1; b_2; b_3 \}$

$\begin{vmatrix}
    x - x_1 & y - y_1 & z - z_1 \\
    x_2 - x_1 & y_2 - y_1 & z_2 - z_1 \\
    x_3 - x_1 & y_3 - y_1 & z_3 - z_1
\end{vmatrix} = 0$ - уравнение плоскости, проходящей через 3 точки $M_1(x_1; y_1; z_1)$, $M_2(x_2; y_2; z_2)$ и $M_3(x_3; y_3; z_3)$, не лежащие на одной прямой

\subsubsection{Углы между плоскостями}

\textbf{Косинус угла} между плоскостями $\alpha_1$: $A_1 x + B_1 y + C_1 z + D_1 = 0$ и $\alpha_2$: $A_2 x + B_2 y + C_2 z + D_2 = 0$ вычисляется по формуле: $\cos (\alpha_1; \alpha_2) = \pm \frac{\vec{N_1} * \vec{N_2}}{|\vec{N_1}| * |\vec{N_2}|} = \pm \frac{A_1 A_2 + B_1 B_2 + C_1 C_2}{\sqrt{A_1^2 + B_1^2 + C_1^2} * \sqrt{A_2^2 + B_2^2 + C_2^2}} \ge 0$

Если $\alpha_1 \perp \alpha_2$, то $\vec{N_1} \perp \vec{N_2}$, если $\alpha_1 || \alpha_2$, то $\vec{N_1} || \vec{N_2}$

\subsubsection{Примеры решения задач}

\hfill

\textbf{Пример 1}

Составьте уравнение плоскости $\alpha$, параллельной оси $Ox$ и проходящей через точки $A(1; 0; 5)$ и $B(0; -4; 8)$.

$M_0 = A(1; 0; 5)$, $\vec{a} = \vec{AB} = \{ -1; -4; 3 \}$, $\vec{b} = \vec{i} = \{ 1; 0; 0 \}$

$\begin{vmatrix}
    x - 1 & y - 0 & z - 5 \\
    -1 & -4 & 3 \\
    1 & 0 & 0
\end{vmatrix} = 0 \Longleftrightarrow 3(y-0) - (-4(z-5)) = 0 \Longleftrightarrow 3y - (-4z + 20) = 0 \Longleftrightarrow 3y + 4z - 20 = 0$

\hfill

\textbf{Пример 2}

Найдите угол между плоскостями $\alpha_1$: $x - y + 80 = 0$, $\alpha_2$: $3x + 4y + 5z - 17 = 0$

$N_1 = \{1; -1; 0 \}$, $N_2 = \{3; 4; 5 \}$

$\cos (\alpha_1; \alpha_2) = \frac{3 + (-4) + 0}{\sqrt{2} * \sqrt{50}} = \frac{-1}{\sqrt{2} * \sqrt{50}} = 0.1$, $\angle (\alpha_1; \alpha_2) = \arccos 0.1 \approx 84.3^\circ$

\hfill

\textbf{Пример 3}

Найдите расстояние от точки $P(0; -1; 5)$ до плоскости $\alpha$, проходящей через точку $A(8; 1; -2)$ перпендикулярно вектору $\vec{n} = \{ 1; 2; -2 \}$

\hfill

\textbf{Первый способ} решения данной задачи.

$r = |\proj_{\vec{n}} \vec{PA}| = |\frac{\vec{n} * \vec{PA}}{|\vec{n}|}|$, $\vec{PA} = \{ 8; 2; -7 \}$

$r = \frac{8 + 4 + 14}{3} = \frac{26}{3}$

\textbf{Второй способ решения данной задачи}

$r = \frac{|A x_0 + B y_0 + C z_0 + D|}{\sqrt{A^2 + B^2 + C^2}}$

$x + 2y - 2z - 14 = 0$ - уравнение данной плоскости

$r = \frac{|0 + 2*(-1) - 2 * 5 - 14|}{3} = \frac{26}{3}$

\end{flushleft}

\pagebreak
\section{Алгебра и геометрия - 23.11.2022}

\subsection{Прямая в пространстве}

\begin{flushleft}

Вектор $\vec{S}$, являющийся коллинеарным данной прямой, называется направляющим.

\subsubsection{Уравнения прямой в пространстве}

$\frac{x - x_0}{m} = \frac{y - y_0}{n} = \frac{z - z_0}{p}$, где $M_0(x_0; y_0; z_0)$ - фиксированная точка, лежащая на прямой; $\vec{S} = \{ m; n; p \}$, $m^2 + n^2 + p^2 \ne 0$ - \textbf{каноническое} уравнение прямой.

$\frac{x - x_1}{x_2 - x_1} = \frac{y - y_1}{y_2 - y_1} = \frac{z - z_1}{z_2 - z_1}$ - уравнение прямой \textbf{через две точки} $M_1(x_1; y_1; z_1)$ и $M_2(x_2; y_2; z_2)$

Уравнение прямой, заданной \textbf{параметрически} ($t \in (- \infty; +\infty) $):

\begin{equation}
    \begin{cases}
        x = x_0 + mt \\
        y = y_0 + n t \\
        z = z_0 + pt
    \end{cases}
\end{equation}

Уравнение прямой, полученной в результате \textbf{пересечения двух плоскостей} ($\vec{N_1} = \{ A_1; B_1; C_1 \} \ne \lambda \vec{N_2} = \{ A_2; B_2; C_2 \}$):

\begin{equation}
    \begin{cases}
        A_1 x + B_1 y + C_1 z + D_1 = 0 \\
        A_2 x + B_2 y + C_2 z + D_2 = 0
    \end{cases}
\end{equation}

\subsubsection{Связь между направляющим вектором и векторами нормали}

Направляющий вектор $\vec{S}$ прямой $l$: $\vec{N_1} = \{ A_1; B_1; C_1 \}$ и $\vec{N_2} = \{ A_2; B_2; C_2 \}$: $S = \vec{N_1} \times \vec{N_2}$

\subsubsection{Косинус угла между двумя прямыми}

Даны две прямые: $l_1$: $\frac{x - x_1}{m_1} = \frac{y - y_1}{n_1} = \frac{z - z_1}{p_1}$, $l_2$: $\frac{x - x_2}{m_2} = \frac{y - y_2}{n_2} = \frac{z - z_2}{p_2}$

$\cos (l_1; l_2) = \pm \frac{\vec{S_1} * \vec{S_2}}{|\vec{S_1}| * |\vec{S_2}|} = \pm \frac{m_1 m_2 + n_1 n_2 + p_1 p_2}{\sqrt{m_1^2 + n_1^2 + p_1^2} * \sqrt{m_2^2 + n_2^2 + p_2^2}} \ge 0$

\subsubsection{Угол между прямой и плоскостью}

Даны прямая $l$: $\frac{x - x_0}{m} = \frac{y - y_0}{n} = \frac{z - z_0}{p}$ и плоскость $\alpha$: $Ax + By + C z + D = 0$

$\angle (l; \alpha) = \pm \frac{\vec{S} * \vec{N}}{|\vec{S}| * |\vec{N}|} = \pm \frac{A m + B n + C p}{\sqrt{A^2 + B^2 + C^2} * \sqrt{m^2 + n ^2 + p^2}}$

\subsubsection{Примеры решения задач}

\paragraph{Пример 1.} Составьте канонические и параметрические уравнения высоты $l$, опущенной на плоскость $(B C D)$ в пирамиде $A B C D$, если т. $A(0; 5; -2)$, т. $B(1; 2; -1)$, т. $C(4; 5; 0)$, т. $D(1; 1; 1)$

$\vec{BC} = \{ 3; 3; 1 \}, \vec{B D} = \{ 0; -1; 2 \}, \vec{S} = \vec{BC} \times \vec{B D} = \begin{vmatrix}
    \vec{i} & \vec{j} & \vec{k} \\
    3 & 3 & 1 \\
    0 & -1 & 2
\end{vmatrix} = \{ 7; 6; -3 \}$

Найдем каноническое уравнение: $\frac{x - x_0}{m} = \frac{y - y_0}{n} = \frac{z - z_0}{p} \Longleftrightarrow \frac{x}{7} = \frac{y - y_5}{-6} = \frac{z + 2}{-3}$, $\vec{S} = \{ m; n; p \} = \{ 7; -6; -3 \}$, $M_0(x_0; y_0; z_0) = A(0; 5; -2)$

Найдем параметрическое уравнение данной прямой:

\begin{equation}
    \begin{cases}
        x = x_0 + mt \\
        y = y_0 + nt \\
        z = z_0 + pt
    \end{cases}
    \Longleftrightarrow
    \begin{cases}
        x = 0 + 7t \\
        y = 5 - 6t \\
        z = -2 - 3t
    \end{cases}
\end{equation}

\paragraph{Пример 2.} Составьте канонические и параметрические уравнения прямой, проходящей через точку $A(3; 2; -1)$ перпендикулярно плоскости $x O z$

$\vec{j} = \{ 0; 1; 0 \}$ - вспомогательный вектор.

Найдем каноническое уравнение: $\frac{x - 3}{0} = \frac{y - 2}{1} = \frac{z + 1}{0}$

Найдем параметрическое уравнение данной прямой:

\begin{equation}
    \begin{cases}
        x = 3 \\
        y = 2 + t \\
        z = -1
    \end{cases}
\end{equation}

\paragraph{Пример 3.} Найдите угол между прямыми $l_1$: $\frac{x - 3}{2} = \frac{y + 4}{2} = \frac{z - 5}{-1}$ и $l_2$:

\begin{equation}
    \begin{cases}
        x + y - 2z - 1 = 0 \\
        2x - z + 8 = 0
    \end{cases}
\end{equation}

$\vec{N_1} = \{ 1; 1; -2 \}$, $\vec{N_2} = \{ 2; 0; -1 \}$, $\vec{S_1} = \{ 2; 2; -1 \}$, $\vec{S_2} = \vec{N_1} \times \vec{N_2} = \begin{vmatrix}
    \vec{i} & \vec{j} & \vec{k} \\
    1 & 1 & -2 \\
    2 & 0 & -1
\end{vmatrix} = \{ -1; -3; -2 \} $

$\cos(l_1; l_2) = \pm \frac{\vec{S_1} * \vec{S_2}}{|\vec{S_1}| * |\vec{S_2}|} = \pm \frac{-2 + (-6) + 2}{3 * \sqrt{14}} = \frac{2}{\sqrt{14}} \approx 0.535$, $\angle (l_1; l_2) \approx 57.7^\circ$

\end{flushleft}

\end{document}