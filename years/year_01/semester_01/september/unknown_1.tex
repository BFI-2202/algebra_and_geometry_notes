\documentclass{article}
\usepackage[utf8]{inputenc}

\usepackage[T2A]{fontenc}
\usepackage[utf8]{inputenc}
\usepackage[russian]{babel}

\usepackage{amsmath}
\usepackage{multienum}

\def\vec{\ensuremath\overrightarrow}
\def\proj{\ensuremath\text{Пр}}

\title{Алгебра и геометрия}
\author{Лисид Лаконский}
\date{September 2022}

\begin{document}

\maketitle

\tableofcontents
\pagebreak

\section{Алгебра и геометрия - неизвестно}

\subsection{Матрицы}

\begin{flushleft}

\textbf{Матрицой} размера $m \times n$ называется матрица, у которой m строк и n столбцов

\textbf{Примеры} матриц: $A = \begin{pmatrix}
    -2 & 3 \\
    0 & 1 \\
    1 & 4 \\
    5 & 8
\end{pmatrix}$, $B = \begin{pmatrix}
    1 \\
    8 \\
    1
\end{pmatrix}$, $C = \begin{pmatrix}
    0 & -1 & 3 & 4
\end{pmatrix}$

\textbf{Нулевой} матрицой называется матрица, состоящая лишь из нулей, \textbf{единичной матрицей} называется матрица, по главной диагонали которой расположены единицы: $E = \begin{pmatrix}
    1 & 0 & 0 \\
    0 & 1 & 0 \\
    0 & 0 & 1
\end{pmatrix}$

\subsubsection{Элементарные операции над матрицами}

\paragraph{Транспонирование} $A^T = \begin{pmatrix}
    -2 & 0 & 1 & 5
\end{pmatrix}$, $B^T = \begin{pmatrix}
    1 & 2 & 0 & -1
\end{pmatrix}$, $C^T = \begin{pmatrix}
    0 \\
    -1 \\
    3 \\ 4
\end{pmatrix}$

\paragraph{Умножение на скаляр}

$$k * \begin{pmatrix}
    a_{11} & a_{12} & \dots & a_{1n} \\
    \dots & \dots & \dots & \dots \\
    a_{n1} & a_{n2} & \dots & a_{n n}
\end{pmatrix} = \begin{pmatrix}
    k * a_{11} & k * a_{12} & \dots & k * a_{1n} \\
    \dots & \dots & \dots & \dots \\
    k * a_{n1} & k * a_{n2} & \dots & k * a_{n n}
\end{pmatrix}$$

\paragraph{Сложение}

Пусть $A = \begin{pmatrix}
    a_{11} & a_{12} & \dots & a_{1n} \\
    \dots & \dots & \dots & \dots \\
    a_{n1} & a_{n2} & \dots & a_{n n}
\end{pmatrix}$, $B = \begin{pmatrix}
    b_{11} & b_{12} & \dots & b_{1n} \\
    \dots & \dots & \dots & \dots \\
    b_{n1} & b_{n2} & \dots & b_{n n}
\end{pmatrix}$, тогда $$A + B = \begin{pmatrix}
    a_{11} + b_{11} & a_{12} + b_{12} & \dots & a_{1n} + b_{1n} \\
    \dots & \dots & \dots & \dots \\
    a_{n1} + b_{n1} & a_{n2} + b_{n2} & \dots & a_{n n} + b_{n n}
\end{pmatrix}$$

\paragraph{Умножение на матрицу}

Условие \textbf{возможности перемножения} двух матриц: \textbf{количество столбцов} одной матрицы должно быть равно \textbf{количеству строк} другой.

\textbf{Пример}:

\hspace{5mm} $A = \begin{pmatrix}
    2 & -1 \\
    3 & -2
\end{pmatrix}, B = \begin{pmatrix}
    1 & 2 & 5 \\
    0 & 1 & -1
\end{pmatrix}, A * B = \begin{pmatrix}
    a_{11} * b_{11} + a_{12} * b_{21} & a_{11} * b_{12} + a_{12} * b_{22} \\
    a_{21} * b_{11} + a_{22} * b_{21} & a_{21} * b_{12} + a_{22} * b_{22}
\end{pmatrix} = \begin{pmatrix}
    2 & 3 & 11 \\
    3 & 4 & 17
\end{pmatrix}$

\subsubsection{Свойства произведения матриц}

\begin{multienumerate}
    \mitemxx{$A * B \ne B * A$}{$(A * B) * C = A * (B * C)$}
    \mitemxx{$(A + B) * C = A * C + B * C$}{$A * (B + C) = A * B + A * C$}
    \mitemxx{$A * E = E * A = A$}{$A * 0 = 0 * A = 0$}
\end{multienumerate}

\subsubsection{Определитель матрицы}

$\Delta = \Delta A = \det A = |A| = \begin{vmatrix}
    a_{11} & \dots & a_{1n} \\
    \dots & \dots & \dots \\
    a_{n1} & \dots & a_{n n}
\end{vmatrix}$

\paragraph{Определитель матрицы $2 \times 2$}

\hfill

\hspace{5mm} $\Delta = \begin{vmatrix}
    a_{11} & a_{12} \\
    a_{21} & a_{22}
\end{vmatrix} = a_{11} * a_{22} - a_{12} * a_{21}$

\paragraph{Определитель матрицы $3 \times 3$}

\hfill

\hspace{5mm} $\Delta = \begin{vmatrix}
    a_{11} & a_{12} & a_{13} \\
    a_{21} & a_{22} & a_{23} \\
    a_{31} & a_{32} & a_{33}
\end{vmatrix} = (a_{11} * a_{22} * a_{33} + a_{21} * a_{32} * a_{13} + a_{12} * a_{23} * a_{31}) - (a_{13} * a_{22} * a_{31} + a_{12} * a_{21} * a_{33} + a_{32} * a_{23} * a_{11})$

\paragraph{Определитель квадратной матрицы произвольной размерности}

\hfill

\hspace{5mm} $\Delta = \begin{vmatrix}
    a_{11} & a_{12} & a_{13} \\
    a_{21} & a_{22} & a_{23} \\
    a_{31} & a_{32} & a_{33}
\end{vmatrix} = a_{11} * \begin{vmatrix}
    a_{22} & a_{23} \\
    a_{32} & a_{33}
\end{vmatrix} - a_{12} * \begin{vmatrix}
    a_{21} & a_{23} \\
    a_{31} & a_{33}
\end{vmatrix} + a_{13} * \begin{vmatrix}
    a_{21} & a_{22} \\
    a_{31} & a_{32}
\end{vmatrix}$

\subsubsection{Алгебраическое дополнение}

$A_{ij} = (-1)^{i + j} * M_{ij}$

$A^{-1} = \frac{1}{\det A} * \begin{pmatrix}
    A_{11} & A_{21} & \dots & A_{n 1} \\
    \dots & \dots & \dots & \dots \\
    A_{1 n} & A_{2n} & \dots & A_{n n}
\end{pmatrix}$

\paragraph{Обратная матрица} Матрица $A^{-1}$ называется обратной к матрице $A$, если их произведение дает в результате единичную матрицу.

$A^{-1}$ существует, если:

\begin{multienumerate}
    \mitemxx{$A$ — квадратная матрица}{$\det A \ne 0$}
\end{multienumerate}

Пример: найдем матрицу, обратную $A = \begin{pmatrix}
    -3 & 0 & 1 \\
    1 & 1 & -1 \\
    1 & -1 & 0
\end{pmatrix}$

$\det A = 0 + (-1) + 0 - 1 + 3 - 0 = 1$

\begin{multienumerate}
    \mitemxxx{$A_{11} = (-1)^{1+1} * \begin{vmatrix}
        1 & -1 \\
        -1 & 0
    \end{vmatrix} = -1$}{$A_{21} = -1$}{$A_{31} = -1$}
    \mitemxxx{$A_{12} = (-1)^{1+2} * \begin{vmatrix}
        1 & -1 \\
        1 & 0
    \end{vmatrix} = -1$}{$A_{22} = -1$}{$A_{23} = -3$}
    \mitemxxx{$A_{13} = (-1)^{1+3} * \begin{vmatrix}
        1 & 1 \\
        1 & -1
    \end{vmatrix} = -2$}{$A_{23} = -3$}{$A_{33} = -3$}
\end{multienumerate}

$A^{-1} = \begin{pmatrix}
    -1 & -1 & -1 \\
    -1 & -1 & -2 \\
    -2 & -3 & -3
\end{pmatrix}$

\subsection{Решение систем линейных уравнений}

Пусть задана система $n$ линейных уравнений с $n$ неизвестными:

\begin{equation}
    \begin{cases}
        a_{11} * x_1 + a_{12} * x_2 + \dots + a_{1 n} * x_{n} = b_1 \\
        \dots \\
        a_{n 1} * x_1 + a_{n 2} * x_2 + a_{n n} * x_n = b_n
    \end{cases}
\end{equation}

$A = X * B$, $A = \begin{pmatrix}
    a_{11} & \dots & a_{1 n} \\
    \dots & \dots & \dots \\
    a_{n 1} & \dots & a_{n n}
\end{pmatrix}$, $X = \begin{pmatrix}
    X_1 \\
    X_2 \\
    \dots \\
    X_n
\end{pmatrix}$

\subsubsection{Правило Крамера}

$X_{i} = \frac{\Delta X_i}{\Delta A}$, $\Delta X_{i} = \begin{pmatrix}
    a_{11} & \dots & a_{1 i - 1} & b_1 & a_{1 i + 1} &\dots &  a_{1 n} \\
    a_{21} & \dots & a_{2 i - 1} & b_2 & a_{2i + 1} & \dots & a_{2 n} \\
    a_{n 1} & \dots & a_{n i - 1} & b_n & a_{n i + 1} & \dots & a_{n n}
\end{pmatrix}$

\paragraph{Пример}

\begin{equation}
    \begin{cases}
        2x_1 - x_2 + 3x_3 = 3 \\
        x_1 + 2x_2 - x_3 = -1 \\
        3x_1 - 2x_2 + 2x_3 = 5
    \end{cases}
\end{equation}

$A = \begin{pmatrix}
    2 & -1 & 3 \\
    1 & 2 & -1 \\
    3 & -2 & 2
\end{pmatrix}$, $B = \begin{pmatrix}
    3 \\
    -1 \\
    5
\end{pmatrix}$, $X = \begin{pmatrix}
    x_1 \\
    x_2 \\
    x_3
\end{pmatrix}$

\bigskip

$\Delta A = 8 + (-6) + 3 - 4 - 18 + 2 = -15$

$\Delta X_{1} = \begin{pmatrix}
    3 & -1 & 3 \\
    -1 & 2 & 1 \\
    5 & -2 & -2
\end{pmatrix} = +12 + 6 + (-5) - 30 - 6 + 2 = -15$

$\Delta X_{2} = \begin{pmatrix}
    2 & 3 & 3 \\
    1 & -1 & -1 \\
    3 & 5 & 2
\end{pmatrix} = -15$

$\Delta X_3 = \begin{pmatrix}
    2 & -1 & 3 \\
    1 & 2 & -1 \\
    3 & -2 & 5
\end{pmatrix} = 19 - 19 = 0$

$X_1 = \frac{-15}{-15} = 1$, $X_2 = -1$, $X_3 = 0$, $X = \begin{pmatrix}
    1 \\
    -1 \\
    0
\end{pmatrix}$

\end{flushleft}

\subsubsection{Матричный способ}

$X = A^{-1} * B$

\end{document}