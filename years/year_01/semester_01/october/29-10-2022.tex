\documentclass{article}
\usepackage[utf8]{inputenc}

\usepackage[T2A]{fontenc}
\usepackage[utf8]{inputenc}
\usepackage[russian]{babel}

\usepackage{amsmath}
\usepackage{multienum}

\def\vec{\ensuremath\overrightarrow}
\def\proj{\ensuremath\text{Пр}}

\title{Алгебра и геометрия}
\author{Лисид Лаконский}
\date{October 2022}

\begin{document}

\maketitle

\tableofcontents
\pagebreak

\section{Алгебра и геометрия - 29.10.2022}

\subsection{Линейные пространства}

\begin{flushleft}

\textbf{Линейным пространством} называется множество элементов произвольной природы, на котором определены операции \textbf{сложения} и \textbf{умножения на число}, согласованные друг с другом и \textbf{замкнутые в этом множестве}.

\hfill

\textbf{Замкнутость в множестве} означает то, что результаты выполнения операций над его элементами остаются элементами множества.

\subsubsection{Аксиомы линейного пространства}

\textbf{Сложением (обобщенным сложением)} называется операция, которая любым двум элементам данного множества ставит в соответствие элемент этого же множества, называемый их суммой: $x, y \in D \rightarrow z \in D, z = x + y$

Причем данная операция удовлетворяет следующим условиям:

\begin{enumerate}
    \item \textbf{ассоциативности:} $x \bigoplus (y \bigoplus z) = (x \bigoplus y) \bigoplus z$
    \item \textbf{коммутативности:} $x \bigoplus y = y \bigoplus x$
    \item \textbf{нулевого элемента:} $x \bigoplus \theta = x$
    \item \textbf{обратного элемента:} $x \bigoplus \overline{x} = \theta$
\end{enumerate}

Множества с операциями такого типа называются \textbf{абелевыми группами}.

\hfill

Умножением на число называется операция, которая любому элементу данного множества и любому действительному числу $\alpha$ ставит в соответствие элемент того же множества, называемый их произведением: $x \in D; \alpha \in R \rightarrow z \in D, z = \alpha \bigodot x$

Причем данная операция удовлетворяет следующим условиям:

\begin{enumerate}
    \item $\alpha \bigodot (\beta \bigodot x) = (\alpha \bigodot \beta) \bigodot x$
    \item $1 \bigodot x = x$
\end{enumerate}

\hfill

\textbf{Условия согласования операций сложения и умножения}:

\begin{enumerate}
    \item $(\alpha + \beta) \bigodot x = \alpha \bigodot x \bigoplus \beta \bigodot x$
    \item $\alpha \bigodot (x \bigoplus y) = \alpha \bigodot x + \alpha \bigodot y$
\end{enumerate}

\subsubsection{Примеры линейных пространств}

\textbf{Пример 1.} Множество действительных чисел является линейным пространством.

\hfill

\textbf{Пример 2.} Множество матриц также является линейным пространством.

\hfill

\textbf{Пример 3.} Рассмотрим множество ($A$) многочленов второго порядка (вида $ax^2 + bx + c$).

Оно не является линейным пространством: при сложении элементов этого множества мы можем получить элемент, не принадлежащий множеству. Например, $(2x^2 + 3x + 1) + (-2x^2 - 5x) = -2x + 1 \notin A$

\hfill

\textbf{Пример 4.} Множество векторов является линейным пространством.

\hfill

\textbf{Пример 5.} Множество векторов, выходящих из данной точки и заканчивающихся в конце прямой линии, на которой лежит данная точка.

Данное пространство не является линейным.

\subsubsection{Следствия из аксиом линейного пространства}

\begin{enumerate}
    \item В линейном пространстве существует единственный нулевой элемент
    \item В линейном пространстве у каждого элемента должен существовать обратный элемент
    \item Если выполняется $\alpha \bigodot x = 0$, то либо $\alpha$ равно нулю, либо $x$ является нулевым элементом
    \item Разностью элементов называют операцию, обратную сложению
\end{enumerate}

\subsubsection{Линейная комбинация элементов}

\textbf{Линейной комбинацией элементов называют} элемент $\alpha_1 \bigodot x_1 \bigoplus \alpha_2 \bigodot x_2 + ... + \alpha_n \bigodot x_n = \theta$ $(*)$, где $\alpha_i$ - действительные числа 

\hfill

Если равенство $(*)$ выполняется только при всех $a_i$ равных нулю, то все элементы $x_i$ являются \textbf{линейно независимыми}. Иначе эти элементы называются \textbf{линейно зависимыми}

\hfill

Для того, чтобы система векторов \textbf{была линейно зависимой}, необходимо и достаточно, чтобы хотя бы один вектор являлся линейной комбинацией остальных.

\hfill

\textbf{Доказательство необходимости. } Предполагаем, что наши системы векторов являются линейно зависимыми. Не нарушим общность, если предположим, что первый элемент отличен от нуля. Тогда мы можем записать:

$\alpha_1 x_1 = -\alpha_2 x_2 - \alpha_3 x_3 - ... - \alpha_n x_n \Longleftrightarrow x_1 = -\frac{\alpha_2}{\alpha_1} x_2 - \frac{\alpha_3}{\alpha_1} x_3 - ... - \frac{\alpha_n}{\alpha_1} x_n$

Что и требовалось доказать

\hfill

\textbf{Доказательство достаточности} тоже легко сочинить.

\subsubsection{Размерность линейного пространства}

Если существует натуральное число $n$ такое, что наше пространство содержит $n$ линейно независимых векторов, а прибавление любого лишнего вектора делает эти вектора линейно зависимыми, тогда мы говорим, что линейное пространство \textbf{имеет размерность $n$}

\subsubsection{Базис линейного пространства}

Упорядоченная система векторов $e_1, e_2, ..., e_n$ называется базисом линейного пространства, если

\begin{enumerate}
    \item Эти вектора являются линейно независимыми
    \item Любой вектор линейного пространства можно выразить как линейную комбинацию из этих векторов: $x = \xi_1 e_1 + \xi_2 e_2 + ... + \xi_n e_n$, где $\xi_i$ - координаты вектора $e$ в базисе $e_1,e_2,...,e_n$
\end{enumerate}

\textbf{Замечание 1. } Координаты в разложении по конкретному базису определяются однозначно.

\textbf{Замечание 2. } В линейном пространстве существует бесконечное множество базисов. Если линейное пространство имеет размерность $n$, то базис будет состоять из $n$ векторов.

\textbf{Замечание 3.} На плоскости в качестве базиса могут использоваться любых два неколлинеарных вектора

\hfill

\textbf{Пример 1.}
\hfill

Например, если мы работаем на плоскости, то имеем ортонормированный ($\vec{i}, \vec{j}$) базис. Дано $e_1 = 2 \vec{i} + \vec{j}, e_2 = -1\vec{i} + 2\vec{j}, p = 3\vec{i} + 5\vec{j}$.

Запишем вектор $p$ в новом базисе $e_1, e_2$: $\overline{p} = \xi_1 \overline{e_1} + \xi_2 \overline{e_2}$

\hfill

$\begin{pmatrix}
    3 \\
    5
\end{pmatrix} = \xi_1 \begin{pmatrix}
    2 \\
    1
\end{pmatrix} + \xi_2 \begin{pmatrix}
    -1 \\
    2
\end{pmatrix}$

\begin{equation}
    \begin{cases}
        3 = 2x - y \\
        5 = x + 2y
    \end{cases}
\end{equation}

Решая систему уравнений, получим: $x = 2.2, y = 1.4$

\hfill

\textbf{Ответ}: $\overline{p} = 2.2 \overline{e_1} + 1.4 \overline{e_2}$

\hfill

\textbf{Свойства базиса линейного пространства}

Пусть мы рассматриваем любое $n$-мерное линейное пространство, и $e_1, e_2, ..., e_n$ - базис в $n$-мерном линейном пространстве.

\begin{enumerate}
    \item $\alpha = \xi_1 e_1 + \xi_2 e_2 + ... + \xi_n e_n, b = \lambda_1 e_1 + \lambda_2 e_2 + ... + \lambda_n e_n$, то $\overline{a} + \overline{b} = (\xi_1 + \lambda_1) e_1 + (\xi_2 + \lambda_2) e_2 + ... + (\xi_n + \lambda_n) e_n$
    \item $\alpha \vec{a} = \alpha \xi_1 \overline{e_1} + ... + \alpha \xi_n \overline{e_n}$
\end{enumerate}

\pagebreak
\subsection{Векторная алгебра}

\subsubsection{Скалярное произведение векторов}

Скалярное произведение векторов - \textbf{число}.

\hfill 

$a * b = |\vec{a}| * |\vec{b}| \cos \alpha$, где $\alpha$ - угол между данными векторами.

\hfill

Обладает следующими свойствами:

\begin{multienumerate}
    \mitemxx{$a * b = b * a$}{$(\alpha a) * b$}
    \mitemxx{$(a + b) * e = a c + b c$}{$a * a \ge 0$}
\end{multienumerate}

Допустим, имеем $\alpha = \{ x_a; y_a; z_a \}, b = \{ x_b; y_b; z_b \}$, то $ab = x_a x_b + y_a y_b + z_a z_b$

\hfill

$\cos \alpha = \frac{a b}{|a| |b|} = \frac{x_a x_b + y_a y_b + z_a z_b}{\sqrt{x_a^2 + y_a^2 + z_a^2} \sqrt{x_b^2 + y_b^2 + z_b^2}}$

\hfill

Необходимым и достаточным \textbf{условием перпендикулярности векторов} $a$ и $b$ является равенство нулю их скалярного произведения, $a * b > 0$ - угол острый, $a * b < 0$ - угол тупой

\subsubsection{Скалярная проекция вектора}

$\proj_{b} \overline{a} = X_{\cos \alpha} + Y_{\cos \beta} + Z_{\cos \gamma}, \proj_{x} \vec{a} = a * i, \proj_{y} a = a * j$, где $\alpha, \beta, \gamma$ - углы, которые в сост. с коор. осями.

\hfill

$e = \{ \cos \alpha, \cos \beta, \cos \gamma \}$ - вектор в направлении $b$

\hfill 

$\proj_{b} a = |a| \cos \alpha = | a | \frac{a b}{|a| |b|} = \frac{a * b}{|b|}$

\end{flushleft}

\end{document}