\documentclass{article}
\usepackage[utf8]{inputenc}

\usepackage[T2A]{fontenc}
\usepackage[utf8]{inputenc}
\usepackage[russian]{babel}

\usepackage{amsmath}

\def\vec{\ensuremath\overrightarrow}

\title{Алгебра и геометрия}
\author{Лисид Лаконский}
\date{October 2022}

\begin{document}

\maketitle

\tableofcontents
\pagebreak

\section{Алгебра и геометрия - 24.10.2022}

\subsection{Прямая на плоскости}

\begin{flushleft}

Ненулевой вектор $\vec{S}$, параллельный прямой l, называется направляющим вектором прямой.

Ненулевой вектор $\vec{N}$, перпендикулярный прямой l, называется вектором нормали прямой l.

\subsubsection{Уравнения прямой на плоскости}

\begin{enumerate}
    \item $y = k x + b$, где $k = \tg \alpha$
    \item $y - y_0 = k (x - x_0)$, где $k = \tg \alpha$ - уравнение прямой, проходящей через точку $M(x_0; y_0)$ с заданным угловым коэффициентом $k$
    \item $A x + B y + C = 0, A^2 + B^2 \ne 0$ - общее уравнение прямой (вектор нормали прямой: $\vec{N} = \{ A; B \}$)
    \item $A (x - x_0) + B(y - y_0) = 0, A^2 + B^2 \ne 0$ - уравнение прямой, проходящей через точку $M(x_0; y_0)$ с заданным вектором нормали $\vec{N} = \{ A; B \}$
    \item $\frac{x - x_0}{m} = \frac{y - y_0}{n}, m^2 + n^2 \ne 0$ - каноническое уравнение прямой (направляющий вектор $\vec{S} = \{ m; n \}$, $M(x_0; y_0)$
    \item $\frac{x - x_1}{x_2 - x_1} = \frac{y - y_1}{y_2 - y_1}$ - уравнение прямой, проходящей через заданные точки $M_1(x_1; y_1)$ и $M_2(x_2; y_2)$
\end{enumerate}

\subsubsection{Угол между двумя прямыми}

$l_1: y = k_1 x + b$, $l_2: y = k_2 x + b_2$

$\tg \alpha = \pm \frac{k_2 - k_1}{1 + k_1 * k_2} \ge 0$

\begin{enumerate}
    \item $l_1 \perp l_2 \Longleftrightarrow k_2 = -\frac{1}{k_1}$
    \item $l_1 \parallel l_2 \Longleftrightarrow k_1 = k_2$
\end{enumerate}

\subsubsection{Примеры}

\textbf{Пример 1.} Найти координаты центра описанной около треугольника $ABC$, где $A(0, 3)$, $B(2; 5)$, $C(-2; 7)$.

Пусть точка $D$ - середина $AB$, ее координаты - $D(1; 4)$, точка $P$ - середина $BC$, ее координаты - $P(0; 6)$

\hfill

$\vec{N} = \vec{AB} = \{ 2; 2 \}, 2(x - 1) + 2(y - 4) = 0 \Longleftrightarrow 2x + 2y - 10 = 0$

$\vec{BC} = \{ -4; 2 \}, -4(x - 0) + 2 (y - 6) = 0 \Longleftrightarrow -4x + 2y - 12 = 0$

\hfill

\begin{equation}
    \begin{cases}
        2x + 2y - 10 = 0 \\
        -4x + 2y - 12 = 0
    \end{cases}
\end{equation}

\textbf{Ответ:} $S(-\frac{1}{3}; \frac{16}{3})$

\hfill

\textbf{Пример 2.} Даны две вершины $A_1(2; 4)$, $A_2(3; 1)$, $\triangle A_1 A_2 A_3$, $N(4; 0)$ - точка пересечения медиан.

Составить уравнение сторон этого треугольника и найти точку третьей вершины.

\hfill

$X_N = \frac{x_1 + x_2 + x_3}{3}, y_N = \frac{y_1 + y_2 + y_3}{3}$ - координаты точки пересечения медиан.

$x_3 = 3X_N - x_1 - x_2 = 12 - 2 - 3 = 7$, $y_3 = 3Y_N - y_1 - y_2 = -5$

$A_3(7; -5)$ - координаты третьей вершины

\hfill

$(A_1 A_2): \frac{x - x_1}{x_2 - x_1} = \frac{y - y_1}{y_2 - y_1} \Longleftrightarrow \frac{x - 2}{3 - 2} = \frac{y - 4}{1 - 4} \Longleftrightarrow -3x + y = y - 4 \Longleftrightarrow -3x - y + 10 = 0$

\hfill

$(A_2 A_3): 3x + 2y - 11 = 0$

\hfill 

$(A_1 A_3): 9x + 5y - 38 = 0$

\hfill

\textbf{Пример 3.} Даны вершины $A_1 (1; 0)$, $A_2(3; 5)$ треугольника $\triangle A_1 A_2 A_3$, $N(-1; 3)$ - точка пересечения высот данного треугольника.

Определить координаты $A_3$.

\hfill

$\vec{A_1 N} = \{ -2; 3 \}$, $-2 (x - 3) + 3 (y - 5) = 0 \longleftrightarrow -2x + 3x - 9 = 0$

$\vec{A_2 N} = \{ -4; -2 \}$, $A_2 N \perp (A_1 A_3)$

\hfill

Уравнение прямой $A_1 A_3$: $-4 (x - 1) - 2y \Longleftrightarrow -4x -2y + 4 = 0 \Longleftrightarrow 2x + y - 2 = 0$

\hfill

\begin{equation}
    \begin{cases}
        -2x + 3y - 9 = 0 \\
        2x + y - 2 = 0
    \end{cases}
\end{equation}

\textbf{Точка пересечения} - $A_3 (-\frac{3}{8}; \frac{11}{4})$ 

\hfill

$y = \frac{2x + 9}{3}, -2 = -\frac{1}{k}, k = \frac{1}{2}, \vec{n} = -\frac{1}{2} \vec{A_2 N} = \vec{A_3 N} = \{ 2; 1 \}$

\hfill

\textbf{Пример 4.} 

\end{flushleft}

\end{document}