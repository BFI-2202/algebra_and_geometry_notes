\documentclass{article}
\usepackage[utf8]{inputenc}

\usepackage[T2A]{fontenc}
\usepackage[utf8]{inputenc}
\usepackage[russian]{babel}

\usepackage{amsmath}

\title{Алгебра и геометрия}
\author{Лисид Лаконский}
\date{October 2022}

\begin{document}

\maketitle

\section{Алгебра и геометрия - 04.10.2022}

\subsection{Ранг матрицы}

Пусть дана матрица $A$ размера $m * n$.

Возьмем любые $k$ ($k \le min(n;m)$) строк и $k$ столбцов матрицы $A$.

На их пересечении стоят элементы, образующие определитель $k$-того порядка, который и называется минором $k$-го порядка.

Под минором 1-го порядка матрицы $A$ понимается любой элемент.

Рангом $r$ матрицы $A$ называется наивысший порядок минора матрицы $A$, отличный от нуля.

Следовательно, если у нас матрица из четырех строк и трех столбцов, максимальный минор может быть три на три. Но если все они равны нулю, то мы не можем сказать, что ранг матрицы равен нулю.

Из определения следует:

\begin{enumerate}
    \item $r$ - целое число ($0 \le r \le min(m;n)$)
    \item Все миноры ($r + 1$) порядка либо нулевые, либо не существуют. 
\end{enumerate}

$A = \begin{pmatrix}
    2 & 1 & 3 & 7 \\
    0 & 4 & -1 & 0 \\
    0 & 0 & 8 & 1
\end{pmatrix}, r(A) = 3$

Миноры 1-го порядка: любой элемент матрицы. 

Миноры второго порядка: любой определитель этой матрицы 2x2: $\begin{pmatrix} 2 & 1 \\ 0 & 4 \end{pmatrix}, \begin{pmatrix} 1 & 3 \\ 4 & -1 \end{pmatrix}$

Миноры третьего порядка: $\begin{pmatrix} 2 & 1 & 3 \\ 0 & 4 & -1 \\ 0 & 0 & 8\end{pmatrix}, \det A = 64 \ne 0$

Минора четвертого порядка у данной матрицы не существует.

\subsection{Действия над матрицами}

\begin{enumerate}
    \item Умножение строки или столбца на число, отличное от нуля.
    \item Сложение: прибавление к одной строке (столбцу) другой, умноженной на число.
    \item Перемещение (замена местами) двух строк или двух столбцов.
    \item Вычеркивание нулевой строки или столбца.
\end{enumerate}

$
A = \begin{pmatrix}
 2 & 3 & 4 & 5 \\
 3 & 5 & 2 & 4 \\
 5 & 9 & -2 & 2 \\
\end{pmatrix} = \begin{pmatrix}
    1 & 2 & -2 & -1 \\
    3 & 5 & 2 & 4 \\
    5 & 9 & -2 & 2
\end{pmatrix} = \begin{pmatrix}
    1 & 2 & -2 & -1 \\
    0 & -1 & 8 & 7 \\
    0 & -1 & 8 & 7
\end{pmatrix} = \begin{pmatrix}
 1 & 2 & -2 & -1 \\
 0 & -1 & 8 & 7
\end{pmatrix}, \begin{vmatrix} 1 & 2 \\ 0 & -1 \end{vmatrix} = -1 \ne 0
$

\subsection{Теорема Кронекера-Капелли}

Рассмотрим систему $m$ линейных уравнений с $n$ неизвестными:

\begin{equation}
    \begin{cases}
        a_{11} * x_1 + a_{12} * x_2 + ... + a_{1n} * x_n = b_1 \\
        ... \\
        a_{m1} * x_1 + a_{m2} * x_2 + ... + a_{mn} * x_n = b_m
    \end{cases}
\end{equation}

$A = \begin{pmatrix}
    a_{11} & ... & a_{1n} \\ 
    a_{m1} & ... & a_{m n}
\end{pmatrix}, B = \begin{pmatrix}
    b_1 \\
    ... \\
    b_m
\end{pmatrix}, X = \begin{pmatrix}
    ?
\end{pmatrix}, (A|B) = \begin{pmatrix}
    a_{11} & ... & a_{1 n} & b_1 \\
    a_{m1} & ... & a_{m n} & b_m
\end{pmatrix}$

Система называется совместной, если она имеет решение. (*) Операции только над строками.

$$r(A) = r(A|B) \equiv r$$

Если $r = n$, то система имеет единственное решение. Если $r < n$, то система имеет бесконечное множество решений, зависящих от ($n - r$) свободных неизвестных.

\subsection{Метод Гаусса}

Если столбец $B = \begin{pmatrix} b_1 \\ ... \\ b_n \end{pmatrix}$ свободных членов - нулевой, то система называется однородной.

Однородная система всегда имеет решение, и она всегда совместна, так как имеет тривиальное (нулевое) решение: $x_1 = 0, x_2 = 0, ..., x_n = 0$.

Если в однородной системе число неизвестных $n$ равно числу уравнений $m$, то она имеет ненулевое решение тогда и только тогда, когда определитель системы равен нулю.

\begin{equation}
    \begin{cases}
        x_1 + 5x_2 + 4x_3 - x_4 = 2 \\
        2x_1 - x_2 - x_3 + 2x_4 = 3 \\
        3x_1 + 4x_2 + 3x_3 + x_4 = 5
    \end{cases}\,.
\end{equation}

$A = \begin{pmatrix}
    1 & 5 & 4 & -1 \\
    2 & -1 & -1 & 2 \\
    3 & 4 & 3 & 1
\end{pmatrix}, B = \begin{pmatrix}
    2 \\
    3 \\
    5
\end{pmatrix}, (A|B) = \begin{pmatrix}
    1 & 5 & 4 & -1 & 2 \\
    2 & -1 & -1 & 2 & 3 \\
    3 & 4 & 3 & 1 & 5
\end{pmatrix} = \begin{pmatrix}
    1 & 5 & 4 & -1 & 2 \\
    0 & -11 & -9 & 4 & -1 \\
    0 & -11 & -9 & 4 & -1
\end{pmatrix} = \begin{pmatrix}
    1 & 5 & 4 & -1 & 2 \\
    0 & -11 & -9 & 4 & -1
\end{pmatrix}, \begin{vmatrix} 1 & -5 \\ 0 & -11 \end{vmatrix} = -11 \ne 0, r(A|B) = 2 = r(A)$

$r < n \rightarrow$ система имеет бесконечное кол-во решений, зависящих от (4 - 2) = 2 свободных неизвестных.

Пусть $\begin{vmatrix}
    1 & 5 \\
    0 & -11
\end{vmatrix}$ - базисный минор, тогда $x_1$ и $x_2$ - базисные члены, а $x_3$ и $x_4$ - свободные.

$
\begin{pmatrix}
    1 & 5 & 4 & -1 & 2 \\
    0 & -11 & -9 & 4 & -1
\end{pmatrix}
$

Преобразуем данную матрицу, делая по главной диагонали базисного минора единицы, а по побочной нули.

$\begin{pmatrix}
    1 & 5 & 4 & -1 & 2 \\
    0 & 1 & \frac{9}{11} & -\frac{4}{11} & \frac{1}{11}
\end{pmatrix} = \begin{pmatrix}
    10 & -\frac{-1}{11} & \frac{9}{11} & \frac{17}{11} \\
    0 & 1 & \frac{9}{11} & -\frac{4}{11} & \frac{1}{11}
\end{pmatrix}$

Выпишем в виде системы уравнений:

\begin{equation}
    \begin{cases}
        1x_1 + 0x_2 - \frac{1}{11}x_3 + \frac{9}{11}x_4 = \frac{17}{11} \\
        0x_1 + 1_x2 + \frac{9}{11}x_3 - \frac{4}{11}x_4 = \frac{1}{11}
    \end{cases}
\end{equation}

\begin{equation}
    \begin{cases}
        x_1 = \frac{1}{11}x_3 - \frac{9}{11}x_4 + \frac{17}{11} \\
        x_2 = -\frac{9}{11}x_3 + \frac{4}{11}x_4 + \frac{1}{11}
    \end{cases}
\end{equation}

Пусть $x_3 = c_1$, а $x_4 = c_2$ ($c_1, c_2 \in R$), тогда наша система приобретает вид:

\begin{equation}
    \begin{cases}
        x_1 = \frac{1}{11}c_1 - \frac{9}{11}c_2 + \frac{17}{11} \\
        x_3 = -\frac{9}{11}c_1 + \frac{4}{11}c_2 + \frac{1}{11} \\
        x_3 = c_1 \\
        c_4 = c_2
    \end{cases}
\end{equation}

Исследуем на совместность систему

\begin{equation}
    \begin{cases}
        2x_1 + 2x_2 + x_3 = 6 \\
        x_1 + 2x_2 + 4x_3 = 4 \\
        3x_1 + 4x_2 + 5x_3 = 9
    \end{cases}
\end{equation}

$(A|B) = \begin{pmatrix}
    2 & 2 & 1 & 6 \\
    1 & 2 & 4 & 4 \\
    3 & 4 & 5 & 9
\end{pmatrix} = \begin{pmatrix}
    1 & 2 & 4 & 4 \\
    2 & 2 & 1 & 6 \\
    3 & 4 & 5 & 9
\end{pmatrix} = \begin{pmatrix}
    1 & 2 & 4 & 4 \\
    0 & -2 & -7 & -2 \\
    0 & -2 & -7 & -3
\end{pmatrix} = \begin{pmatrix}
    1 & 2 & 4 & 4 \\
    0 & -2 & -7 & -2 \\
    0 & 0 & 0 & 0 & -1
\end{pmatrix}, r(A|B) = 3, r(A) = 2$

Следовательно, система несовместна и решений не имеет.

\end{document}
