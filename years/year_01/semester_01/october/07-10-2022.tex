\documentclass{article}
\usepackage[utf8]{inputenc}

\usepackage[T2A]{fontenc}
\usepackage[utf8]{inputenc}
\usepackage[russian]{babel}

\usepackage{amsmath}

\title{Алгебра и геометрия}
\author{Лисид Лаконский}
\date{October 2022}

\begin{document}

\maketitle

\section{Алгебра и геометрия - 07.10.2022}

\subsection{Собственные значения и собственные векторы матрицы}

Матрицы могут представляться на плоскости - для этого нужны собственные значения и собственные векторы.

Пусть дана квадратная матрица $A$ $n$-ого порядка. Ненулевой вектор $X = \begin{pmatrix}
    x_1 \\
    ... \\
    x_n
\end{pmatrix}$ называется собственным вектором матрицы $A$, если под действием этой матрицы он переходит в коллинеарный ему:

$$
A * X = \lambda X, \lambda \in R
$$

Где $\lambda$ - собственное значение соответствующего ему вектора матрицы $A$.

Для нахождения $\lambda$ составляют характеристическое уравнение:

$$
|A - \lambda E| = 0
$$

Если $\lambda_0$ - сосбтвенное значение матрицы $A$, то соответствующие собственные векторы находим из системы однородных линейных уравнений.

(*) Однородными называются системы, где матрица-столбец свободных членов $B$ полностью состоит из нулей

$$
(A - \lambda_0 E ) * X = 0
$$

\subsubsection{Примеры}

$$A = \begin{pmatrix}
    4 & -2 & -1 \\
    -1 & 3 & -1 \\
    1 & -2 & 2
\end{pmatrix}
$$

Составим характеристическое уравнение $|A - \lambda E| = 0$:


$$
\begin{vmatrix}
    4-\lambda & -2 & -1 \\
    -1 & 3-\lambda & -1 \\
    1 & -2 & 2 - \lambda
\end{vmatrix}
$$

$(4 - \lambda)(3 - \lambda)(2 - \lambda) - 2 + 2 + 3 - \lambda - 8 + 2 \lambda - 4 + 2 \lambda = (12 - 7 \lambda + \lambda^2)(2 - \lambda) + 3\lambda - 9 = 24 - 12 \lambda - 14 \lambda + 7 \lambda^2 + 2 \lambda^2 - \lambda^3 + 3\lambda - 9 = -\lambda^3 - 6\lambda^2 - 23\lambda + 15 = 0$

$\lambda_1 = 1$, вынесем общий множитель:

$
\frac{-\lambda^3 + 9\lambda^2 - 23\lambda + 15}{\lambda - 1} = (\lambda - 1)(-\lambda^2 + 8\lambda - 15)
$

Решаем через дискриминант или через теорему Виета: что угодно.

Итого имеем:

$\lambda_1 = 1, \lambda_2 = 3, \lambda_3 = 5$

Найдем теперь собственные векторы.

Пусть $\lambda = \lambda_1 = 1$, тогда $(A - \lambda E) * X = 0$:

\begin{equation}
    \begin{cases}
        (4-1)x_1 - 2x_2 - x_3 = 0 \\
        -x_1 + (3 - 1)x_2 - x_3 = 0 \\
        x_1 - 2x_2 + (2 - 1)x_3 = 0
    \end{cases}
\end{equation}

В матричном виде:

$\begin{pmatrix}
    3 & -2 & - 1 \\
    -1 & 2 & -1 \\
    1 & -2 & 1
\end{pmatrix} = \begin{pmatrix}
    3 & -2 & -1 \\
    1 & -2 & 1
\end{pmatrix} = \begin{pmatrix}
    1 & -2 & 1 \\
    0 & 4 & -4
\end{pmatrix} = \begin{pmatrix}
    1 & -2 & 1 \\ 
    0 & 1 & -1
\end{pmatrix} = \begin{pmatrix}
    1 & 0 & -1 \\
    0 & 1 & -1
\end{pmatrix}$

Преобразуем обратно в систему:

\begin{equation}
    \begin{cases}
        x_1 - x_3 = 0 \\
        x_2 - x_3 = 0 
    \end{cases}
\end{equation}

Отсюда видим, что $x_1 = x_3, x_2 = x_3$

Пусть $x_1 = 1$, тогда $x_1$ = $\begin{pmatrix}
    1 \\
    1 \\
    1
\end{pmatrix}$, видим что $X_1 = C_1 \begin{pmatrix}
    1 \\
    1 \\
    1
\end{pmatrix}$

Последовательно найдем теперь второй и третий собственный векторы: $X_2$ и $X_3$.

Пусть $\lambda = \lambda_2 = 3$, тогда $(A - \lambda E) * X = 0$

\begin{equation}
    \begin{cases}
        (4-3)x_1 - 2x_2 - x_3 = 0 \\
        -x_1 + (3 - 3)x_2 - x_3 = 0 \\
        x_1 - 2x_2 + (2 - 3)x_3 = 0
    \end{cases}
\end{equation}

Преобразуем в матричный вид: $\begin{pmatrix}
    1 & -2 & -1 \\
    -1 & 0 & 1 \\
    1 & -2 & 1
\end{pmatrix} = \begin{pmatrix}
    1 & -2 & -1 \\ 1 & 0 & 1
\end{pmatrix} = \begin{pmatrix}
    2 & -2 & 0 \\
    1 & 0 & 1
\end{pmatrix} = \begin{pmatrix}
    1 & -1 & 0 \\
    1 & 0 & 1
\end{pmatrix}$

Из этого видно, что $x_1 = x_2, x_1 = -x_3$

$X_2 = C_2 \begin{pmatrix}
    1 \\
    1 \\
    -1
\end{pmatrix}$

Найдем третий собственный вектор.

Пусть $\lambda = \lambda_3 = 5$, тогда $(A - \lambda E) * X = 0$

\begin{equation}
    \begin{cases}
        (4 - 5)x_1 - 2x_2 - x_3 = 0 \\
        -x_1 + (3 - 5)x_2 - x_3 = 0 \\
        x_1 - 2x_2 + (2 - 5)x_3 = 0
    \end{cases}
\end{equation}

Запишем данную систему уравнений в матричном виде: $\begin{pmatrix}
    -1 & -2 & -1 \\
    -1 & -2 & -1 \\
    1 & -2 & -3
\end{pmatrix} = \begin{pmatrix}
    -1 & -2 & -1 \\
    1 & -2 & -3
\end{pmatrix} = \begin{pmatrix}
    -1 & -2 & -1 \\
    0 & -4 & -4
\end{pmatrix}$

Чему соответствует следующая система уравнений:

\begin{equation}
    \begin{cases}
        -x_1 - 2x_2 - x_3 = 0 \\
        -4x_2 - 4x_3 = 0
    \end{cases}
\end{equation}

Я зашел в какую-то фигню, где-то ошибся, но, в общем, ответ должен получиться следующий: $X_3 = C_3 \begin{pmatrix}
    1 \\
    -1 \\
    1
\end{pmatrix}$

\subsection{Векторная алгебра. Операции над векторами}

Вектором $\overrightarrow{AB}$ называется направленный отрезок $AB$, заданный своим началом $A$ и концом $B$.

Длиной (модулем) $|\overrightarrow{AB}|$ вектора $\overrightarrow{AB}$ называется длина отрезка $AB$.

Два вектора называются коллинеарными, если они параллельны одной прямой (параллельны друг другу).

Три вектора называются компланарными, если они параллельны одной плоскости.

Координаты $x, y, z$ вектора $\overrightarrow{a}$ это коэффициенты разложения вектора по базису, то есть по трем некомпланарным векторам, обозначаемым как $\overrightarrow{e_1}, \overrightarrow{e_2}, \overrightarrow{e_3}$.

$\overrightarrow{e_1} = \{1, 0, 0\}, \overrightarrow{e_2} = \{0, 1, 0\}, \overrightarrow{e_2} = \{0, 0, 1\}, \overrightarrow{a} = x * \overrightarrow{e_1} + y * \overrightarrow{e_2} + z * \overrightarrow{e_3}$

Если $\overrightarrow{e_1}, \overrightarrow{e_2}, \overrightarrow{e_3}$ взаимно перпендикулярны и единичные векторы: $\overrightarrow{i}, \overrightarrow{j}, \overrightarrow{k}$, то такой базис называется ортонормированным.

\subsubsection{Пример}

Разложить вектор $\overrightarrow{a} = \{4; 2; 0\}$, если возможно, по векторам $\overrightarrow{p} = \{1; -1; 2\}, \overrightarrow{q} = \{2; 2; -1\}, \overrightarrow{r} = \{3; 7; -7\}$ 

Для того, чтобы это было возможно, должно соблюдаться следующее выражение: $(\overrightarrow{p} * \overrightarrow{q}) * \overrightarrow{r} \ne 0$ - достаточное условие некомпланарности.

$(\overrightarrow{p} * \overrightarrow{q}) * \overrightarrow{r} = \begin{vmatrix}
    1 & -1 & 2 \\
    2 & 2 & -1 \\
    3 & 7 & -7
\end{vmatrix} \ne 0$

$\det X = -14 + 28 + 3 - 12 + 7 - 14 = 2 \ne 0$, следовательно, мы можем разложить данный вектор по трем некомпланарным векторам.

$\overrightarrow{a} = x * \overrightarrow{p} + y * \overrightarrow{q} + z * \overrightarrow{r}$

$\overrightarrow{p} = 1 * \overrightarrow{i} - 1 * \overrightarrow{j} + 2 * \overrightarrow{k}, \overrightarrow{q} = 2 * \overrightarrow{i} + 2 * \overrightarrow{j} - \overrightarrow{k}, \overrightarrow{r} = 3 * \overrightarrow{i} + 7 * \overrightarrow{j} - 7 * \overrightarrow{k}$

$x * \overrightarrow{p} + y * \overrightarrow{q} + z * \overrightarrow{r} = x * \overrightarrow{i} - x * \overrightarrow{j} + 2x\overrightarrow{k} + 2y\overrightarrow{i} + 2y\overrightarrow{i} + 2y\overrightarrow{j} - y\overrightarrow{k} + 3z\overrightarrow{i} + 7z\overrightarrow{j} - 7z\overrightarrow{k}$

Далее для разложения по базису нам необходимо вынести $\overrightarrow{i}, \overrightarrow{j}, \overrightarrow{k}$

$... = (x + 2y + 3z) + \overrightarrow{j}(-x + 2y + 7z) + \overrightarrow{k}(2x - y - 7z)$

$\overrightarrow{a} = x\overrightarrow{p} + y\overrightarrow{q} + z\overrightarrow{r}$

\begin{equation}
    \begin{cases}
        x + 2y + 3z = 4 \\
        -x + 2y + 7z = 2 \\
        2x - y - 7z = 0
    \end{cases}
\end{equation}

Решим данную систему уравнений каким угодно способом, сначала составив расширенную матрицу системы:

$
\begin{pmatrix}
    1 & 2 & 3 & 4 \\
    -1 & 2 & 7 & 2 \\
    2 & -1 & -7 & 0
\end{pmatrix} = \begin{pmatrix}
    1 & 2 & 3 & 4 \\
    1 & -2 & -7 & -2 \\
    1 & 1 & 0 & 2
\end{pmatrix} = \begin{pmatrix}
    1 & 2 & 3 & 4 \\
    1 & 1 & 0 & 2 \\
    0 & -3 & -7 & -4
\end{pmatrix} = \begin{pmatrix}
    1 & -1 & -4 & 0 \\
    1 & 1 & 0 & 2 \\
    0 & -3 & -7 & -4
\end{pmatrix}
$

\begin{equation}
\begin{cases}
    x_1 = -x_2 + 2 \\
    x_3 = \frac{4}{7} - \frac{3}{7}x_2 \\
    x_1 = -x_2-4x_3 = 0
\end{cases}
\end{equation}

\begin{equation}
    \begin{cases}
        x_2 = -1 \\
        x_1 = 3 \\
        x_3 = 1
    \end{cases}
\end{equation}

Тогда $\overrightarrow{a} = 3\overrightarrow{p} - \overrightarrow{q} + \overrightarrow{r}$

\end{document}
