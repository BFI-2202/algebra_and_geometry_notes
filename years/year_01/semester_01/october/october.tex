\documentclass{article}
\usepackage[utf8]{inputenc}

\usepackage[T2A]{fontenc}
\usepackage[utf8]{inputenc}
\usepackage[russian]{babel}

\usepackage{amsmath}
\usepackage{multienum}

\def\vec{\ensuremath\overrightarrow}
\def\proj{\ensuremath\text{Пр}}

\title{Алгебра и геометрия}
\author{Лисид Лаконский}
\date{October 2022}

\begin{document}

\maketitle

\tableofcontents
\pagebreak

\section{Алгебра и геометрия - 04.10.2022}

\subsection{Ранг матрицы}

Пусть дана матрица $A$ размера $m * n$.

Возьмем любые $k$ ($k \le min(n;m)$) строк и $k$ столбцов матрицы $A$.

На их пересечении стоят элементы, образующие определитель $k$-того порядка, который и называется минором $k$-го порядка.

Под минором 1-го порядка матрицы $A$ понимается любой элемент.

Рангом $r$ матрицы $A$ называется наивысший порядок минора матрицы $A$, отличный от нуля.

Следовательно, если у нас матрица из четырех строк и трех столбцов, максимальный минор может быть три на три. Но если все они равны нулю, то мы не можем сказать, что ранг матрицы равен нулю.

Из определения следует:

\begin{enumerate}
    \item $r$ - целое число ($0 \le r \le min(m;n)$)
    \item Все миноры ($r + 1$) порядка либо нулевые, либо не существуют. 
\end{enumerate}

$A = \begin{pmatrix}
    2 & 1 & 3 & 7 \\
    0 & 4 & -1 & 0 \\
    0 & 0 & 8 & 1
\end{pmatrix}, r(A) = 3$

Миноры 1-го порядка: любой элемент матрицы. 

Миноры второго порядка: любой определитель этой матрицы 2x2: $\begin{pmatrix} 2 & 1 \\ 0 & 4 \end{pmatrix}, \begin{pmatrix} 1 & 3 \\ 4 & -1 \end{pmatrix}$

Миноры третьего порядка: $\begin{pmatrix} 2 & 1 & 3 \\ 0 & 4 & -1 \\ 0 & 0 & 8\end{pmatrix}, \det A = 64 \ne 0$

Минора четвертого порядка у данной матрицы не существует.

\subsection{Действия над матрицами}

\begin{enumerate}
    \item Умножение строки или столбца на число, отличное от нуля.
    \item Сложение: прибавление к одной строке (столбцу) другой, умноженной на число.
    \item Перемещение (замена местами) двух строк или двух столбцов.
    \item Вычеркивание нулевой строки или столбца.
\end{enumerate}

$
A = \begin{pmatrix}
 2 & 3 & 4 & 5 \\
 3 & 5 & 2 & 4 \\
 5 & 9 & -2 & 2 \\
\end{pmatrix} = \begin{pmatrix}
    1 & 2 & -2 & -1 \\
    3 & 5 & 2 & 4 \\
    5 & 9 & -2 & 2
\end{pmatrix} = \begin{pmatrix}
    1 & 2 & -2 & -1 \\
    0 & -1 & 8 & 7 \\
    0 & -1 & 8 & 7
\end{pmatrix} = \begin{pmatrix}
 1 & 2 & -2 & -1 \\
 0 & -1 & 8 & 7
\end{pmatrix}, \begin{vmatrix} 1 & 2 \\ 0 & -1 \end{vmatrix} = -1 \ne 0
$

\subsection{Теорема Кронекера — Капелли}

\begin{theorem}[Теорема Кронекера — Капелли]
Система линейных алгебраических уравнений совместна тогда и только тогда, когда ранг её основной матрицы равен рангу её расширенной матрицы.
\end{theorem}

Рассмотрим систему $m$ линейных уравнений с $n$ неизвестными:

\begin{equation}
    \begin{cases}
        a_{11} * x_1 + a_{12} * x_2 + ... + a_{1n} * x_n = b_1 \\
        ... \\
        a_{m1} * x_1 + a_{m2} * x_2 + ... + a_{mn} * x_n = b_m
    \end{cases}
\end{equation}

$A = \begin{pmatrix}
    a_{11} & ... & a_{1n} \\ 
    a_{m1} & ... & a_{m n}
\end{pmatrix}, B = \begin{pmatrix}
    b_1 \\
    ... \\
    b_m
\end{pmatrix}, X = \begin{pmatrix}
    ?
\end{pmatrix}, (A|B) = \begin{pmatrix}
    a_{11} & ... & a_{1 n} & b_1 \\
    a_{m1} & ... & a_{m n} & b_m
\end{pmatrix}$

Система называется совместной, если она имеет решение. (*) Операции только над строками.

$$r(A) = r(A|B) \equiv r$$

Если $r = n$, то система имеет единственное решение. Если $r < n$, то система имеет бесконечное множество решений, зависящих от ($n - r$) свободных неизвестных.

\subsection{Метод Гаусса}

Если столбец $B = \begin{pmatrix} b_1 \\ ... \\ b_n \end{pmatrix}$ свободных членов - нулевой, то система называется однородной.

Однородная система всегда имеет решение, и она всегда совместна, так как имеет тривиальное (нулевое) решение: $x_1 = 0, x_2 = 0, ..., x_n = 0$.

Если в однородной системе число неизвестных $n$ равно числу уравнений $m$, то она имеет ненулевое решение тогда и только тогда, когда определитель системы равен нулю.

\begin{equation}
    \begin{cases}
        x_1 + 5x_2 + 4x_3 - x_4 = 2 \\
        2x_1 - x_2 - x_3 + 2x_4 = 3 \\
        3x_1 + 4x_2 + 3x_3 + x_4 = 5
    \end{cases}\,.
\end{equation}

$A = \begin{pmatrix}
    1 & 5 & 4 & -1 \\
    2 & -1 & -1 & 2 \\
    3 & 4 & 3 & 1
\end{pmatrix}, B = \begin{pmatrix}
    2 \\
    3 \\
    5
\end{pmatrix}, (A|B) = \begin{pmatrix}
    1 & 5 & 4 & -1 & 2 \\
    2 & -1 & -1 & 2 & 3 \\
    3 & 4 & 3 & 1 & 5
\end{pmatrix} = \begin{pmatrix}
    1 & 5 & 4 & -1 & 2 \\
    0 & -11 & -9 & 4 & -1 \\
    0 & -11 & -9 & 4 & -1
\end{pmatrix} = \begin{pmatrix}
    1 & 5 & 4 & -1 & 2 \\
    0 & -11 & -9 & 4 & -1
\end{pmatrix}, \begin{vmatrix} 1 & -5 \\ 0 & -11 \end{vmatrix} = -11 \ne 0, r(A|B) = 2 = r(A)$

$r < n \rightarrow$ система имеет бесконечное кол-во решений, зависящих от (4 - 2) = 2 свободных неизвестных.

Пусть $\begin{vmatrix}
    1 & 5 \\
    0 & -11
\end{vmatrix}$ - базисный минор, тогда $x_1$ и $x_2$ - базисные члены, а $x_3$ и $x_4$ - свободные.

$
\begin{pmatrix}
    1 & 5 & 4 & -1 & 2 \\
    0 & -11 & -9 & 4 & -1
\end{pmatrix}
$

Преобразуем данную матрицу, делая по главной диагонали базисного минора единицы, а по побочной нули.

$\begin{pmatrix}
    1 & 5 & 4 & -1 & 2 \\
    0 & 1 & \frac{9}{11} & -\frac{4}{11} & \frac{1}{11}
\end{pmatrix} = \begin{pmatrix}
    10 & -\frac{-1}{11} & \frac{9}{11} & \frac{17}{11} \\
    0 & 1 & \frac{9}{11} & -\frac{4}{11} & \frac{1}{11}
\end{pmatrix}$

Выпишем в виде системы уравнений:

\begin{equation}
    \begin{cases}
        1x_1 + 0x_2 - \frac{1}{11}x_3 + \frac{9}{11}x_4 = \frac{17}{11} \\
        0x_1 + 1_x2 + \frac{9}{11}x_3 - \frac{4}{11}x_4 = \frac{1}{11}
    \end{cases}
\end{equation}

\begin{equation}
    \begin{cases}
        x_1 = \frac{1}{11}x_3 - \frac{9}{11}x_4 + \frac{17}{11} \\
        x_2 = -\frac{9}{11}x_3 + \frac{4}{11}x_4 + \frac{1}{11}
    \end{cases}
\end{equation}

Пусть $x_3 = c_1$, а $x_4 = c_2$ ($c_1, c_2 \in R$), тогда наша система приобретает вид:

\begin{equation}
    \begin{cases}
        x_1 = \frac{1}{11}c_1 - \frac{9}{11}c_2 + \frac{17}{11} \\
        x_3 = -\frac{9}{11}c_1 + \frac{4}{11}c_2 + \frac{1}{11} \\
        x_3 = c_1 \\
        c_4 = c_2
    \end{cases}
\end{equation}

Исследуем на совместность систему

\begin{equation}
    \begin{cases}
        2x_1 + 2x_2 + x_3 = 6 \\
        x_1 + 2x_2 + 4x_3 = 4 \\
        3x_1 + 4x_2 + 5x_3 = 9
    \end{cases}
\end{equation}

$(A|B) = \begin{pmatrix}
    2 & 2 & 1 & 6 \\
    1 & 2 & 4 & 4 \\
    3 & 4 & 5 & 9
\end{pmatrix} = \begin{pmatrix}
    1 & 2 & 4 & 4 \\
    2 & 2 & 1 & 6 \\
    3 & 4 & 5 & 9
\end{pmatrix} = \begin{pmatrix}
    1 & 2 & 4 & 4 \\
    0 & -2 & -7 & -2 \\
    0 & -2 & -7 & -3
\end{pmatrix} = \begin{pmatrix}
    1 & 2 & 4 & 4 \\
    0 & -2 & -7 & -2 \\
    0 & 0 & 0 & 0 & -1
\end{pmatrix}, r(A|B) = 3, r(A) = 2$

Следовательно, система несовместна и решений не имеет.

\pagebreak
\section{Алгебра и геометрия - 07.10.2022}

\subsection{Собственные значения и собственные векторы матрицы}

Матрицы могут представляться на плоскости - для этого нужны собственные значения и собственные векторы.

Пусть дана квадратная матрица $A$ $n$-ого порядка. Ненулевой вектор $X = \begin{pmatrix}
    x_1 \\
    ... \\
    x_n
\end{pmatrix}$ называется собственным вектором матрицы $A$, если под действием этой матрицы он переходит в коллинеарный ему:

$$
A * X = \lambda X, \lambda \in R
$$

Где $\lambda$ - собственное значение соответствующего ему вектора матрицы $A$.

Для нахождения $\lambda$ составляют характеристическое уравнение:

$$
|A - \lambda E| = 0
$$

Если $\lambda_0$ - сосбтвенное значение матрицы $A$, то соответствующие собственные векторы находим из системы однородных линейных уравнений.

(*) Однородными называются системы, где матрица-столбец свободных членов $B$ полностью состоит из нулей

$$
(A - \lambda_0 E ) * X = 0
$$

\subsubsection{Примеры}

$$A = \begin{pmatrix}
    4 & -2 & -1 \\
    -1 & 3 & -1 \\
    1 & -2 & 2
\end{pmatrix}
$$

Составим характеристическое уравнение $|A - \lambda E| = 0$:


$$
\begin{vmatrix}
    4-\lambda & -2 & -1 \\
    -1 & 3-\lambda & -1 \\
    1 & -2 & 2 - \lambda
\end{vmatrix}
$$

$(4 - \lambda)(3 - \lambda)(2 - \lambda) - 2 + 2 + 3 - \lambda - 8 + 2 \lambda - 4 + 2 \lambda = (12 - 7 \lambda + \lambda^2)(2 - \lambda) + 3\lambda - 9 = 24 - 12 \lambda - 14 \lambda + 7 \lambda^2 + 2 \lambda^2 - \lambda^3 + 3\lambda - 9 = -\lambda^3 - 6\lambda^2 - 23\lambda + 15 = 0$

$\lambda_1 = 1$, вынесем общий множитель:

$
\frac{-\lambda^3 + 9\lambda^2 - 23\lambda + 15}{\lambda - 1} = (\lambda - 1)(-\lambda^2 + 8\lambda - 15)
$

Решаем через дискриминант или через теорему Виета: что угодно.

Итого имеем:

$\lambda_1 = 1, \lambda_2 = 3, \lambda_3 = 5$

Найдем теперь собственные векторы.

Пусть $\lambda = \lambda_1 = 1$, тогда $(A - \lambda E) * X = 0$:

\begin{equation}
    \begin{cases}
        (4-1)x_1 - 2x_2 - x_3 = 0 \\
        -x_1 + (3 - 1)x_2 - x_3 = 0 \\
        x_1 - 2x_2 + (2 - 1)x_3 = 0
    \end{cases}
\end{equation}

В матричном виде:

$\begin{pmatrix}
    3 & -2 & - 1 \\
    -1 & 2 & -1 \\
    1 & -2 & 1
\end{pmatrix} = \begin{pmatrix}
    3 & -2 & -1 \\
    1 & -2 & 1
\end{pmatrix} = \begin{pmatrix}
    1 & -2 & 1 \\
    0 & 4 & -4
\end{pmatrix} = \begin{pmatrix}
    1 & -2 & 1 \\ 
    0 & 1 & -1
\end{pmatrix} = \begin{pmatrix}
    1 & 0 & -1 \\
    0 & 1 & -1
\end{pmatrix}$

Преобразуем обратно в систему:

\begin{equation}
    \begin{cases}
        x_1 - x_3 = 0 \\
        x_2 - x_3 = 0 
    \end{cases}
\end{equation}

Отсюда видим, что $x_1 = x_3, x_2 = x_3$

Пусть $x_1 = 1$, тогда $x_1$ = $\begin{pmatrix}
    1 \\
    1 \\
    1
\end{pmatrix}$, видим что $X_1 = C_1 \begin{pmatrix}
    1 \\
    1 \\
    1
\end{pmatrix}$

Последовательно найдем теперь второй и третий собственный векторы: $X_2$ и $X_3$.

Пусть $\lambda = \lambda_2 = 3$, тогда $(A - \lambda E) * X = 0$

\begin{equation}
    \begin{cases}
        (4-3)x_1 - 2x_2 - x_3 = 0 \\
        -x_1 + (3 - 3)x_2 - x_3 = 0 \\
        x_1 - 2x_2 + (2 - 3)x_3 = 0
    \end{cases}
\end{equation}

Преобразуем в матричный вид: $\begin{pmatrix}
    1 & -2 & -1 \\
    -1 & 0 & 1 \\
    1 & -2 & 1
\end{pmatrix} = \begin{pmatrix}
    1 & -2 & -1 \\ 1 & 0 & 1
\end{pmatrix} = \begin{pmatrix}
    2 & -2 & 0 \\
    1 & 0 & 1
\end{pmatrix} = \begin{pmatrix}
    1 & -1 & 0 \\
    1 & 0 & 1
\end{pmatrix}$

Из этого видно, что $x_1 = x_2, x_1 = -x_3$

$X_2 = C_2 \begin{pmatrix}
    1 \\
    1 \\
    -1
\end{pmatrix}$

Найдем третий собственный вектор.

Пусть $\lambda = \lambda_3 = 5$, тогда $(A - \lambda E) * X = 0$

\begin{equation}
    \begin{cases}
        (4 - 5)x_1 - 2x_2 - x_3 = 0 \\
        -x_1 + (3 - 5)x_2 - x_3 = 0 \\
        x_1 - 2x_2 + (2 - 5)x_3 = 0
    \end{cases}
\end{equation}

Запишем данную систему уравнений в матричном виде: $\begin{pmatrix}
    -1 & -2 & -1 \\
    -1 & -2 & -1 \\
    1 & -2 & -3
\end{pmatrix} = \begin{pmatrix}
    -1 & -2 & -1 \\
    1 & -2 & -3
\end{pmatrix} = \begin{pmatrix}
    -1 & -2 & -1 \\
    0 & -4 & -4
\end{pmatrix}$

Чему соответствует следующая система уравнений:

\begin{equation}
    \begin{cases}
        -x_1 - 2x_2 - x_3 = 0 \\
        -4x_2 - 4x_3 = 0
    \end{cases}
\end{equation}

Я зашел в какую-то фигню, где-то ошибся, но, в общем, ответ должен получиться следующий: $X_3 = C_3 \begin{pmatrix}
    1 \\
    -1 \\
    1
\end{pmatrix}$

\subsection{Векторная алгебра. Операции над векторами}

Вектором $\overrightarrow{AB}$ называется направленный отрезок $AB$, заданный своим началом $A$ и концом $B$.

Длиной (модулем) $|\overrightarrow{AB}|$ вектора $\overrightarrow{AB}$ называется длина отрезка $AB$.

Два вектора называются коллинеарными, если они параллельны одной прямой (параллельны друг другу).

Три вектора называются компланарными, если они параллельны одной плоскости.

Координаты $x, y, z$ вектора $\overrightarrow{a}$ это коэффициенты разложения вектора по базису, то есть по трем некомпланарным векторам, обозначаемым как $\overrightarrow{e_1}, \overrightarrow{e_2}, \overrightarrow{e_3}$.

$\overrightarrow{e_1} = \{1, 0, 0\}, \overrightarrow{e_2} = \{0, 1, 0\}, \overrightarrow{e_2} = \{0, 0, 1\}, \overrightarrow{a} = x * \overrightarrow{e_1} + y * \overrightarrow{e_2} + z * \overrightarrow{e_3}$

Если $\overrightarrow{e_1}, \overrightarrow{e_2}, \overrightarrow{e_3}$ взаимно перпендикулярны и единичные векторы: $\overrightarrow{i}, \overrightarrow{j}, \overrightarrow{k}$, то такой базис называется ортонормированным.

\subsubsection{Пример}

Разложить вектор $\overrightarrow{a} = \{4; 2; 0\}$, если возможно, по векторам $\overrightarrow{p} = \{1; -1; 2\}, \overrightarrow{q} = \{2; 2; -1\}, \overrightarrow{r} = \{3; 7; -7\}$ 

Для того, чтобы это было возможно, должно соблюдаться следующее выражение: $(\overrightarrow{p} * \overrightarrow{q}) * \overrightarrow{r} \ne 0$ - достаточное условие некомпланарности.

$(\overrightarrow{p} * \overrightarrow{q}) * \overrightarrow{r} = \begin{vmatrix}
    1 & -1 & 2 \\
    2 & 2 & -1 \\
    3 & 7 & -7
\end{vmatrix} \ne 0$

$\det X = -14 + 28 + 3 - 12 + 7 - 14 = 2 \ne 0$, следовательно, мы можем разложить данный вектор по трем некомпланарным векторам.

$\overrightarrow{a} = x * \overrightarrow{p} + y * \overrightarrow{q} + z * \overrightarrow{r}$

$\overrightarrow{p} = 1 * \overrightarrow{i} - 1 * \overrightarrow{j} + 2 * \overrightarrow{k}, \overrightarrow{q} = 2 * \overrightarrow{i} + 2 * \overrightarrow{j} - \overrightarrow{k}, \overrightarrow{r} = 3 * \overrightarrow{i} + 7 * \overrightarrow{j} - 7 * \overrightarrow{k}$

$x * \overrightarrow{p} + y * \overrightarrow{q} + z * \overrightarrow{r} = x * \overrightarrow{i} - x * \overrightarrow{j} + 2x\overrightarrow{k} + 2y\overrightarrow{i} + 2y\overrightarrow{i} + 2y\overrightarrow{j} - y\overrightarrow{k} + 3z\overrightarrow{i} + 7z\overrightarrow{j} - 7z\overrightarrow{k}$

Далее для разложения по базису нам необходимо вынести $\overrightarrow{i}, \overrightarrow{j}, \overrightarrow{k}$

$... = (x + 2y + 3z) + \overrightarrow{j}(-x + 2y + 7z) + \overrightarrow{k}(2x - y - 7z)$

$\overrightarrow{a} = x\overrightarrow{p} + y\overrightarrow{q} + z\overrightarrow{r}$

\begin{equation}
    \begin{cases}
        x + 2y + 3z = 4 \\
        -x + 2y + 7z = 2 \\
        2x - y - 7z = 0
    \end{cases}
\end{equation}

Решим данную систему уравнений каким угодно способом, сначала составив расширенную матрицу системы:

$
\begin{pmatrix}
    1 & 2 & 3 & 4 \\
    -1 & 2 & 7 & 2 \\
    2 & -1 & -7 & 0
\end{pmatrix} = \begin{pmatrix}
    1 & 2 & 3 & 4 \\
    1 & -2 & -7 & -2 \\
    1 & 1 & 0 & 2
\end{pmatrix} = \begin{pmatrix}
    1 & 2 & 3 & 4 \\
    1 & 1 & 0 & 2 \\
    0 & -3 & -7 & -4
\end{pmatrix} = \begin{pmatrix}
    1 & -1 & -4 & 0 \\
    1 & 1 & 0 & 2 \\
    0 & -3 & -7 & -4
\end{pmatrix}
$

\begin{equation}
\begin{cases}
    x_1 = -x_2 + 2 \\
    x_3 = \frac{4}{7} - \frac{3}{7}x_2 \\
    x_1 = -x_2-4x_3 = 0
\end{cases}
\end{equation}

\begin{equation}
    \begin{cases}
        x_2 = -1 \\
        x_1 = 3 \\
        x_3 = 1
    \end{cases}
\end{equation}

Тогда $\overrightarrow{a} = 3\overrightarrow{p} - \overrightarrow{q} + \overrightarrow{r}$

\pagebreak
\section{Алгебра и геометрия - 14.10.2022}

\subsection{Центр масс}

\begin{flushleft}

Если точки $A$ и $B$ заданы координатами $A(x_1; y_1; z_1), B(x_2; y_2; z_2)$, то координаты вектора $\overrightarrow{AB}: \{x_2 - x_1; y_2 - y_1; z_2 - z_1\} $.

\hfill

Разделить отрезок в соотношении $\lambda \ne -1$ это значит на прямой $AB$ найти такую точку $M$, что вектор $\overrightarrow{AM} = \lambda \overrightarrow{MB}$.

\hfill 

Если заданы координаты точек $A(x_1;y_1;z_1), B(x_2;y_2;z_2)$, то координаты делящей точки $M(x_m;y_m;z_m)$ находят по формулам: $x_m = \frac{x_1 = \lambda x_2}{1 + \lambda}, y_m = \frac{y_1 + \lambda y_2}{1 + \lambda}, z_m = \frac{z_1 + \lambda z_2}{1 + \lambda}$

\hfill

Если $M$ - середина $AB$, то $\lambda = 1$, а формулы $x_m = \frac{x_1 + x_2}{2}, y_m = \frac{y_1 + y_2}{2}, z_m = \frac{z_1 + z_2}{2}$

\end{flushleft}

\subsubsection{Пример}

\begin{flushleft}

Дано: $A_1(1; 3), m_1 = 10; A_2(7; 8), m_2 = 30; A_3(0; 4), m_3 = 5$. Определить $S$ - центр масс системы.

Пусть $C_1$ делит $A_1A_2$ в соотношении $\lambda = \frac{m_2}{m_1} = 3$, тогда $x_c = \frac{1 + 3 * 7}{4} = \frac{22}{4}, y_c = \frac{27}{4}$

Пусть $S$ делит $CA_3$ в соотношении $\lambda = \frac{m_3}{m_1 + m_2} = \frac{1}{8}$, тогда $x_s = \frac{11}{2} * \frac{8}{9} = \frac{44}{9}, y_s = \frac{\frac{27}{4} + \frac{11}{8} * 4 + \frac{1}{8}}{\frac{9}{8}} = \frac{29}{4} * \frac{8}{9} = \frac{58}{9}$.

Ответ: $S(\frac{44}{9}; \frac{58}{9})$

\end{flushleft}

\subsubsection{Некоторые нюансы}

\begin{flushleft}

1) Можно доказать, что центр масс $S(x_s; y_s; z_s)$ материальной системы точек $A_1(x_1; y_1; z_1), A_2(x_2; y_2; z_2), ..., A_n(x_n; y_n; z_n)$, в которых сосредоточены массы $m_1, m_2, ..., m_n$ имеет следующие координаты:

$x_s = \frac{x_1 * m_1 + x_2 * m_2 + ... + x_n * m_n}{m_1 + m_2 + ... + m_n}, y_s = \frac{y_1 * m_1 + ... + y_n * m_n}{m_1 + ... + m_n}, z_s = \frac{z_1 * m_1 + ... + z_n * m_n}{m_1 + ... + m_n}$

\hfill

2) Центры масс треугольника с координатами $A_1(x_1; y_1; z_1), A_2(x_2; y_2; z_2), A_3(x_3; y_3; z_3)$ (то есть, центр масс однородной треугольной пластины) находится в точке пересечения медиан.

Если предпложить, что $n = 3, m_1 = m_2 = m_3$, то $S(\frac{x_1 + x_2 + x_3}{3}; \frac{y_1 + y_2 + y_3}{3}; \frac{z_1 + z_2 + z_3}{3})$

\end{flushleft}

\subsection{Направляющие косинусы}

\begin{flushleft}

Пусть $\alpha, \beta, \gamma$ - углы, которые образуют $\vec{a} = \{x, z, z\}$ с осями $O_x, O_y, O_z$.

Тогда направляющие косинусы $\cos \alpha, \cos \beta, \cos \gamma$ вектора $\vec{a}$ связаны соотношением $\cos^2 \alpha + \cos^2 \beta + \cos^2 \gamma = 1$ и определяются формулами:

$\cos \alpha = \frac{x}{|\vec{a}|} = \frac{x}{\sqrt{x_2 + y_2 + z_2}}, \cos \beta = \frac{y}{|\vec{a}|}, \cos \gamma = \frac{z}{|\vec{a}|}$

\end{flushleft}

\subsubsection{Пример}

\begin{flushleft}

Найти длину и направляющие косинусы $\vec{AM}$, если т. $M$ делит $AB$ в соотношении $\lambda = -2$, где $A(5; 6; -1), B(0; -3; 2)$.

Найдем координаты точки $M$: $x_m = -5, y_m = -12, z_m = 5$. Таким образом, $M(-5; -12; 5)$.

$\vec{AM} = \{-10; -18; -6\}, |\vec{AM}| = \sqrt{100 + 324 + 36} = \sqrt{460} = 2\sqrt{115}$

Найдем направляющие косинусы: $\cos \alpha = \frac{-10}{2\sqrt{115}} \approx -0.466, \cos \beta = \frac{-18}{2\sqrt{115}} \approx -0.839, \cos \gamma = \frac{-6}{2\sqrt{115}} \approx 0.28$.

Выполним проверку: $\cos^2 \alpha + \cos^2 \beta + \cos^2 \gamma = \frac{100}{460} + \frac{324}{460} + \frac{36}{460} = 1$

\end{flushleft}

\subsection{Решение практической работы, вариант 21}

\subsubsection{Задание 5, нахождение центра тяжести системы}

\begin{flushleft}

Дано: $A_1(5; -4), A_2(0; 2), A_3(6; 6), m_1 = 25, m_2 = 45, m_3 = 15$.

Согласно формуле,
$S_x = \frac{5 * 25 + 0 * 45 + 6 * 15}{25 + 45 + 15} = \frac{215}{85} = \frac{43}{17}$,
$S_y = \frac{-4 * 25 + 2 * 45 + 6 * 15}{25 + 45 + 15} = \frac{80}{85} = \frac{16}{17}$.

Ответ: $S(\frac{43}{17}; \frac{16}{17})$

\end{flushleft}

\subsubsection{Задание 6, нахождение длины и направляющих косинусов}

\begin{flushleft}

Дано: $A(-2; -5), B(4; 1), \lambda = \frac{2}{7}$.

Найдем координаты точки $M$: $M_x = \frac{-2 + 4}{1 + \frac{2}{7}} = \frac{2}{\frac{9}{7}} = \frac{14}{9}, M_y = \frac{-5 + 1}{1 + \frac{2}{7}} = \frac{-4}{\frac{9}{7}} = \frac{-4 * 7}{9} = -\frac{28}{9}$, таким образом $M(\frac{14}{9}; -\frac{28}{9})$

$\vec{AM} = \{\frac{14}{9} + 2; -\frac{28}{9} + 5\} = \{\frac{32}{9}; \frac{17}{9}\}, |\vec{AM}| = \sqrt{\frac{1024}{81} + \frac{289}{81}} = \sqrt{\frac{1313}{81}}$ 

Найдем направляющие косинусы: $\cos \alpha = \frac{\frac{32}{9}}{\sqrt{\frac{1313}{81}}} \approx 0.883, \cos \beta = \frac{\frac{17}{9}}{\sqrt{\frac{1313}{81}}} \approx 0.469$.

Ответ: сами выпишите из того, что написано выше.

\end{flushleft}

\pagebreak
\section{Алгебра и геометрия - 15.10.2022}

\subsection{Ранг матрицы}

\begin{flushleft}

Рангом матрицы называется порядок наибольшего минора, отличного от нуля, который можно из этой матрицы получить.

\hfill

$A = \begin{pmatrix}
    1 & 2 & 3 & 4 \\
    5 & 0 & 1 & 1 \\
    0 & 1 & 2 & -1 \\
    1 & 2 & -1 & 5
\end{pmatrix}$ - миноров первого порядка полно, второго - тоже, третьего - тоже имеется, четвертого - лишь один.

Если определитель четвертого порядка не равен нулю, то $r(A) = 4$, но это нужно считать.

\subsubsection{Теорема об окаймляющих минорах}

Если матрица $A$ имеет ненулевой минор $\Delta \ne 0$ к-ого порядка, а все миноры, содержащие $\Delta$ $k + 1$-го порядка равны нулю, то ранг матрицы $A$ равен $k$.

\subsubsection{Другой способ подсчета ранга}

Ранг матрицы равен количеству не полностью нулевых строк, если данная матрица приведена к ступенчатому виду.

\hfill

Например, ранг матрицы $\begin{pmatrix}
    1 & 2 & 3 & 4 & 5 \\
    0 & 1 & 2 & -2 & 1 \\
    0 & 0 & 3 & 1 & 7
\end{pmatrix}$ равен трем.

\hfill

Если матрица не приведена к ступенчатому виду - ее надо к ней привести.

\subsection{Теорема Кронекера - Капелли}

Система линейных уравнений имеет решение (является совместной), если ранг расширенной матрицы совпадает с рангом матрицы системы.

Если ранг расширенной матрицы не совпадает с рангом матрицы системы, то решений нет.

Если $r(A) = r(A*) = n$ - то будет единственное решение.

Если $r(A) = r(A*) < n$ - решений бесконечно много - система неопределена, $r$ неизвестных назовем \textbf{базисными}, а $n - r$ неизвестных назовем \textbf{свободными} (через них все будем выражать).

\hfill

В случае однородной системы всегда имеется хотя бы нулевое решение.

\subsubsection{Фундаментальная система решений}

$r(A*) = r(A) = r < n$

\begin{equation}
    \begin{cases}
        a_{11} x_1 + a_{12} x_2 + ... + a_{1r} x_r = b_1 - a_{1r + 1} x_{r + 1} - a_{1r + 2} x_{r + 2} - ... \\
        a_{21} x_1 + a_{22} x_2 + ... + a_{2r} x_r = b_2 - a_{2r + 1} x_{r + 1} - a_{2r + 2} x_{r + 2} - ... \\
        ... \\
        a_{r1} x_1 + a_{r2} x_2 + ... + a_{r r} x_r = b_r - a_{r r + 1} x_{r + 1} - ...
    \end{cases}
\end{equation}

\hfill

$x_1(c_1 c_2 ... c_{n - r}), x_2(c_1 c_2 ... c_{n - r}), c_r (c_1 c_2 ... c_{n - r})$

\subsubsection{Примеры}

\textbf{Пример 1.}

\begin{equation}
    \begin{cases}
        x + y + z = 1 \\
        x + y + 2z = 1 \\
        2x + 2y + 4z = 3
    \end{cases}
\end{equation}

Составим матрицу расширенную системы:

$\begin{pmatrix}
    1 & 1 & 1 & 1 \\
    1 & 1 & 2 & 1 \\
    2 & 2 & 4 & 3
\end{pmatrix} \sim \begin{pmatrix}
    1 & 1 & 1 & 1 \\
    0 & 0 & 1 & 0 \\
    0 & 0 & 2 & 1
\end{pmatrix} \sim \begin{pmatrix}
    1 & 1 & 1 & 1 \\
    0 & 0 & 1 & 0 \\
    0 & 0 & 0 & 1
\end{pmatrix}, r(A) \ne r(A*)$

\hfill 

\textbf{Система несовместна, решений нет}

\hfill

\textbf{Пример 2.}

\begin{equation}
    \begin{cases}
        x + y + z = 1 \\
        x + y + 2z = 1 \\
        2x + 2y + 4z = 2
    \end{cases}
\end{equation}

Составим расширенную матрицу системы:

$\begin{pmatrix}
    1 & 1 & 1 & 1 \\
    1 & 1 & 2 & 1 \\
    2 & 2 & 4 & 2
\end{pmatrix} \sim \begin{pmatrix}
    1 & 1 & 1 & 1 \\
    0 & 0 & 1 & 0 \\
    0 & 0 & 2 & 0
\end{pmatrix} \sim \begin{pmatrix}
    1 & 1 & 1 & 1 \\
    0 & 0 & 1 & 0 \\
    0 & 0 & 0 & 0
\end{pmatrix}, r(A) = r(A*) < n$ - \textbf{бесконечное множество решений}.

Решение: $\begin{pmatrix}
    1 - a \\
    a \\
    0
\end{pmatrix}$

\hfill

\textbf{Пример 3.}

\begin{equation}
    \begin{cases}
        x + y + z = 1 \\
        2x + 2y + 2z = 2 \\
        -4x - 4y - 4z = -4
    \end{cases}
\end{equation}

Составим расширенную матрицу системы:

$\begin{pmatrix}
    1 & 1 & 1 & 1 \\
    2 & 2 & 2 & 2 \\
    -4 & -4 & -4 & -4
\end{pmatrix} \sim \begin{pmatrix}
    1 & 1 & 1 & 1 \\
    0 & 0 & 0 & 0 \\
    0 & 0 & 0 & 0
\end{pmatrix}, r(A) = r(A*) = 1$

\hfill

Имеем решение: $\begin{pmatrix}
    1 - a - b \\
    a \\
    b
\end{pmatrix}$

\textbf{Пример 4.}

\begin{equation}
    \begin{cases}
        x_1 - 5x_2 + 2x_3 - 16x_4 + 3x_5 = 0 \\
        x_1 + 11x_2 - 12x_3 + 34x_4 - 5x_5 = 0 \\
        2x_1 - 2x_3 - 3x_3 - 7x_4 + 2x_5 = 0 \\
        3x_1 + x_2 - 8x_3 + 2x_4 + x_5 = 0
    \end{cases}
\end{equation}

Запишем в виде матрицы:

$\begin{pmatrix}
    1 & -5 & 2 & -16 & 3 \\
    1 & 11 & -12 & 34 & -5 \\
    2 & -2 & -3 & -7 & 2 \\
    3 & 1 & -8 & 2 & 1
\end{pmatrix} \sim \begin{pmatrix}
    1 & -5 & 2 & -16 & 3 \\
    0 & 16 & -14 & 50 & -8 \\
    0 & 8 & -7 & 25 & -4 \\
    0 & 16 & -14 & 50 & -8
\end{pmatrix} \sim \begin{pmatrix}
    1 & -5 & 2 & -16 & 3 \\
    0 & 16 & -14 & 50 & -8
\end{pmatrix}$ 

\hfill

\begin{equation}
    \begin{cases}
        x_1 - 5x_2 = -3x_5 + 16x_4 - 2x_3 \\ 
        8x_2 = 4x_5 - 25x_4 + 7x_3
    \end{cases}
\end{equation}

Ступенчатая матрица содержит две ненулевые строки, значит количество базисных переменных равно двум, а количество свободных - трем.

\hfill

$x_2 = \frac{x_5}{2} - \frac{25x_4}{8} + \frac{7x_3}{8}, x_1 = 5x_2 - 3x_5 + 16x_4 - 2x_3 = \frac{5}{2}x_5 - \frac{125}{8}x_4 + \frac{35}{8}x_3 - 3x_5 + 16x_4 - 2x_3 = \frac{1}{2}x_5 + \frac{3}{8}x_4 + \frac{19}{8}x_3$

\hfill

Итоговый ответ: $\begin{pmatrix}
    -\frac{1}{2}x_5 + \frac{3}{8}x_4 + \frac{19}{8}x_3 \\
    \frac{1}{2}x_5 - \frac{25}{8}x_4 + \frac{7}{8}x_3 \\
    x_3 \\
    x_4 \\
    x_5
\end{pmatrix} \sim C_1 \begin{pmatrix}
    -\frac{1}{2} \\
    \frac{1}{2} \\
    0 \\
    0 \\
    1
\end{pmatrix} + C_2 \begin{pmatrix}
    \frac{3}{8} \\
    -\frac{25}{8} \\
    0 \\
    1 \\
    0
\end{pmatrix} + C_3 \begin{pmatrix}
    \frac{19}{8} \\
    \frac{7}{8} \\
    1 \\
    0 \\
    0
\end{pmatrix}$

\hfill

$X_{\text{общ}} = C_1 E_1 + C_2E_2 + C_3E_3$

\hfill

\textbf{Пример 5.}

Имеем следующую расширенную матрицу системы:

$\begin{pmatrix}
    2 & 1 & -1 & 3 & 2 \\
    -4 & 0 & 1 & -7 & 3 \\
    0 & 2 & -3 & 1 & 1 \\
    2 & 3 & -4 & -2 & 3
\end{pmatrix} \sim \begin{pmatrix}
    2 & 1 & -1 & 3 & 2 \\
    0 & -2 & 3 & -1 & -1 \\
    0 & 2 & -3 & 1 & 1 \\
    0 & 2 & -3 & 1 & 1
\end{pmatrix} \sim \begin{pmatrix}
    2 & 1 & -1 & 3 & 2 \\
    0 & -2 & 3 & -1 & -1 \\
\end{pmatrix}$

\begin{equation}
    \begin{cases}
        2x_1 + x_2 - x_3 - 3x_4 = 2 \\
        -2x_2 + 3x_3 - x_4 = -1
    \end{cases}
\end{equation}

Имеем:

\begin{equation}
    \begin{cases}
        2x_1 + x_2 = 2 + x_3 + 3x_4 \\
        -2x_2 = -1 - 3x_3 + x4
    \end{cases}
\end{equation}

\hfill

$x_2 = \frac{1}{2} + \frac{3}{2}x_3 - \frac{1}{2}x_4, 2x_1 = 2 + x_3 + + 3x_4 - x_2 = 2 + x_3 + 3x_4 - \frac{1}{2} - \frac{3}{2}x_3 + \frac{1}{2}x_4 = \frac{3}{2} - \frac{1}{2}x_3 + \frac{7}{2}x_4$

\hfill

$\begin{pmatrix}
    \frac{3}{4} - \frac{1}{4}x_3 + \frac{7}{4}x_4 \\
    \frac{1}{2} + \frac{3}{2}x_3 - \frac{1}{2}x_4 \\
    x_3 \\
    x_4
\end{pmatrix} = \begin{pmatrix}
    \frac{3}{4} \\
    \frac{1}{2} \\
    0 \\
    0
\end{pmatrix} + C_1 \begin{pmatrix}
    -x_1 \\
    \frac{3}{2} \\
    1 \\
    0
\end{pmatrix} + C_2 \begin{pmatrix}
    \frac{7}{4} \\
    -\frac{1}{2} \\
    0 \\
    1
\end{pmatrix}$

\hfill

$X = X_{\text{ч.р}} + X_{\text{одн}}, X_{\text{одн}} = C_1E_1 + C_2E_2$, $X_{\text{ч.р}}$ - наш столбик из циферок.

\end{flushleft}

\pagebreak
\section{Алгебра и геометрия - 21.10.2022}

\subsection{Скалярное произведение}

\begin{flushleft}

$\vec{a} * \vec{b} = |\vec{a}| * |\vec{b}| * \cos(\vec{a};\vec{b})$

Если или $\vec{a} = \vec{0}$ или $\vec{b} = \vec{0}$, то скалярное произведение будет равно нулю.

Два ненулевых вектора перпендикулярны тогда и только тогда, когда их скалярное произведение равно нулю.

$\vec{a} \perp \vec{b} \Longleftrightarrow \vec{a} * \vec{b} = 0 (\vec{a} \ne 0, \vec{b} \ne 0)$

\subsubsection{Примеры}

\textbf{Пример 1.}

\hfill

Найти $\cos \angle N M P$, если $M(1; 2; -4), N(4; 2; 0), P(-3; 2; -1)$
$\vec{MN} = \{3; 0; 4\}, \vec{MP} = \{-4; 0; 3\}$

$\cos \angle N M P = \frac{\vec{MN} * \vec{MP}}{|\vec{MN}| * |\vec{MP}|} = 0, \cos \angle N M P = 90^{\circ}$

\subsection{Скалярная проекция}

Скалярная проекция: $\text{ПР}_{\vec{b}} \vec{a} = \frac{\vec{a} * \vec{b}}{|\vec{b}|}$

Векторая проекция: $\vec{\text{ПР}}_{\vec{b}} \vec{a} = \text{ПР}_{\vec{b}} \vec{a} * \frac{\vec{b}}{|\vec{b}|}$

\subsubsection{Примеры}

\textbf{Пример 1.}

\hfill

$\text{ПР}_{\vec{b}} \vec{a} - ?, \vec{\text{ПР}}_{\vec{b}} \vec{a} - ?$

\hfill

$\vec{a} = 2\vec{A B} - \vec{C D}, \vec{b} = \vec{O C} \times \vec{A D}, A(1; 0; -1), B(1; -1; -2), C(4; 1; 0), D(0; 4; 3), O(0; 0; 0)$

\hfill

$\vec{AB} = \{0; -1; -1 \}, 2\vec{AB} = \{0; -2; -2 \}, \vec{CD} = \{-4; 3; 3 \}, \vec{O C} = \{4; 1; 0 \}, \vec{AD} = \{-1; 4; 4 \}$

\hfill

$\vec{a} = {4; -5; -5}, \vec{b} = \vec{O C} \times \vec{AD} = \begin{vmatrix}
    \vec{i} & \vec{j} & \vec{k} \\
    4 & 1 & 0 \\
    -1 & 4 & 4
\end{vmatrix} = 4\vec{i} - 16\vec{j} + 17\vec{k}$

\hfill

$\text{ПР}_{\vec{b}} \vec{a} = \frac{\vec{a} * \vec{b}}{|\vec{b}|} = \frac{4 * 4 + (-5) * (-16) + (-5) * 17}{\sqrt{4^2 + (-16)^2 + 17^2}} = \frac{11}{\sqrt{561}}$

$\vec{\text{ПР}}_{\vec{b}} \vec{a} = \text{ПР}_{\vec{b}} \vec{a} * \frac{\vec{b}}{|\vec{b}|} = \frac{11}{\sqrt{561}} * \frac{\vec{b}}{\sqrt{561}} = \frac{11}{561} \{ 4; -16; 16 \} = \{ \frac{4}{51}; -\frac{16}{51}; \frac{1}{3} \}$

\subsection{Векторное произведение}

Вектороное прозведение $\vec{a} \times \vec{b} = c$

$\vec{c}$ должен соответствовать следующим требованиям:

\begin{enumerate}
    \item $|\vec{c}| = |\vec{a} \times \vec{b}| = |\vec{a}| * |\vec{b}| * \sin (\vec{a} \vec{b})$ 
    \item $\vec{c} \perp \vec{a}, \vec{c} \perp \vec{b}$
    \item Тройка векторов $(\vec{a}, \vec{b}, \vec{c})$ правая
\end{enumerate}

\subsubsection{Основные задачи на векторное произведение}

\textbf{1) } Нахождение площади параллелограмма или треугольника, построенного на плоскости.

$S_{\text{пар}} = 2S_{\triangle} = |\vec{a} \times \vec{b}|$

\hfill 

\textbf{2) } Нахождение $\vec{N}$, перпендикулярного двум неколлинеарным векторам:

$\vec{a} || \vec{b}$, то $\vec{N} = \lambda (\vec{a} \times \vec{b}), \lambda \in R, \lambda \ne 0$

\subsubsection{Свойства векторного произведения}

\begin{enumerate}
    \item $\vec{a} \times \vec{b} = -\vec{b} \times \vec{a}$
    \item $\vec{a} \times \vec{b} = \vec{0} \Longleftrightarrow \lambda \vec{a} = \vec{b} \lor \vec{a} = \vec{0}, \vec{b} = \vec{0}$ 
    \item $\vec{a} \times (\vec{b} + \vec{c}) = \vec{a} \times \vec{b} + \vec{a} \times \vec{c}$
    \item $\lambda \vec{a} \times \vec{b} = \lambda (\vec{a} \times \vec{b}) = \vec{a} \times (\lambda \vec{b})$
\end{enumerate}

$\vec{a} = \{x_1; y_1; z_1\}, \vec{b} = \{x_2; y_2; z_2\}, \vec{a} \times \vec{b} = \begin{vmatrix}
    \vec{i} & \vec{j} & \vec{k} \\
    x_1 & y_1 & z_1 \\
    x_2 & y_2 & z_2
\end{vmatrix} = \vec{i} (y_1 z_1 - z_1 y_2) - \vec{j} (z_2 z_2 - z_1 x_2) + \vec{k} (x_1 y_2 - x_2 y_1)$

\subsubsection{Примеры}

\textbf{Пример 1.}

\hfill

$S_{\triangle} - ?, \vec{a} = 5\vec{m} - 8\vec{n}, \vec{b} = -\vec{m} + 2\vec{n}, |\vec{m}| = 1, |\vec{n}| = 2, \angle (\vec{m}; \vec{n}) = \frac{3}{4} \pi$

\hfill

$S_{\triangle} = \frac{1}{2} S_{\text{пар}} = \frac{1}{2} |\vec{a} \times \vec{b}|$

$\vec{a} \times \vec{b} = (5\vec{m} - 8\vec{n}) \times (-\vec{m} + 2\vec{n}) = 5\vec{m} \times (-\vec{m}) + 5\vec{m} \times 2\vec{n} + (-8\vec{n}) \times (-\vec{m}) + (-8\vec{n}) \times 2\vec{n} = 10\vec{m} \times \vec{n} + 8 \vec{n} \times \vec{m} = 10 \vec{m} \times \vec{n} - 8 \vec{n} \times \vec{n} = 2\vec{m} \times \vec{n}$

\hfill

$|\vec{a} \times \vec{b}| = |2\vec{m} \times \vec{n}| = 2 * |\vec{m}| * |\vec{n}| * \sin \angle (\vec{m}; \vec{n}) = 2 * 1 * 2 * \frac{\sqrt{2}}{2} = 2\sqrt{2}$

\hfill

$S_{\triangle} = \frac{1}{2} |\vec{a} \times \vec{b}| = \frac{1}{2} * 2 * \sqrt{2} = \sqrt{2}$

\hfill

\textbf{Пример 2.}

\hfill

$S_{\triangle A B C} - ?, h_{a} - ?, A(1; 3; 5), B(0; -1; -3), C(4; 3; -3)$

\hfill

$S_{\triangle A B C} = \frac{1}{2} |\vec{BA} \times \vec{BC}|$

$\vec{BA} = \{ 1; 4; 8 \}, \vec{BC} = \{ 4; 3; 0 \}$

$\vec{BA} \times \vec{BC} = \begin{vmatrix}
    \vec{i} & \vec{j} & \vec{k} \\
    1 & 4 & 8 \\
    4 & 3 & 0
\end{vmatrix} = -24\vec{i} + 32\vec{j} - 13\vec{k} = \{ -24; 32; -13 \}, |\vec{BA} \times \vec{BC}| = \sqrt{(-24)^2 + 32^2 + (-13)^2} = \sqrt{1769}$

\hfill

$S_{\triangle A B C} = \frac{1}{2} * \sqrt{1769} \approx 21.03$

$S_{\triangle ABC} = \frac{1}{2} * h * BC, |\vec{BC}| = 5, h = \frac{21 * 2}{5} \approx 8.4$

\hfill

\textbf{Пример 3.}

\hfill

$\vec{N} \perp M_1 M_2 M_3, M_1 (1; 3; 0), M_2 (-2; 1; -1), M_3 (0; 1; -1), \vec{N} - ?$

\hfill

$\vec{N} \perp \vec{M_1 M_2}, \vec{N} \perp \vec{M_1 M_3}$

\hfill

$\vec{N} = \lambda (\vec{M_1 M_2} \times \vec{M_1 M_3}), \vec{M_1 M_2} = \{ -3; -2; -1\}, \vec{M_1 M_3} = \{-1; -2; -1 \}, \vec{M_1 M_2} \text{ not parallel to } \vec{M_1 M_3}$

\hfill

$\vec{N} = \lambda \begin{vmatrix}
    \vec{i} & \vec{j} & \vec{k} \\
    -3 & -2 & -1 \\
    -1 & -2 & -1
\end{vmatrix} = \lambda (0\vec{i} - 2\vec{j} + 4\vec{k}) = \frac{1}{2} \{0; -2; 4 \} = \{ 0; -1; 2 \}$

\subsection{Смешанное произведение}

Смешанным произведением трех векторов $\vec{a}, \vec{b}, \vec{c}$ называют число: $(\vec{a} \times \vec{b}) * \vec{c} = \begin{vmatrix}
    x_a & y_a & z_a \\
    x_b & y_b & z_b \\
    x_c & y_c & z_c
\end{vmatrix}$

\hfill

$V_{\text{параллелепипеда}} = | (\vec{a} \times \vec{b}) * \vec{c} |, V_{\text{тр. пир.}} = \frac{1}{6} V_{\text{ПАРАЛ}} = \frac{1}{6} | (\vec{a} \times \vec{b}) * \vec{c} |$

\subsubsection{Примеры}

\textbf{Пример 1.}

\hfill

$V_{A B C D} - ?, AH - ?, A(2; -4; 5), B(-1; -3; 4), C(5; 5; -1), D(1; -2; 2)$

$\vec{BA} = \{ 3; -1; 1 \}, \vec{BC} = \{6; 8; -5\}, \vec{B D} = \{2; 1; -2\}$

\hfill

$(\vec{a} \times {b}) * \vec{c} = \begin{vmatrix}
    3 & -1 & 1 \\
    6 & 8 & -5 \\
    2 & 1 & -2
\end{vmatrix} = -48 + 6 + 10 - 16 + 15 - 12 = -45$

$V_{\text{ТР. ПИР}} = \frac{1}{3} S_{\text{осн}} * h = \frac{1}{6} | (\vec{BA} \times \vec{BC}) * \vec{B D} | = \frac{45}{6}$

$S_{\triangle} = \frac{1}{2} | \vec{a} \times \vec{b} | = \frac{1}{2} | \vec{BC} \times \vec{B D}| = \frac{1}{2} \begin{vmatrix}
    \vec{i} & \vec{j} & \vec{k} \\
    6 & 8 & -5 \\
    2 & 1 & -2
\end{vmatrix} = \frac{1}{2} * |\{ -11; 2; -16 \}| = \frac{1}{2} \sqrt{(-11)^2 + 2^2 + (-10)^2} = \frac{15}{2}$

$h = \frac{3 V_{\text{ТР. ПИР.}}}{S_{\text{осн.}}} = \frac{45}{15} = 3$

\end{flushleft}

\pagebreak
\section{Алгебра и геометрия - 24.10.2022}

\subsection{Прямая на плоскости}

\begin{flushleft}

Ненулевой вектор $\vec{S}$, параллельный прямой l, называется направляющим вектором прямой.

Ненулевой вектор $\vec{N}$, перпендикулярный прямой l, называется вектором нормали прямой l.

\subsubsection{Уравнения прямой на плоскости}

\begin{enumerate}
    \item $y = k x + b$, где $k = \tg \alpha$
    \item $y - y_0 = k (x - x_0)$, где $k = \tg \alpha$ - уравнение прямой, проходящей через точку $M(x_0; y_0)$ с заданным угловым коэффициентом $k$
    \item $A x + B y + C = 0, A^2 + B^2 \ne 0$ - общее уравнение прямой (вектор нормали прямой: $\vec{N} = \{ A; B \}$)
    \item $A (x - x_0) + B(y - y_0) = 0, A^2 + B^2 \ne 0$ - уравнение прямой, проходящей через точку $M(x_0; y_0)$ с заданным вектором нормали $\vec{N} = \{ A; B \}$
    \item $\frac{x - x_0}{m} = \frac{y - y_0}{n}, m^2 + n^2 \ne 0$ - каноническое уравнение прямой (направляющий вектор $\vec{S} = \{ m; n \}$, $M(x_0; y_0)$
    \item $\frac{x - x_1}{x_2 - x_1} = \frac{y - y_1}{y_2 - y_1}$ - уравнение прямой, проходящей через заданные точки $M_1(x_1; y_1)$ и $M_2(x_2; y_2)$
\end{enumerate}

\subsubsection{Угол между двумя прямыми}

$l_1: y = k_1 x + b$, $l_2: y = k_2 x + b_2$

$\tg \alpha = \pm \frac{k_2 - k_1}{1 + k_1 * k_2} \ge 0$

\begin{enumerate}
    \item $l_1 \perp l_2 \Longleftrightarrow k_2 = -\frac{1}{k_1}$
    \item $l_1 \parallel l_2 \Longleftrightarrow k_1 = k_2$
\end{enumerate}

\subsubsection{Примеры}

\textbf{Пример 1.} Найти координаты центра описанной около треугольника $ABC$, где $A(0, 3)$, $B(2; 5)$, $C(-2; 7)$.

Пусть точка $D$ - середина $AB$, ее координаты - $D(1; 4)$, точка $P$ - середина $BC$, ее координаты - $P(0; 6)$

\hfill

$\vec{N} = \vec{AB} = \{ 2; 2 \}, 2(x - 1) + 2(y - 4) = 0 \Longleftrightarrow 2x + 2y - 10 = 0$

$\vec{BC} = \{ -4; 2 \}, -4(x - 0) + 2 (y - 6) = 0 \Longleftrightarrow -4x + 2y - 12 = 0$

\hfill

\begin{equation}
    \begin{cases}
        2x + 2y - 10 = 0 \\
        -4x + 2y - 12 = 0
    \end{cases}
\end{equation}

\textbf{Ответ:} $S(-\frac{1}{3}; \frac{16}{3})$

\hfill

\textbf{Пример 2.} Даны две вершины $A_1(2; 4)$, $A_2(3; 1)$, $\triangle A_1 A_2 A_3$, $N(4; 0)$ - точка пересечения медиан.

Составить уравнение сторон этого треугольника и найти точку третьей вершины.

\hfill

$X_N = \frac{x_1 + x_2 + x_3}{3}, y_N = \frac{y_1 + y_2 + y_3}{3}$ - координаты точки пересечения медиан.

$x_3 = 3X_N - x_1 - x_2 = 12 - 2 - 3 = 7$, $y_3 = 3Y_N - y_1 - y_2 = -5$

$A_3(7; -5)$ - координаты третьей вершины

\hfill

$(A_1 A_2): \frac{x - x_1}{x_2 - x_1} = \frac{y - y_1}{y_2 - y_1} \Longleftrightarrow \frac{x - 2}{3 - 2} = \frac{y - 4}{1 - 4} \Longleftrightarrow -3x + y = y - 4 \Longleftrightarrow -3x - y + 10 = 0$

\hfill

$(A_2 A_3): 3x + 2y - 11 = 0$

\hfill 

$(A_1 A_3): 9x + 5y - 38 = 0$

\hfill

\textbf{Пример 3.} Даны вершины $A_1 (1; 0)$, $A_2(3; 5)$ треугольника $\triangle A_1 A_2 A_3$, $N(-1; 3)$ - точка пересечения высот данного треугольника.

Определить координаты $A_3$.

\hfill

$\vec{A_1 N} = \{ -2; 3 \}$, $-2 (x - 3) + 3 (y - 5) = 0 \longleftrightarrow -2x + 3x - 9 = 0$

$\vec{A_2 N} = \{ -4; -2 \}$, $A_2 N \perp (A_1 A_3)$

\hfill

Уравнение прямой $A_1 A_3$: $-4 (x - 1) - 2y \Longleftrightarrow -4x -2y + 4 = 0 \Longleftrightarrow 2x + y - 2 = 0$

\hfill

\begin{equation}
    \begin{cases}
        -2x + 3y - 9 = 0 \\
        2x + y - 2 = 0
    \end{cases}
\end{equation}

\textbf{Точка пересечения} - $A_3 (-\frac{3}{8}; \frac{11}{4})$ 

\hfill

$y = \frac{2x + 9}{3}, -2 = -\frac{1}{k}, k = \frac{1}{2}, \vec{n} = -\frac{1}{2} \vec{A_2 N} = \vec{A_3 N} = \{ 2; 1 \}$

\hfill

\textbf{Пример 4.} 

\end{flushleft}

\pagebreak
\section{Алгебра и геометрия - 29.10.2022}

\subsection{Линейные пространства}

\begin{flushleft}

\textbf{Линейным пространством} называется множество элементов произвольной природы, на котором определены операции \textbf{сложения} и \textbf{умножения на число}, согласованные друг с другом и \textbf{замкнутые в этом множестве}.

\hfill

\textbf{Замкнутость в множестве} означает то, что результаты выполнения операций над его элементами остаются элементами множества.

\subsubsection{Аксиомы линейного пространства}

\textbf{Сложением (обобщенным сложением)} называется операция, которая любым двум элементам данного множества ставит в соответствие элемент этого же множества, называемый их суммой: $x, y \in D \rightarrow z \in D, z = x + y$

Причем данная операция удовлетворяет следующим условиям:

\begin{enumerate}
    \item \textbf{ассоциативности:} $x \bigoplus (y \bigoplus z) = (x \bigoplus y) \bigoplus z$
    \item \textbf{коммутативности:} $x \bigoplus y = y \bigoplus x$
    \item \textbf{нулевого элемента:} $x \bigoplus \theta = x$
    \item \textbf{обратного элемента:} $x \bigoplus \overline{x} = \theta$
\end{enumerate}

Множества с операциями такого типа называются \textbf{абелевыми группами}.

\hfill

Умножением на число называется операция, которая любому элементу данного множества и любому действительному числу $\alpha$ ставит в соответствие элемент того же множества, называемый их произведением: $x \in D; \alpha \in R \rightarrow z \in D, z = \alpha \bigodot x$

Причем данная операция удовлетворяет следующим условиям:

\begin{enumerate}
    \item $\alpha \bigodot (\beta \bigodot x) = (\alpha \bigodot \beta) \bigodot x$
    \item $1 \bigodot x = x$
\end{enumerate}

\hfill

\textbf{Условия согласования операций сложения и умножения}:

\begin{enumerate}
    \item $(\alpha + \beta) \bigodot x = \alpha \bigodot x \bigoplus \beta \bigodot x$
    \item $\alpha \bigodot (x \bigoplus y) = \alpha \bigodot x + \alpha \bigodot y$
\end{enumerate}

\subsubsection{Примеры линейных пространств}

\textbf{Пример 1.} Множество действительных чисел является линейным пространством.

\hfill

\textbf{Пример 2.} Множество матриц также является линейным пространством.

\hfill

\textbf{Пример 3.} Рассмотрим множество ($A$) многочленов второго порядка (вида $ax^2 + bx + c$).

Оно не является линейным пространством: при сложении элементов этого множества мы можем получить элемент, не принадлежащий множеству. Например, $(2x^2 + 3x + 1) + (-2x^2 - 5x) = -2x + 1 \notin A$

\hfill

\textbf{Пример 4.} Множество векторов является линейным пространством.

\hfill

\textbf{Пример 5.} Множество векторов, выходящих из данной точки и заканчивающихся в конце прямой линии, на которой лежит данная точка.

Данное пространство не является линейным.

\subsubsection{Следствия из аксиом линейного пространства}

\begin{enumerate}
    \item В линейном пространстве существует единственный нулевой элемент
    \item В линейном пространстве у каждого элемента должен существовать обратный элемент
    \item Если выполняется $\alpha \bigodot x = 0$, то либо $\alpha$ равно нулю, либо $x$ является нулевым элементом
    \item Разностью элементов называют операцию, обратную сложению
\end{enumerate}

\subsubsection{Линейная комбинация элементов}

\textbf{Линейной комбинацией элементов называют} элемент $\alpha_1 \bigodot x_1 \bigoplus \alpha_2 \bigodot x_2 + ... + \alpha_n \bigodot x_n = \theta$ $(*)$, где $\alpha_i$ - действительные числа 

\hfill

Если равенство $(*)$ выполняется только при всех $a_i$ равных нулю, то все элементы $x_i$ являются \textbf{линейно независимыми}. Иначе эти элементы называются \textbf{линейно зависимыми}

\hfill

Для того, чтобы система векторов \textbf{была линейно зависимой}, необходимо и достаточно, чтобы хотя бы один вектор являлся линейной комбинацией остальных.

\hfill

\textbf{Доказательство необходимости. } Предполагаем, что наши системы векторов являются линейно зависимыми. Не нарушим общность, если предположим, что первый элемент отличен от нуля. Тогда мы можем записать:

$\alpha_1 x_1 = -\alpha_2 x_2 - \alpha_3 x_3 - ... - \alpha_n x_n \Longleftrightarrow x_1 = -\frac{\alpha_2}{\alpha_1} x_2 - \frac{\alpha_3}{\alpha_1} x_3 - ... - \frac{\alpha_n}{\alpha_1} x_n$

Что и требовалось доказать

\hfill

\textbf{Доказательство достаточности} тоже легко сочинить.

\subsubsection{Размерность линейного пространства}

Если существует натуральное число $n$ такое, что наше пространство содержит $n$ линейно независимых векторов, а прибавление любого лишнего вектора делает эти вектора линейно зависимыми, тогда мы говорим, что линейное пространство \textbf{имеет размерность $n$}

\subsubsection{Базис линейного пространства}

Упорядоченная система векторов $e_1, e_2, ..., e_n$ называется базисом линейного пространства, если

\begin{enumerate}
    \item Эти вектора являются линейно независимыми
    \item Любой вектор линейного пространства можно выразить как линейную комбинацию из этих векторов: $x = \xi_1 e_1 + \xi_2 e_2 + ... + \xi_n e_n$, где $\xi_i$ - координаты вектора $e$ в базисе $e_1,e_2,...,e_n$
\end{enumerate}

\textbf{Замечание 1. } Координаты в разложении по конкретному базису определяются однозначно.

\textbf{Замечание 2. } В линейном пространстве существует бесконечное множество базисов. Если линейное пространство имеет размерность $n$, то базис будет состоять из $n$ векторов.

\textbf{Замечание 3.} На плоскости в качестве базиса могут использоваться любых два неколлинеарных вектора

\hfill

\textbf{Пример 1.}
\hfill

Например, если мы работаем на плоскости, то имеем ортонормированный ($\vec{i}, \vec{j}$) базис. Дано $e_1 = 2 \vec{i} + \vec{j}, e_2 = -1\vec{i} + 2\vec{j}, p = 3\vec{i} + 5\vec{j}$.

Запишем вектор $p$ в новом базисе $e_1, e_2$: $\overline{p} = \xi_1 \overline{e_1} + \xi_2 \overline{e_2}$

\hfill

$\begin{pmatrix}
    3 \\
    5
\end{pmatrix} = \xi_1 \begin{pmatrix}
    2 \\
    1
\end{pmatrix} + \xi_2 \begin{pmatrix}
    -1 \\
    2
\end{pmatrix}$

\begin{equation}
    \begin{cases}
        3 = 2x - y \\
        5 = x + 2y
    \end{cases}
\end{equation}

Решая систему уравнений, получим: $x = 2.2, y = 1.4$

\hfill

\textbf{Ответ}: $\overline{p} = 2.2 \overline{e_1} + 1.4 \overline{e_2}$

\hfill

\textbf{Свойства базиса линейного пространства}

Пусть мы рассматриваем любое $n$-мерное линейное пространство, и $e_1, e_2, ..., e_n$ - базис в $n$-мерном линейном пространстве.

\begin{enumerate}
    \item $\alpha = \xi_1 e_1 + \xi_2 e_2 + ... + \xi_n e_n, b = \lambda_1 e_1 + \lambda_2 e_2 + ... + \lambda_n e_n$, то $\overline{a} + \overline{b} = (\xi_1 + \lambda_1) e_1 + (\xi_2 + \lambda_2) e_2 + ... + (\xi_n + \lambda_n) e_n$
    \item $\alpha \vec{a} = \alpha \xi_1 \overline{e_1} + ... + \alpha \xi_n \overline{e_n}$
\end{enumerate}

\subsection{Векторная алгебра}

\subsubsection{Скалярное произведение векторов}

Скалярное произведение векторов - \textbf{число}.

\hfill 

$a * b = |\vec{a}| * |\vec{b}| \cos \alpha$, где $\alpha$ - угол между данными векторами.

\hfill

Обладает следующими свойствами:

\begin{multienumerate}
    \mitemxx{$a * b = b * a$}{$(\alpha a) * b$}
    \mitemxx{$(a + b) * e = a c + b c$}{$a * a \ge 0$}
\end{multienumerate}

Допустим, имеем $\alpha = \{ x_a; y_a; z_a \}, b = \{ x_b; y_b; z_b \}$, то $ab = x_a x_b + y_a y_b + z_a z_b$

\hfill

$\cos \alpha = \frac{a b}{|a| |b|} = \frac{x_a x_b + y_a y_b + z_a z_b}{\sqrt{x_a^2 + y_a^2 + z_a^2} \sqrt{x_b^2 + y_b^2 + z_b^2}}$

\hfill

Необходимым и достаточным \textbf{условием перпендикулярности векторов} $a$ и $b$ является равенство нулю их скалярного произведения, $a * b > 0$ - угол острый, $a * b < 0$ - угол тупой

\subsubsection{Скалярная проекция вектора}

$\proj_{b} \overline{a} = X_{\cos \alpha} + Y_{\cos \beta} + Z_{\cos \gamma}, \proj_{x} \vec{a} = a * i, \proj_{y} a = a * j$, где $\alpha, \beta, \gamma$ - углы, которые в сост. с коор. осями.

\hfill

$e = \{ \cos \alpha, \cos \beta, \cos \gamma \}$ - вектор в направлении $b$

\hfill 

$\proj_{b} a = |a| \cos \alpha = | a | \frac{a b}{|a| |b|} = \frac{a * b}{|b|}$

\end{flushleft}


\end{document}
