\documentclass{article}
\usepackage[utf8]{inputenc}

\usepackage[T2A]{fontenc}
\usepackage[utf8]{inputenc}
\usepackage[russian]{babel}

\usepackage{amsmath}
\def\vec{\ensuremath\overrightarrow}

\title{Алгебра и геометрия}
\author{Лисид Лаконский}
\date{October 2022}

\begin{document}

\maketitle

\tableofcontents
\pagebreak


\section{Алгебра и геометрия - 04.10.2022}

\subsubsection{Ранг матрицы}

Пусть дана матрица $A$ размера $m * n$.

Возьмем любые $k$ ($k \le min(n;m)$) строк и $k$ столбцов матрицы $A$.

На их пересечении стоят элементы, образующие определитель $k$-того порядка, который и называется минором $k$-го порядка.

Под минором 1-го порядка матрицы $A$ понимается любой элемент.

Рангом $r$ матрицы $A$ называется наивысший порядок минора матрицы $A$, отличный от нуля.

Следовательно, если у нас матрица из четырех строк и трех столбцов, максимальный минор может быть три на три. Но если все они равны нулю, то мы не можем сказать, что ранг матрицы равен нулю.

Из определения следует:

\begin{enumerate}
    \item $r$ - целое число ($0 \le r \le min(m;n)$)
    \item Все миноры ($r + 1$) порядка либо нулевые, либо не существуют. 
\end{enumerate}

$A = \begin{pmatrix}
    2 & 1 & 3 & 7 \\
    0 & 4 & -1 & 0 \\
    0 & 0 & 8 & 1
\end{pmatrix}, r(A) = 3$

Миноры 1-го порядка: любой элемент матрицы. 

Миноры второго порядка: любой определитель этой матрицы 2x2: $\begin{pmatrix} 2 & 1 \\ 0 & 4 \end{pmatrix}, \begin{pmatrix} 1 & 3 \\ 4 & -1 \end{pmatrix}$

Миноры третьего порядка: $\begin{pmatrix} 2 & 1 & 3 \\ 0 & 4 & -1 \\ 0 & 0 & 8\end{pmatrix}, \det A = 64 \ne 0$

Минора четвертого порядка у данной матрицы не существует.

\subsection{Действия над матрицами}

\begin{enumerate}
    \item Умножение строки или столбца на число, отличное от нуля.
    \item Сложение: прибавление к одной строке (столбцу) другой, умноженной на число.
    \item Перемещение (замена местами) двух строк или двух столбцов.
    \item Вычеркивание нулевой строки или столбца.
\end{enumerate}

$
A = \begin{pmatrix}
 2 & 3 & 4 & 5 \\
 3 & 5 & 2 & 4 \\
 5 & 9 & -2 & 2 \\
\end{pmatrix} = \begin{pmatrix}
    1 & 2 & -2 & -1 \\
    3 & 5 & 2 & 4 \\
    5 & 9 & -2 & 2
\end{pmatrix} = \begin{pmatrix}
    1 & 2 & -2 & -1 \\
    0 & -1 & 8 & 7 \\
    0 & -1 & 8 & 7
\end{pmatrix} = \begin{pmatrix}
 1 & 2 & -2 & -1 \\
 0 & -1 & 8 & 7
\end{pmatrix}, \begin{vmatrix} 1 & 2 \\ 0 & -1 \end{vmatrix} = -1 \ne 0
$

\subsection{Теорема Кронекера-Капелли}

Рассмотрим систему $m$ линейных уравнений с $n$ неизвестными:

\begin{equation}
    \begin{cases}
        a_{11} * x_1 + a_{12} * x_2 + ... + a_{1n} * x_n = b_1 \\
        ... \\
        a_{m1} * x_1 + a_{m2} * x_2 + ... + a_{mn} * x_n = b_m
    \end{cases}
\end{equation}

$A = \begin{pmatrix}
    a_{11} & ... & a_{1n} \\ 
    a_{m1} & ... & a_{m n}
\end{pmatrix}, B = \begin{pmatrix}
    b_1 \\
    ... \\
    b_m
\end{pmatrix}, X = \begin{pmatrix}
    ?
\end{pmatrix}, (A|B) = \begin{pmatrix}
    a_{11} & ... & a_{1 n} & b_1 \\
    a_{m1} & ... & a_{m n} & b_m
\end{pmatrix}$

Система называется совместной, если она имеет решение. (*) Операции только над строками.

$$r(A) = r(A|B) \equiv r$$

Если $r = n$, то система имеет единственное решение. Если $r < n$, то система имеет бесконечное множество решений, зависящих от ($n - r$) свободных неизвестных.

\subsection{Метод Гаусса}

Если столбец $B = \begin{pmatrix} b_1 \\ ... \\ b_n \end{pmatrix}$ свободных членов - нулевой, то система называется однородной.

Однородная система всегда имеет решение, и она всегда совместна, так как имеет тривиальное (нулевое) решение: $x_1 = 0, x_2 = 0, ..., x_n = 0$.

Если в однородной системе число неизвестных $n$ равно числу уравнений $m$, то она имеет ненулевое решение тогда и только тогда, когда определитель системы равен нулю.

\begin{equation}
    \begin{cases}
        x_1 + 5x_2 + 4x_3 - x_4 = 2 \\
        2x_1 - x_2 - x_3 + 2x_4 = 3 \\
        3x_1 + 4x_2 + 3x_3 + x_4 = 5
    \end{cases}\,.
\end{equation}

$A = \begin{pmatrix}
    1 & 5 & 4 & -1 \\
    2 & -1 & -1 & 2 \\
    3 & 4 & 3 & 1
\end{pmatrix}, B = \begin{pmatrix}
    2 \\
    3 \\
    5
\end{pmatrix}, (A|B) = \begin{pmatrix}
    1 & 5 & 4 & -1 & 2 \\
    2 & -1 & -1 & 2 & 3 \\
    3 & 4 & 3 & 1 & 5
\end{pmatrix} = \begin{pmatrix}
    1 & 5 & 4 & -1 & 2 \\
    0 & -11 & -9 & 4 & -1 \\
    0 & -11 & -9 & 4 & -1
\end{pmatrix} = \begin{pmatrix}
    1 & 5 & 4 & -1 & 2 \\
    0 & -11 & -9 & 4 & -1
\end{pmatrix}, \begin{vmatrix} 1 & -5 \\ 0 & -11 \end{vmatrix} = -11 \ne 0, r(A|B) = 2 = r(A)$

$r < n \rightarrow$ система имеет бесконечное кол-во решений, зависящих от (4 - 2) = 2 свободных неизвестных.

Пусть $\begin{vmatrix}
    1 & 5 \\
    0 & -11
\end{vmatrix}$ - базисный минор, тогда $x_1$ и $x_2$ - базисные члены, а $x_3$ и $x_4$ - свободные.

$
\begin{pmatrix}
    1 & 5 & 4 & -1 & 2 \\
    0 & -11 & -9 & 4 & -1
\end{pmatrix}
$

Преобразуем данную матрицу, делая по главной диагонали базисного минора единицы, а по побочной нули.

$\begin{pmatrix}
    1 & 5 & 4 & -1 & 2 \\
    0 & 1 & \frac{9}{11} & -\frac{4}{11} & \frac{1}{11}
\end{pmatrix} = \begin{pmatrix}
    10 & -\frac{-1}{11} & \frac{9}{11} & \frac{17}{11} \\
    0 & 1 & \frac{9}{11} & -\frac{4}{11} & \frac{1}{11}
\end{pmatrix}$

Выпишем в виде системы уравнений:

\begin{equation}
    \begin{cases}
        1x_1 + 0x_2 - \frac{1}{11}x_3 + \frac{9}{11}x_4 = \frac{17}{11} \\
        0x_1 + 1_x2 + \frac{9}{11}x_3 - \frac{4}{11}x_4 = \frac{1}{11}
    \end{cases}
\end{equation}

\begin{equation}
    \begin{cases}
        x_1 = \frac{1}{11}x_3 - \frac{9}{11}x_4 + \frac{17}{11} \\
        x_2 = -\frac{9}{11}x_3 + \frac{4}{11}x_4 + \frac{1}{11}
    \end{cases}
\end{equation}

Пусть $x_3 = c_1$, а $x_4 = c_2$ ($c_1, c_2 \in R$), тогда наша система приобретает вид:

\begin{equation}
    \begin{cases}
        x_1 = \frac{1}{11}c_1 - \frac{9}{11}c_2 + \frac{17}{11} \\
        x_3 = -\frac{9}{11}c_1 + \frac{4}{11}c_2 + \frac{1}{11} \\
        x_3 = c_1 \\
        c_4 = c_2
    \end{cases}
\end{equation}

Исследуем на совместность систему

\begin{equation}
    \begin{cases}
        2x_1 + 2x_2 + x_3 = 6 \\
        x_1 + 2x_2 + 4x_3 = 4 \\
        3x_1 + 4x_2 + 5x_3 = 9
    \end{cases}
\end{equation}

$(A|B) = \begin{pmatrix}
    2 & 2 & 1 & 6 \\
    1 & 2 & 4 & 4 \\
    3 & 4 & 5 & 9
\end{pmatrix} = \begin{pmatrix}
    1 & 2 & 4 & 4 \\
    2 & 2 & 1 & 6 \\
    3 & 4 & 5 & 9
\end{pmatrix} = \begin{pmatrix}
    1 & 2 & 4 & 4 \\
    0 & -2 & -7 & -2 \\
    0 & -2 & -7 & -3
\end{pmatrix} = \begin{pmatrix}
    1 & 2 & 4 & 4 \\
    0 & -2 & -7 & -2 \\
    0 & 0 & 0 & 0 & -1
\end{pmatrix}, r(A|B) = 3, r(A) = 2$

Следовательно, система несовместна и решений не имеет.

\pagebreak
\section{Алгебра и геометрия - 07.10.2022}

\subsection{Собственные значения и собственные векторы матрицы}

Матрицы могут представляться на плоскости - для этого нужны собственные значения и собственные векторы.

Пусть дана квадратная матрица $A$ $n$-ого порядка. Ненулевой вектор $X = \begin{pmatrix}
    x_1 \\
    ... \\
    x_n
\end{pmatrix}$ называется собственным вектором матрицы $A$, если под действием этой матрицы он переходит в коллинеарный ему:

$$
A * X = \lambda X, \lambda \in R
$$

Где $\lambda$ - собственное значение соответствующего ему вектора матрицы $A$.

Для нахождения $\lambda$ составляют характеристическое уравнение:

$$
|A - \lambda E| = 0
$$

Если $\lambda_0$ - сосбтвенное значение матрицы $A$, то соответствующие собственные векторы находим из системы однородных линейных уравнений.

(*) Однородными называются системы, где матрица-столбец свободных членов $B$ полностью состоит из нулей

$$
(A - \lambda_0 E ) * X = 0
$$

\subsubsection{Примеры}

$$A = \begin{pmatrix}
    4 & -2 & -1 \\
    -1 & 3 & -1 \\
    1 & -2 & 2
\end{pmatrix}
$$

Составим характеристическое уравнение $|A - \lambda E| = 0$:


$$
\begin{vmatrix}
    4-\lambda & -2 & -1 \\
    -1 & 3-\lambda & -1 \\
    1 & -2 & 2 - \lambda
\end{vmatrix}
$$

$(4 - \lambda)(3 - \lambda)(2 - \lambda) - 2 + 2 + 3 - \lambda - 8 + 2 \lambda - 4 + 2 \lambda = (12 - 7 \lambda + \lambda^2)(2 - \lambda) + 3\lambda - 9 = 24 - 12 \lambda - 14 \lambda + 7 \lambda^2 + 2 \lambda^2 - \lambda^3 + 3\lambda - 9 = -\lambda^3 - 6\lambda^2 - 23\lambda + 15 = 0$

$\lambda_1 = 1$, вынесем общий множитель:

$
\frac{-\lambda^3 + 9\lambda^2 - 23\lambda + 15}{\lambda - 1} = (\lambda - 1)(-\lambda^2 + 8\lambda - 15)
$

Решаем через дискриминант или через теорему Виета: что угодно.

Итого имеем:

$\lambda_1 = 1, \lambda_2 = 3, \lambda_3 = 5$

Найдем теперь собственные векторы.

Пусть $\lambda = \lambda_1 = 1$, тогда $(A - \lambda E) * X = 0$:

\begin{equation}
    \begin{cases}
        (4-1)x_1 - 2x_2 - x_3 = 0 \\
        -x_1 + (3 - 1)x_2 - x_3 = 0 \\
        x_1 - 2x_2 + (2 - 1)x_3 = 0
    \end{cases}
\end{equation}

В матричном виде:

$\begin{pmatrix}
    3 & -2 & - 1 \\
    -1 & 2 & -1 \\
    1 & -2 & 1
\end{pmatrix} = \begin{pmatrix}
    3 & -2 & -1 \\
    1 & -2 & 1
\end{pmatrix} = \begin{pmatrix}
    1 & -2 & 1 \\
    0 & 4 & -4
\end{pmatrix} = \begin{pmatrix}
    1 & -2 & 1 \\ 
    0 & 1 & -1
\end{pmatrix} = \begin{pmatrix}
    1 & 0 & -1 \\
    0 & 1 & -1
\end{pmatrix}$

Преобразуем обратно в систему:

\begin{equation}
    \begin{cases}
        x_1 - x_3 = 0 \\
        x_2 - x_3 = 0 
    \end{cases}
\end{equation}

Отсюда видим, что $x_1 = x_3, x_2 = x_3$

Пусть $x_1 = 1$, тогда $x_1$ = $\begin{pmatrix}
    1 \\
    1 \\
    1
\end{pmatrix}$, видим что $X_1 = C_1 \begin{pmatrix}
    1 \\
    1 \\
    1
\end{pmatrix}$

Последовательно найдем теперь второй и третий собственный векторы: $X_2$ и $X_3$.

Пусть $\lambda = \lambda_2 = 3$, тогда $(A - \lambda E) * X = 0$

\begin{equation}
    \begin{cases}
        (4-3)x_1 - 2x_2 - x_3 = 0 \\
        -x_1 + (3 - 3)x_2 - x_3 = 0 \\
        x_1 - 2x_2 + (2 - 3)x_3 = 0
    \end{cases}
\end{equation}

Преобразуем в матричный вид: $\begin{pmatrix}
    1 & -2 & -1 \\
    -1 & 0 & 1 \\
    1 & -2 & 1
\end{pmatrix} = \begin{pmatrix}
    1 & -2 & -1 \\ 1 & 0 & 1
\end{pmatrix} = \begin{pmatrix}
    2 & -2 & 0 \\
    1 & 0 & 1
\end{pmatrix} = \begin{pmatrix}
    1 & -1 & 0 \\
    1 & 0 & 1
\end{pmatrix}$

Из этого видно, что $x_1 = x_2, x_1 = -x_3$

$X_2 = C_2 \begin{pmatrix}
    1 \\
    1 \\
    -1
\end{pmatrix}$

Найдем третий собственный вектор.

Пусть $\lambda = \lambda_3 = 5$, тогда $(A - \lambda E) * X = 0$

\begin{equation}
    \begin{cases}
        (4 - 5)x_1 - 2x_2 - x_3 = 0 \\
        -x_1 + (3 - 5)x_2 - x_3 = 0 \\
        x_1 - 2x_2 + (2 - 5)x_3 = 0
    \end{cases}
\end{equation}

Запишем данную систему уравнений в матричном виде: $\begin{pmatrix}
    -1 & -2 & -1 \\
    -1 & -2 & -1 \\
    1 & -2 & -3
\end{pmatrix} = \begin{pmatrix}
    -1 & -2 & -1 \\
    1 & -2 & -3
\end{pmatrix} = \begin{pmatrix}
    -1 & -2 & -1 \\
    0 & -4 & -4
\end{pmatrix}$

Чему соответствует следующая система уравнений:

\begin{equation}
    \begin{cases}
        -x_1 - 2x_2 - x_3 = 0 \\
        -4x_2 - 4x_3 = 0
    \end{cases}
\end{equation}

Я зашел в какую-то фигню, где-то ошибся, но, в общем, ответ должен получиться следующий: $X_3 = C_3 \begin{pmatrix}
    1 \\
    -1 \\
    1
\end{pmatrix}$

\subsection{Векторная алгебра. Операции над векторами}

Вектором $\overrightarrow{AB}$ называется направленный отрезок $AB$, заданный своим началом $A$ и концом $B$.

Длиной (модулем) $|\overrightarrow{AB}|$ вектора $\overrightarrow{AB}$ называется длина отрезка $AB$.

Два вектора называются коллинеарными, если они параллельны одной прямой (параллельны друг другу).

Три вектора называются компланарными, если они параллельны одной плоскости.

Координаты $x, y, z$ вектора $\overrightarrow{a}$ это коэффициенты разложения вектора по базису, то есть по трем некомпланарным векторам, обозначаемым как $\overrightarrow{e_1}, \overrightarrow{e_2}, \overrightarrow{e_3}$.

$\overrightarrow{e_1} = \{1, 0, 0\}, \overrightarrow{e_2} = \{0, 1, 0\}, \overrightarrow{e_2} = \{0, 0, 1\}, \overrightarrow{a} = x * \overrightarrow{e_1} + y * \overrightarrow{e_2} + z * \overrightarrow{e_3}$

Если $\overrightarrow{e_1}, \overrightarrow{e_2}, \overrightarrow{e_3}$ взаимно перпендикулярны и единичные векторы: $\overrightarrow{i}, \overrightarrow{j}, \overrightarrow{k}$, то такой базис называется ортонормированным.

\subsubsection{Пример}

Разложить вектор $\overrightarrow{a} = \{4; 2; 0\}$, если возможно, по векторам $\overrightarrow{p} = \{1; -1; 2\}, \overrightarrow{q} = \{2; 2; -1\}, \overrightarrow{r} = \{3; 7; -7\}$ 

Для того, чтобы это было возможно, должно соблюдаться следующее выражение: $(\overrightarrow{p} * \overrightarrow{q}) * \overrightarrow{r} \ne 0$ - достаточное условие некомпланарности.

$(\overrightarrow{p} * \overrightarrow{q}) * \overrightarrow{r} = \begin{vmatrix}
    1 & -1 & 2 \\
    2 & 2 & -1 \\
    3 & 7 & -7
\end{vmatrix} \ne 0$

$\det X = -14 + 28 + 3 - 12 + 7 - 14 = 2 \ne 0$, следовательно, мы можем разложить данный вектор по трем некомпланарным векторам.

$\overrightarrow{a} = x * \overrightarrow{p} + y * \overrightarrow{q} + z * \overrightarrow{r}$

$\overrightarrow{p} = 1 * \overrightarrow{i} - 1 * \overrightarrow{j} + 2 * \overrightarrow{k}, \overrightarrow{q} = 2 * \overrightarrow{i} + 2 * \overrightarrow{j} - \overrightarrow{k}, \overrightarrow{r} = 3 * \overrightarrow{i} + 7 * \overrightarrow{j} - 7 * \overrightarrow{k}$

$x * \overrightarrow{p} + y * \overrightarrow{q} + z * \overrightarrow{r} = x * \overrightarrow{i} - x * \overrightarrow{j} + 2x\overrightarrow{k} + 2y\overrightarrow{i} + 2y\overrightarrow{i} + 2y\overrightarrow{j} - y\overrightarrow{k} + 3z\overrightarrow{i} + 7z\overrightarrow{j} - 7z\overrightarrow{k}$

Далее для разложения по базису нам необходимо вынести $\overrightarrow{i}, \overrightarrow{j}, \overrightarrow{k}$

$... = (x + 2y + 3z) + \overrightarrow{j}(-x + 2y + 7z) + \overrightarrow{k}(2x - y - 7z)$

$\overrightarrow{a} = x\overrightarrow{p} + y\overrightarrow{q} + z\overrightarrow{r}$

\begin{equation}
    \begin{cases}
        x + 2y + 3z = 4 \\
        -x + 2y + 7z = 2 \\
        2x - y - 7z = 0
    \end{cases}
\end{equation}

Решим данную систему уравнений каким угодно способом, сначала составив расширенную матрицу системы:

$
\begin{pmatrix}
    1 & 2 & 3 & 4 \\
    -1 & 2 & 7 & 2 \\
    2 & -1 & -7 & 0
\end{pmatrix} = \begin{pmatrix}
    1 & 2 & 3 & 4 \\
    1 & -2 & -7 & -2 \\
    1 & 1 & 0 & 2
\end{pmatrix} = \begin{pmatrix}
    1 & 2 & 3 & 4 \\
    1 & 1 & 0 & 2 \\
    0 & -3 & -7 & -4
\end{pmatrix} = \begin{pmatrix}
    1 & -1 & -4 & 0 \\
    1 & 1 & 0 & 2 \\
    0 & -3 & -7 & -4
\end{pmatrix}
$

\begin{equation}
\begin{cases}
    x_1 = -x_2 + 2 \\
    x_3 = \frac{4}{7} - \frac{3}{7}x_2 \\
    x_1 = -x_2-4x_3 = 0
\end{cases}
\end{equation}

\begin{equation}
    \begin{cases}
        x_2 = -1 \\
        x_1 = 3 \\
        x_3 = 1
    \end{cases}
\end{equation}

Тогда $\overrightarrow{a} = 3\overrightarrow{p} - \overrightarrow{q} + \overrightarrow{r}$

\pagebreak
\section{Алгебра и геометрия - 14.10.2022}

\subsection{Центр масс}

\begin{flushleft}

Если точки $A$ и $B$ заданы координатами $A(x_1; y_1; z_1), B(x_2; y_2; z_2)$, то координаты вектора $\overrightarrow{AB}: \{x_2 - x_1; y_2 - y_1; z_2 - z_1\} $.

\hfill

Разделить отрезок в соотношении $\lambda \ne -1$ это значит на прямой $AB$ найти такую точку $M$, что вектор $\overrightarrow{AM} = \lambda \overrightarrow{MB}$.

\hfill 

Если заданы координаты точек $A(x_1;y_1;z_1), B(x_2;y_2;z_2)$, то координаты делящей точки $M(x_m;y_m;z_m)$ находят по формулам: $x_m = \frac{x_1 = \lambda x_2}{1 + \lambda}, y_m = \frac{y_1 + \lambda y_2}{1 + \lambda}, z_m = \frac{z_1 + \lambda z_2}{1 + \lambda}$

\hfill

Если $M$ - середина $AB$, то $\lambda = 1$, а формулы $x_m = \frac{x_1 + x_2}{2}, y_m = \frac{y_1 + y_2}{2}, z_m = \frac{z_1 + z_2}{2}$

\end{flushleft}

\subsubsection{Пример}

\begin{flushleft}

Дано: $A_1(1; 3), m_1 = 10; A_2(7; 8), m_2 = 30; A_3(0; 4), m_3 = 5$. Определить $S$ - центр масс системы.

Пусть $C_1$ делит $A_1A_2$ в соотношении $\lambda = \frac{m_2}{m_1} = 3$, тогда $x_c = \frac{1 + 3 * 7}{4} = \frac{22}{4}, y_c = \frac{27}{4}$

Пусть $S$ делит $CA_3$ в соотношении $\lambda = \frac{m_3}{m_1 + m_2} = \frac{1}{8}$, тогда $x_s = \frac{11}{2} * \frac{8}{9} = \frac{44}{9}, y_s = \frac{\frac{27}{4} + \frac{11}{8} * 4 + \frac{1}{8}}{\frac{9}{8}} = \frac{29}{4} * \frac{8}{9} = \frac{58}{9}$.

Ответ: $S(\frac{44}{9}; \frac{58}{9})$

\end{flushleft}

\subsubsection{Некоторые нюансы}

\begin{flushleft}

1) Можно доказать, что центр масс $S(x_s; y_s; z_s)$ материальной системы точек $A_1(x_1; y_1; z_1), A_2(x_2; y_2; z_2), ..., A_n(x_n; y_n; z_n)$, в которых сосредоточены массы $m_1, m_2, ..., m_n$ имеет следующие координаты:

$x_s = \frac{x_1 * m_1 + x_2 * m_2 + ... + x_n * m_n}{m_1 + m_2 + ... + m_n}, y_s = \frac{y_1 * m_1 + ... + y_n * m_n}{m_1 + ... + m_n}, z_s = \frac{z_1 * m_1 + ... + z_n * m_n}{m_1 + ... + m_n}$

\hfill

2) Центры масс треугольника с координатами $A_1(x_1; y_1; z_1), A_2(x_2; y_2; z_2), A_3(x_3; y_3; z_3)$ (то есть, центр масс однородной треугольной пластины) находится в точке пересечения медиан.

Если предпложить, что $n = 3, m_1 = m_2 = m_3$, то $S(\frac{x_1 + x_2 + x_3}{3}; \frac{y_1 + y_2 + y_3}{3}; \frac{z_1 + z_2 + z_3}{3})$

\end{flushleft}

\subsection{Направляющие косинусы}

\begin{flushleft}

Пусть $\alpha, \beta, \gamma$ - углы, которые образуют $\vec{a} = \{x, z, z\}$ с осями $O_x, O_y, O_z$.

Тогда направляющие косинусы $\cos \alpha, \cos \beta, \cos \gamma$ вектора $\vec{a}$ связаны соотношением $\cos^2 \alpha + \cos^2 \beta + \cos^2 \gamma = 1$ и определяются формулами:

$\cos \alpha = \frac{x}{|\vec{a}|} = \frac{x}{\sqrt{x_2 + y_2 + z_2}}, \cos \beta = \frac{y}{|\vec{a}|}, \cos \gamma = \frac{z}{|\vec{a}|}$

\end{flushleft}

\subsubsection{Пример}

\begin{flushleft}

Найти длину и направляющие косинусы $\vec{AM}$, если т. $M$ делит $AB$ в соотношении $\lambda = -2$, где $A(5; 6; -1), B(0; -3; 2)$.

Найдем координаты точки $M$: $x_m = -5, y_m = -12, z_m = 5$. Таким образом, $M(-5; -12; 5)$.

$\vec{AM} = \{-10; -18; -6\}, |\vec{AM}| = \sqrt{100 + 324 + 36} = \sqrt{460} = 2\sqrt{115}$

Найдем направляющие косинусы: $\cos \alpha = \frac{-10}{2\sqrt{115}} \approx -0.466, \cos \beta = \frac{-18}{2\sqrt{115}} \approx -0.839, \cos \gamma = \frac{-6}{2\sqrt{115}} \approx 0.28$.

Выполним проверку: $\cos^2 \alpha + \cos^2 \beta + \cos^2 \gamma = \frac{100}{460} + \frac{324}{460} + \frac{36}{460} = 1$

\end{flushleft}

\subsection{Решение практической работы, вариант 21}

\subsubsection{Задание 5, нахождение центра тяжести системы}

\begin{flushleft}

Дано: $A_1(5; -4), A_2(0; 2), A_3(6; 6), m_1 = 25, m_2 = 45, m_3 = 15$.

Согласно формуле,
$S_x = \frac{5 * 25 + 0 * 45 + 6 * 15}{25 + 45 + 15} = \frac{215}{85} = \frac{43}{17}$,
$S_y = \frac{-4 * 25 + 2 * 45 + 6 * 15}{25 + 45 + 15} = \frac{80}{85} = \frac{16}{17}$.

Ответ: $S(\frac{43}{17}; \frac{16}{17})$

\end{flushleft}

\subsubsection{Задание 6, нахождение длины и направляющих косинусов}

\begin{flushleft}

Дано: $A(-2; -5), B(4; 1), \lambda = \frac{2}{7}$.

Найдем координаты точки $M$: $M_x = \frac{-2 + 4}{1 + \frac{2}{7}} = \frac{2}{\frac{9}{7}} = \frac{14}{9}, M_y = \frac{-5 + 1}{1 + \frac{2}{7}} = \frac{-4}{\frac{9}{7}} = \frac{-4 * 7}{9} = -\frac{28}{9}$, таким образом $M(\frac{14}{9}; -\frac{28}{9})$

$\vec{AM} = \{\frac{14}{9} + 2; -\frac{28}{9} + 5\} = \{\frac{32}{9}; \frac{17}{9}\}, |\vec{AM}| = \sqrt{\frac{1024}{81} + \frac{289}{81}} = \sqrt{\frac{1313}{81}}$ 

Найдем направляющие косинусы: $\cos \alpha = \frac{\frac{32}{9}}{\sqrt{\frac{1313}{81}}} \approx 0.883, \cos \beta = \frac{\frac{17}{9}}{\sqrt{\frac{1313}{81}}} \approx 0.469$.

Ответ: сами выпишите из того, что написано выше.

\end{flushleft}

\end{document}
