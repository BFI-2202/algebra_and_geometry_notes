\documentclass{article}
\usepackage[utf8]{inputenc}

\usepackage[T2A]{fontenc}
\usepackage[utf8]{inputenc}
\usepackage[russian]{babel}

\usepackage{amsmath}

\def\vec{\ensuremath\overrightarrow}

\title{Алгебра и геометрия}
\author{Лисид Лаконский}
\date{October 2022}

\begin{document}

\maketitle

\tableofcontents
\pagebreak

\section{Алгебра и геометрия - 14.10.2022}

\subsection{Центр масс}

\begin{flushleft}

Если точки $A$ и $B$ заданы координатами $A(x_1; y_1; z_1), B(x_2; y_2; z_2)$, то координаты вектора $\overrightarrow{AB}: \{x_2 - x_1; y_2 - y_1; z_2 - z_1\} $.

\hfill

Разделить отрезок в соотношении $\lambda \ne -1$ это значит на прямой $AB$ найти такую точку $M$, что вектор $\overrightarrow{AM} = \lambda \overrightarrow{MB}$.

\hfill 

Если заданы координаты точек $A(x_1;y_1;z_1), B(x_2;y_2;z_2)$, то координаты делящей точки $M(x_m;y_m;z_m)$ находят по формулам: $x_m = \frac{x_1 = \lambda x_2}{1 + \lambda}, y_m = \frac{y_1 + \lambda y_2}{1 + \lambda}, z_m = \frac{z_1 + \lambda z_2}{1 + \lambda}$

\hfill

Если $M$ - середина $AB$, то $\lambda = 1$, а формулы $x_m = \frac{x_1 + x_2}{2}, y_m = \frac{y_1 + y_2}{2}, z_m = \frac{z_1 + z_2}{2}$

\end{flushleft}

\subsubsection{Пример}

\begin{flushleft}

Дано: $A_1(1; 3), m_1 = 10; A_2(7; 8), m_2 = 30; A_3(0; 4), m_3 = 5$. Определить $S$ - центр масс системы.

Пусть $C_1$ делит $A_1A_2$ в соотношении $\lambda = \frac{m_2}{m_1} = 3$, тогда $x_c = \frac{1 + 3 * 7}{4} = \frac{22}{4}, y_c = \frac{27}{4}$

Пусть $S$ делит $CA_3$ в соотношении $\lambda = \frac{m_3}{m_1 + m_2} = \frac{1}{8}$, тогда $x_s = \frac{11}{2} * \frac{8}{9} = \frac{44}{9}, y_s = \frac{\frac{27}{4} + \frac{11}{8} * 4 + \frac{1}{8}}{\frac{9}{8}} = \frac{29}{4} * \frac{8}{9} = \frac{58}{9}$.

Ответ: $S(\frac{44}{9}; \frac{58}{9})$

\end{flushleft}

\subsubsection{Некоторые нюансы}

\begin{flushleft}

1) Можно доказать, что центр масс $S(x_s; y_s; z_s)$ материальной системы точек $A_1(x_1; y_1; z_1), A_2(x_2; y_2; z_2), ..., A_n(x_n; y_n; z_n)$, в которых сосредоточены массы $m_1, m_2, ..., m_n$ имеет следующие координаты:

$x_s = \frac{x_1 * m_1 + x_2 * m_2 + ... + x_n * m_n}{m_1 + m_2 + ... + m_n}, y_s = \frac{y_1 * m_1 + ... + y_n * m_n}{m_1 + ... + m_n}, z_s = \frac{z_1 * m_1 + ... + z_n * m_n}{m_1 + ... + m_n}$

\hfill

2) Центры масс треугольника с координатами $A_1(x_1; y_1; z_1), A_2(x_2; y_2; z_2), A_3(x_3; y_3; z_3)$ (то есть, центр масс однородной треугольной пластины) находится в точке пересечения медиан.

Если предпложить, что $n = 3, m_1 = m_2 = m_3$, то $S(\frac{x_1 + x_2 + x_3}{3}; \frac{y_1 + y_2 + y_3}{3}; \frac{z_1 + z_2 + z_3}{3})$

\end{flushleft}

\subsection{Направляющие косинусы}

\begin{flushleft}

Пусть $\alpha, \beta, \gamma$ - углы, которые образуют $\vec{a} = \{x, z, z\}$ с осями $O_x, O_y, O_z$.

Тогда направляющие косинусы $\cos \alpha, \cos \beta, \cos \gamma$ вектора $\vec{a}$ связаны соотношением $\cos^2 \alpha + \cos^2 \beta + \cos^2 \gamma = 1$ и определяются формулами:

$\cos \alpha = \frac{x}{|\vec{a}|} = \frac{x}{\sqrt{x_2 + y_2 + z_2}}, \cos \beta = \frac{y}{|\vec{a}|}, \cos \gamma = \frac{z}{|\vec{a}|}$

\end{flushleft}

\subsubsection{Пример}

\begin{flushleft}

Найти длину и направляющие косинусы $\vec{AM}$, если т. $M$ делит $AB$ в соотношении $\lambda = -2$, где $A(5; 6; -1), B(0; -3; 2)$.

Найдем координаты точки $M$: $x_m = -5, y_m = -12, z_m = 5$. Таким образом, $M(-5; -12; 5)$.

$\vec{AM} = \{-10; -18; -6\}, |\vec{AM}| = \sqrt{100 + 324 + 36} = \sqrt{460} = 2\sqrt{115}$

Найдем направляющие косинусы: $\cos \alpha = \frac{-10}{2\sqrt{115}} \approx -0.466, \cos \beta = \frac{-18}{2\sqrt{115}} \approx -0.839, \cos \gamma = \frac{-6}{2\sqrt{115}} \approx 0.28$.

Выполним проверку: $\cos^2 \alpha + \cos^2 \beta + \cos^2 \gamma = \frac{100}{460} + \frac{324}{460} + \frac{36}{460} = 1$

\end{flushleft}

\subsection{Решение практической работы, вариант 21}

\subsubsection{Задание 5, нахождение центра тяжести системы}

\begin{flushleft}

Дано: $A_1(5; -4), A_2(0; 2), A_3(6; 6), m_1 = 25, m_2 = 45, m_3 = 15$.

Согласно формуле,
$S_x = \frac{5 * 25 + 0 * 45 + 6 * 15}{25 + 45 + 15} = \frac{215}{85} = \frac{43}{17}$,
$S_y = \frac{-4 * 25 + 2 * 45 + 6 * 15}{25 + 45 + 15} = \frac{80}{85} = \frac{16}{17}$.

Ответ: $S(\frac{43}{17}; \frac{16}{17})$

\end{flushleft}

\subsubsection{Задание 6, нахождение длины и направляющих косинусов}

\begin{flushleft}

Дано: $A(-2; -5), B(4; 1), \lambda = \frac{2}{7}$.

Найдем координаты точки $M$: $M_x = \frac{-2 + 4}{1 + \frac{2}{7}} = \frac{2}{\frac{9}{7}} = \frac{14}{9}, M_y = \frac{-5 + 1}{1 + \frac{2}{7}} = \frac{-4}{\frac{9}{7}} = \frac{-4 * 7}{9} = -\frac{28}{9}$, таким образом $M(\frac{14}{9}; -\frac{28}{9})$

$\vec{AM} = \{\frac{14}{9} + 2; -\frac{28}{9} + 5\} = \{\frac{32}{9}; \frac{17}{9}\}, |\vec{AM}| = \sqrt{\frac{1024}{81} + \frac{289}{81}} = \sqrt{\frac{1313}{81}}$ 

Найдем направляющие косинусы: $\cos \alpha = \frac{\frac{32}{9}}{\sqrt{\frac{1313}{81}}} \approx 0.883, \cos \beta = \frac{\frac{17}{9}}{\sqrt{\frac{1313}{81}}} \approx 0.469$.

Ответ: сами выпишите из того, что написано выше.

\end{flushleft}

\end{document}
